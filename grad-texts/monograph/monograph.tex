\documentclass[relatorio,nocolorlinks]{inf-ufg}
\usepackage[utf8]{inputenc}
\usepackage[utf8x]{inputenc}
\usepackage{paralist}
\usepackage{algorithmic}
\usepackage{booktabs}
\usepackage{rotating}
\begin{document}
\autor{\'{E}werton Carlos de Ara\'{u}jo Assis}
\autorR{Assis, \'{E}werton Carlos de Ara\'{u}jo}
\titulo{Heur\'{i}sticas e metaheur\'{i}sticas aplicadas ao Problema de Escalonamento Job-Shop}
\subtitulo{Um algoritmo evolucion\'{a}rio h\'{i}brido baseado na fertiliza\c{c}\~{a}o in vitro para solucionar problemas de escalonamento job-shop}
\cidade{Goi\^{a}nia}
\dia{19}
\mes{12}
\ano{2011}
\orientador{Prof. Celso Gon\c{c}alves Camilo J\'{u}nior}
\universidade{Universidade Federal de Goi\'{a}s}
\uni{UFG}
\unidade{Instituto de Inform\'{a}tica}
\concentracao{Otimiza\c{c}\~{a}o e Intelig\^{e}ncia Artificial}
\capa
\publica
\rosto

\begin{aprovacao}
\banca{Anderson da Silva Soares}{Instituto de Inform\'{a}tica -- UFG}
\end{aprovacao}

\chaves{Otimiza\c{c}\~{a}o combinat\'{o}ria, problema de escalonamento job-shop, heur\'{i}sticas, metaheur\'{i}sticas, algoritmos gen\'{e}ticos,
algoritmo evolucion\'{a}rio, computa\c{c}\~{a}o evolucion\'{a}ria}
\begin{resumo}
Existem diversas solu\c{c}\~{o}es, heur\'{i}sticas e metaheur\'{i}sticas, para o problema de escalonamento job-shop. O presente trabalho tem por
objetivo analisar algumas destas solu\c{c}\~{o}es encontradas na literatura, com foco primordial nas metaheur\'{i}sticas e nos padr\~{o}es
encontrados nessas solu\c{c}\~{o}es. O objetivo n\~{a}o \'{e} realizar uma revis\~{a}o de todas as solu\c{c}\~{o}es que foram desenvolvidas, mas as
solu\c{c}\~{o}es recentes e que s\~{a}o preponderantemente referenciadas pela literatura por sua qualidade ou conceitos inovadores apresentados.
A partir das an\'{a}lises feitas, prop\~{o}em-se analisar os efeitos de uma nova solu\c{c}\~{a}o metaheur\'{i}stica, um algoritmo evolucion\'{a}rio
h\'{i}brido baseado na fertiliza\c{c}\~{a}o in vitro, tamb\'{e}m aplicado na mesma classe de problemas. A efici\^{e}ncia e efetividade do
algoritmo desenvolvido \'{e} analisada sobre inst\^{a}ncias geralmente utilizadas pela literatura --- inst\^{a}ncias disponibilizadas pela OR-Library.
A solu\c{c}\~{a}o evolucion\'{a}ria h\'{i}brida baseada na fertiliza\c{c}\~{a}o in vitro apresenta resultados de qualidade, compar\'{a}veis aos
resultados obtidos por outras solu\c{c}\~{o}es encontradas na literatura. Alguns operadores de sele\c{c}\~{a}o e de variabilidade utilizados em
solu\c{c}\~{o}es evolucion\'{a}rias s\~{a}o analizados e obt\'{e}m-se uma an\'{a}lise comparativa de configura\c{c}\~{o}es distintas, corroborando
com o que \'{e} defendido pela literatura sobre a import\^{a}ncia da escolha dos mec\^{a}nismos de sele\c{c}\~{a}o e de variabilidade. A partir de
27 configura\c{c}\~{o}es distintas, 18 inst\^{a}ncias da OR-Library s\~{a}o exercitadas, obtendo-se um comparativo da solu\c{c}\~{a}o
evolucion\'{a}ria h\'{i}brida frente a uma solu\c{c}\~{a}o evolucion\'{a}ria baseada em conceitos can\^{o}nicos.
\end{resumo}

\keys{Combinatorial optimization, Job-shop scheduling problem, Heuristics, Metaheuristics, Genetic Algorithms, Evolutionary algorithm,
Evolutionary computation}
\begin{abstract}{Heuristics and  Metaheuristics applied to the Job-shop scheduling problem: a hybrid evolutionary algorithm based on
in vitro fertilization to solve job-shop scheduling problems}
There are several solutions, heuristics and metaheuristics, to solve the job-shop scheduling problem. The present work aims to analyse some of
these solutions found in literature, with primary focus on metaheuristics and patterns found in these solutions. The goal is not to conduct
a review of all the solutions that have been developed, but the recent ones that are referenced in the literature mainly for their quality and
their innovative concepts presented. From the analysis made​​, we propose to analyze the effects of a new metaheuristic solution, a hybrid
evolutionary algorithm based on in vitro fertilization, also applied in the same class of problems. The efficiency and the effectiveness of the
developed algorithm is analyzed through instances commonly used in the literature --- instances available from OR-Library. The evolutionary hybrid
solution based on in vitro fertilization presents results of quality comparable to results obtained by other solutions in the literature. Some
selection and variability operators used in evolutionary solutions are analyzed and a comparative analysis of different configurations is obtained;
confirming what is advocated by the literature on the importance of the choice of selection mechanisms and variability. From 27 different
configurations, 18 instances of the OR-Library are exercised, resulting in a comparative study between the hybrid evolutionary solution and a
solution based on canonical evolutionary concepts.
\end{abstract}

\tabelas[tabalg]

\chapter{Introdu\c{c}\~{a}o}

Solu\c{c}\~{o}es heur\'{i}sticas e metaheur\'{i}sticas s\~{a}o amplamente utilizadas como meio de solucionar problemas em otimiza\c{c}\~{a}o
\cite{Gendreau2010} \cite{DeJong2006}. Muitas delas s\~{a}o amplamente utilizadas, em suas formas can\^{o}nicas ou h\'{i}bridas, para solucionar
inst\^{a}ncias de problemas de escalonamento job-shop \cite{Chen2011}. Solu\c{c}\~{o}es baseadas em algoritmos evolucion\'{a}rias s\~{a}o comuns
nesse panorama e s\~{a}o frequentemente comparadas a outras abordagens metaheur\'{i}sticas com o fim de legitimar a qualidade das solu\c{c}\~{o}es
propostas ou mesmo evidenciar limita\c{c}\~{o}es e superar defici\^{e}ncias \cite{Gendreau2010}.

Com a evid\^{e}ncia de limita\c{c}\~{o}es e defici\^{e}ncias em algoritmos evolucion\'{a}rios, como o tempo de converg\^{e}ncia e os diversos
efeitos que a aleatoriedade causa sobre a solu\c{c}\~{a}o algor\'{i}tmica, \'{e} comum que novas abordagens, ditas h\'{i}bridas, emerjam com o fim de
superar essas defici\^{e}ncias e explorar qualidades espec\'{i}ficas a cada abordagem algor\'{i}tmica. Assim surge o algoritmo auxiliar
paralelo baseado na fertiliza\c{c}\~{a}o in vitro, inicialmente aplicado em algoritmos gen\'{e}ticos, que tem por finalidade
explorar os mec\^{a}nismos de variabilidade dos algoritmos gen\'{e}ticos e superar as defici\^{e}ncias quanto a efetividade do algoritmo e o
tempo de converg\^{e}ncia do mesmo. Como essa abordagem h\'{i}brida \'{e} nova \cite{Camilo2011}, considera\c{c}\~{o}es pr\'{a}ticas sobre esta
ainda s\~{a}o raras na literatura.

\section{Objetivo}

O presente trabalho tem por objetivo propor uma solu\c{c}\~{a}o algor\'{i}tmica baseada nos algoritmos evolucion\'{a}rios, notavelmente os
algoritmos gen\'{e}ticos e as estrat\'{e}gias evolutivas, em conjunto com o algoritmo auxiliar paralelo baseado na fertiliza\c{c}\~{a}o in vitro,
com o fim de analisar a aplicabilidade dessa abordagem h\'{i}brida em um contexto espec\'{i}fico (problemas de escalonamento job-shop), e os efeitos
que operadores de variabilidade e de sele\c{c}\~{a}o t\^{e}m sobre a mesma.

\section{Organiza\c{c}\~{a}o da monografia}

O texto divide-se em cap\'{i}tulos, sendo o Cap\'{i}tulo \ref{jobshop} respons\'{a}vel por apresentar defini\c{c}\~{o}es e conceitos relacionados
a problemas de escalonamento job-shop; o Cap\'{i}tulo \ref{solucoes} apresenta as defini\c{c}\~{o}es pertinentes a heur\'{i}sticas e
metaheur\'{i}sticas e como estas abordagens est\~{a}o sendo aplicadas para resolver problemas de escalonamento job-shop; o Cap\'{i}tulo \ref{meta}
esbo\c{c}a os conceitos que norteiam a solu\c{c}\~{a}o algor\'{i}tmica proposta; o Cap\'{i}tulo \ref{solucao} apresenta a solu\c{c}\~{a}o
algor\'{i}tmica e as decis\~{o}es de projeto envolvidas; o Cap\'{i}tulo \ref{resultados} apresenta os resultados obtidos a partir da solu\c{c}\~{a}o
algor\'{i}tmica e quais s\~{a}o as considera\c{c}\~{o}es observ\'{a}veis a partir dos resultados; por fim, o Cap\'{i}tulo \ref{conclusao} apresenta
a conclus\~{a}o e trabalhos futuros.

\chapter{Problemas de escalonamento job-shop}
\label{jobshop}

Problemas de escalonamento remontam desde a necessidade de organizar meios de produ\c{c}\~{a}o para melhor satisfazerem a demanda de consumo dos
mercados. A partir da Revolu\c{c}\~{a}o Industrial houve uma crescente necessidade pela sistematiza\c{c}\~{a}o e otimiza\c{c}\~{a}o dos meios de
produ\c{c}\~{a}o, com fins de diminuir o tempo de produ\c{c}\~{a}o e fazer melhor uso de recursos despendidos na manufatura. Problemas de
escalonamento job-shop emergem nesse contexto de manufaturas, no qual tarefas ou produtos (\textit{jobs}) devem ser processados por m\'{a}quinas
a fim de que todas as tarefas sejam conclu\'{i}das. Contudo, problemas de escalonamento job-shop n\~{a}o se prendem a estes contextos de
manufaturas; diversos problemas e contextos podem ser modelados e melhor compreendidos a partir dessa perspectiva.

\section{Defini\c{c}\~{a}o do problema de escalonamento job-shop}

A defini\c{c}\~{a}o formal de um problema de escalonamento job-shop tradicional \cite{French1982} \'{e} feita a seguir: s\~{a}o
dadas $n$ tarefas (ou \textit{jobs}) $ \{ J_{1}, J_{2}, ..., J_{n} \} $ a serem processadas por $m$ m\'{a}quinas $ \{ M_{1}, M_{2}, ..., M_{m} \}
$ --- cada tarefa deve ser processada por cada m\'{a}quina uma \'{u}nica vez. O processamanto de uma tarefa em uma m\'{a}quina \'{e} denominado
opera\c{c}\~{a}o: a opera\c{c}\~{a}o da tarefa $i$ na m\'{a}quina $j$ \'{e} denotada por $o_{ij}$. Restri\c{c}\~{o}es tecnol\'{o}gicas determinam
a ordem de processamento de cada tarefa atrav\'{e}s das $m$ m\'{a}quinas. As tarefas em um problema de escalonamento job-shop n\~{a}o compartilham
uma mesma ordem de processamento atrav\'{e}s das m\'{a}quinas, como acontece em problemas de escalonamento flow-shop, no qual todas as tarefas
compartilham uma mesma ordem de processamento. O ambiente analisado em um problema de escalonamento job-shop \'{e} geralmente denotado por oficina
(\textit{workshop}).

Cada tarefa $i$  ($J_{i}$) apresenta um tempo de processamento para cada m\'{a}quina, i.e. para cada opera\c{c}\~{a}o $o_{ij}$, opera\c{c}\~{a}o
da tarefa $i$ na m\'{a}quina $j$, h\'{a} um tempo de processamento $p_{ij}$. Qualquer tempo de ajuste ou de carregamento (\textit{set-up time})
da m\'{a}quina para processar a opera\c{c}\~{a}o $o_{ij}$ \'{e} inclu\'{i}do em $p_{ij}$. Al\'{e}m do tempo de processamento para cada
opera\c{c}\~{a}o, cada tarefa $i$ ($J_{i}$) pode apresentar um tempo de lan\c{c}amento, denotado por $r_{i}$, que determina a partir de quanto
tempo ou em qual momento a tarefa $J_{i}$ estar\'{a} dispon\'{i}vel para processamento pelas $m$ m\'{a}quinas da oficina.

A partir dessas restri\c{c}\~{o}es e defini\c{c}\~{o}es surge a necessidade de construir um escalonador de tarefas de forma a agendar as $n$
tarefas atrav\'{e}s das $m$ m\'{a}quinas de forma que o tempo total de processamento seja o menor poss\'{i}vel. Al\'{e}m desse crit\'{e}rio de
otimiza\c{c}\~{a}o, outros crit\'{e}rios s\~{a}o usualmente estabelecidos, os quais ser\~{a}o apresentados a seguir. Com base nessa
defini\c{c}\~{a}o cl\'{a}ssica do problema de escalonamento job-shop, outras defini\c{c}\~{o}es tamb\'{e}m foram feitas, como extens\~{a}o, as
quais ser\~{a}o apresentadas em seguida.

O problema de escalonamento job-shop \'{e} considerado um problema numericamente intrat\'{a}vel, NP-dif\'{i}cil (para $m \geq 2$) \cite{French1982}
e que apresenta um limite superior de $(n!)^{m}$ solu\c{c}\~{o}es poss\'{i}veis; portanto, objetivamente, a enumera\c{c}\~{a}o de todas as
solu\c{c}\~{o}es poss\'{i}veis para inst\^{a}ncias com dimens\~{a}o relevante (por exemplo, $10 \times 15$ ou 10 tarefas, 15 m\'{a}quinas) \'{e}
impratic\'{a}vel \cite{Rondon2009}.

\section{Crit\'{e}rios de otimiza\c{c}\~{a}o}

A partir da defini\c{c}\~{a}o preliminar do problema de escalonamento job-shop podemos extend\^{e}-la de forma a obter outros crit\'{e}rios de
otimiza\c{c}\~{a}o. Essa defini\c{c}\~{a}o b\'{a}sica geralmente \'{e} extendida da seguinte forma \cite{French1982} \cite{Rondon2009}:

\begin{description}
\item[$d_{i}$] \'{e} a data de vencimento ou a data de entrega da tarefa $J_{i}$, o tempo ideal para que a tarefa seja completada;
\item[$a_{i}$] \'{e} a janela de tempo no qual a tarefa $J_{i}$ pode ser processada, denotada por $a_{i} = d_{i} - r_{i}$;
\item[$W_{ik}$] \'{e} o tempo de espera da tarefa $J_{i}$ no processamento da opera\c{c}\~{a}o $k$. N\~{a}o necessariamente a $k$-\'{e}sima
opera\c{c}\~{a}o refere-se \`{a} m\'{a}quina $M_{k}$; refere-se \`{a} $k$-\'{e}sima opera\c{c}\~{a}o da ordem de processamento da tarefa
$J_{i}$ (determinada pelas restri\c{c}\~{o}es tecnol\'{o}gicas da inst\^{a}ncia espec\'{i}fica);
\item[$W_{i}$] \'{e} o tempo total de espera da tarefa $J_{i}$, ou $W_{i} = \sum_{k=1}^{m}W_{ik}$;
\item[$C_{i}$] \'{e} o tempo total de processamento da tarefa $J_{i}$, ou $C_{i} = r_{i} + \sum_{k=1}^{m}(W_{ik} + p_{ij(k)})$;
\item[$F_{i}$] \'{e} o tempo total de processamento da tarefa $J_{i}$ sem levar em conta o tempo de lan\c{c}amento desta, i.e. \'{e} o tempo
total em que a tarefa passou pelas m\'{a}quinas e, assim, permaneceu na oficina ($F_{i} = C_{i} - r_{i}$);
\item[$L_{i}$] \'{e} o atraso no processamento da tarefa $J_{i}$ ($L_{i} = C_{i} - d_{i}$). Quando a tarefa \'{e} completada antes do previsto
na data de entrega (antecipa\c{c}\~{a}o), $L_{i}$ ser\'{a} negativo. Portanto, a partir de $L_{i}$ s\~{a}o definidas duas outras vari\'{a}veis
de an\'{a}lise:
$T_{i} = \textit{max}\{C_{i} - d_{i} , 0\}$, o atraso da tarefa $J_{i}$; e
$E_{i} = \textit{max}\{d_{i} - C_{i} , 0\}$, a atencipa\c{c}\~{a}o da tarefa $J_{i}$;
\item[$I_{j}$] \'{e} o tempo de ociosidade da m\'{a}quina $M_{j}$.
\end{description}


% TODO COLOCAR UM DIAGRAMA DE GANTT AQUI


Estas defini\c{c}\~{o}es t\^{e}m como prop\'{o}sito formalizar os crit\'{e}rios de otimiza\c{c}\~{a}o usualmente estabelecidos para problemas de
escalonamento job-shop. O crit\'{e}rio b\'{a}sico, comumente adotado pela literatura em problemas de escalonamento \cite{Chen2011}, tem como objetivo
minimizar o $C_{max}$ (\textit{makespan}), i.e. minimizar o tempo total da tarefa de maior tempo de processamento. Em problemas nos quais as
tarefas n\~{a}o t\^{e}m um tempo de lan\c{c}amento, estas j\'{a} est\~{a}o todas preparadas para serem executadas pelas m\'{a}quinas,
$C_{max} = F_{max}$ , sendo $F_{max}$ o tempo total de processamento da tarefa que permaneceu por mais tempo na oficina (passou por todas as
m\'{a}quinas). Em contexto distinto, no qual algumas tarefas apresentam algum tempo de lan\c{c}amento, podemos ter $C_{max} \neq F_{max}$;
portanto, outro crit\'{e}rio de otimiza\c{c}\~{a}o \'{e} a minimiza\c{c}\~{a}o do valor $F_{max}$, o tempo gasto pelas tarefas na oficina,
referente \`{a} tarefa com maior tempo de processamento pelas m\'{a}quinas. Estes dois crit\'{e}rios de otimiza\c{c}\~{a}o s\~{a}o denotados
como crit\'{e}rios baseados no tempo de conclus\~{a}o das tarefas \cite{French1982}. Outros crit\'{e}rios s\~{a}o apresentados a seguir.

\section{Crit\'{e}rios baseados na data de entrega}

A partir da defini\c{c}\~{a}o de $L_{i}$ alguns crit\'{e}rios de otimiza\c{c}\~{a}o baseados na data de entrega podem ser estabelecidos
\cite{French1982}, notadamente $T_{max}$ e $E_{max}$, respectivamente, o atraso m\'{a}ximo e a antecipa\c{c}\~{a}o m\'{a}xima. A
minimiza\c{c}\~{a}o do valor $T_{max}$ \'{e} apropriada em contextos nos quais o atraso no processamento de tarefas apresenta alguma penalidade.
J\'{a} no caso de minimiza\c{c}\~{a}o do valor $E_{max}$, esta abordagem \'{e} apropriada nos contextos em que h\'{a} um benef\'{i}cio ao
antecipar-se o processamento das tarefas.

Por fim, outra abordagem de minimiza\c{c}\~{a}o relativa \`{a} data de entrega \'{e} a minimiza\c{c}\~{a}o do n\'{u}mero de tarefas atrasadas
($n_{T}$). Esta abordagem \'{e} apropriada em contextos cr\'{i}ticos, nos quais tarefas atrasadas apresentam um alto valor de penalidade.

\section{Crit\'{e}rios baseados em custos}

Em alguns ambientes m\'{a}quinas ociosas podem representar recursos ociosos ou, ainda, o processamento das tarefas est\'{a} sujeito \`{a}
disponibilidade de recursos para cada opera\c{c}\~{a}o. Nestes contextos, outros crit\'{e}rios de otimiza\c{c}\~{a}o surgem. Esses crit\'{e}rios
de otimiza\c{c}\~{a}o levam em conta as seguintes vari\'{a}veis, para um determinado tempo $t$ \cite{French1982}:

\begin{description}
\item[$N_{w(t)}$] o n\'{u}mero de tarefas esperando para serem processadas por alguma m\'{a}quina ou tarefas que ainda n\~{a}o est\~{a}o prontas
para serem processadas;
\item[$N_{p(t)}$] o n\'{u}mero de tarefas sendo processadas no tempo $t$;
\item[$N_{c(t)}$] o n\'{u}mero de tarefas completadas no tempo $t$;
\item[$N_{u(t)}$] o n\'{u}mero de tarefas a serem ainda completadas no tempo $t$.
\end{description}

Assim, alguns crit\'{e}rios de otimiza\c{c}\~{a}o estabelecidos a partir destas vari\'{a}veis s\~{a}o:

\begin{enumerate}
\item minimizar o n\'{u}mero m\'{e}dio de tarefas esperando para serem processadas ($\bar{N}_{w}$) ou minimizar o n\'{u}mero m\'{e}dio de
tarefas n\~{a}o completadas ($\bar{N}_{u}$) de forma a minimizar os custos de invent\'{a}rio;
\item minimizar o n\'{u}mero m\'{e}dio de tarefas completadas ($\bar{N}_{c}$), de forma a reduzir os custos dos bens produzidos;
\item maximizar o n\'{u}mero m\'{e}dio de tarefas sendo processadas ($\bar{N}_{p}$) com o intuito de realizar um uso eficiente das m\'{a}quinas.
Essa maximiza\c{c}\~{a}o do valor $\bar{N}_{p}$ relaciona-se \`{a} minimiza\c{c}\~{a}o do valor $I_{max}$, ou a minimiza\c{c}\~{a}o do tempo
m\'{a}ximo que uma m\'{a}quina fica ociosa.
\end{enumerate}

\section{Pressupostos em problemas de escalonamento job-shop}

Pelas defini\c{c}\~{o}es at\'{e} ent\~{a}o feitas \'{e} poss\'{i}vel identificar a complexidade do problema de escalonamento job-shop e o quanto
este pode se tornar complexo em ambientes reais, nos quais tarefas podem surgir aleatoriamente (ambientes din\^{a}micos) ou mesmo o tempo de
processamento das tarefas pode n\~{a}o ser conhecido \`{a} priori (ambientes estoc\'{a}sticos) \cite{French1982}. Assim, alguns pressupostos
gerais s\~{a}o assumidos ao identificar e analisar problemas de escalonamento job-shop tradicionais \cite{French1982}. Como ser\'{a} apresentado
adiante, esses pressupostos podem ser ignorados ou modificados em extens\~{o}es feitas ao problema de escalonamento job-shop tradicional.

\begin{description}
\item[cada tarefa \'{e} uma entidade] as opera\c{c}\~{o}es de uma determinada tarefa n\~{a}o podem ser processadas simultaneamente, i.e. cada
opera\c{c}\~{a}o constituinte de uma determinada tarefa deve ser processada isoladamente, sem paralelismo, em cada m\'{a}quina;
\item[n\~{a}o existe preemp\c{c}\~{a}o] enquanto uma opera\c{c}\~{a}o \'{e} processada em uma m\'{a}quina, esta continuar\'{a} indispon\'{i}vel
at\'{e} que a opera\c{c}\~{a}o seja conclu\'{i}da --- n\~{a}o \'{e} admiss\'{i}vel nesse contexto a interrup\c{c}\~{a}o do processamento;
\item[cada tarefa \'{e} constitu\'{i}da de $m$ opera\c{c}\~{o}es distintas, uma para cada m\'{a}quina] esse pressuposto \'{e} uma consequ\^{e}ncia
da defini\c{c}\~{a}o anteriormente dada. Assim, temos que uma opera\c{c}\~{a}o (ou, ainda, neste contexto, uma tarefa) n\~{a}o pode ser processada
mais de uma vez em uma determinada m\'{a}quina; e cada tarefa deve ser processada uma \'{u}nica vez em cada m\'{a}quina;
\item[tarefas n\~{a}o s\~{a}o canceladas] cada tarefa deve ser processada pelas m\'{a}quinas;
\item[o tempo de processamento \'{e} independente da agenda constru\'{i}da] o objetivo final do problema de escalonamento job-shop \'{e} a
constru\c{c}\~{a}o de uma agenda na qual as tarefas sejam escalonadas, i.e. uma ordem de processamento das tarefas ser\'{a} definida atendendo os
crit\'{e}rios de otimiza\c{c}\~{a}o estabelecidos e as restri\c{c}\~{o}es envolvidas. Contudo, a agenda definida n\~{a}o influenciar\'{a} os
valores necess\'{a}rios para escalonar as tarefas; assim, o tempo de carregamento da tarefa ou o tempo de prepara\c{c}\~{a}o para o processamento
de uma tarefa n\~{a}o ser\'{a} influenciado pela agenda definida. Esse pressuposto caracteriza essa defini\c{c}\~{a}o tradicional como
independente da sequ\^{e}ncia ou agenda (\textit{sequence-independent});
\item[tarefas podem ter que esperar para serem processadas] este cen\'{a}rio pode ocorrer j\'{a} que, como n\~{a}o \'{e} admiss\'{i}vel a
preemp\c{c}\~{a}o ou o cancelamento de tarefas, e como o tempo de processamento ($p_{ij}$) \'{e} vari\'{a}vel (entre tarefas e m\'{a}quinas
distintas), em alguns contextos tarefas podem ter que esperar para serem processadas em alguma m\'{a}quina ocupada;
\item[existe apenas um \'{u}nico tipo de cada m\'{a}quina] n\~{a}o existe diversas op\c{c}\~{o}es da mesma m\'{a}quina na oficina analisada. Esse
pressuposto elimina a possibilidade de evitar esperas entre as tarefas;
\item[m\'{a}quinas podem estar ociosas] j\'{a} que as diversas tarefas podem n\~{a}o ter conclu\'{i}do seu processamento em alguma das outras
m\'{a}quinas ou est\~{a}o \`{a} espera de outras m\'{a}quinas para serem processadas;
\item[as m\'{a}quinas n\~{a}o podem processar qualquer tarefa mais de uma vez]
\item[as m\'{a}quinas da oficina est\~{a}o sempre dispon\'{i}veis para processamento das tarefas]
\item[as restri\c{c}\~{o}es tecnol\'{o}gicas s\~{a}o imut\'{a}veis e previamente conhecidas] a ordem de processamento das opera\c{c}\~{o}es de
cada tarefa por cada m\'{a}quina \'{e} previamente estabelecida e imut\'{a}vel ao longo do processo de escalonamento;
\item[n\~{a}o existe aleatoriedade] o n\'{u}mero de tarefas, o n\'{u}mero de m\'{a}quinas, o tempo de processamento de cada opera\c{c}\~{a}o, o
tempo de lan\c{c}amento de cada tarefa e outros valores usados para modelar um problema espec\'{i}fico s\~{a}o previamente conhecidos e fixos.
\end{description}

Vale ressaltar que os pressupostos aqui estabelecidos aplicam-se todos ao problema de escalonamento job-shop tradicional, conforme definido em
\cite{French1982}. Sugestivamente, alguns destes pressupostos n\~{a}o s\~{a}o admiss\'{i}veis em contextos e ambientes reais. As defini\c{c}\~{o}es
extendidas, apresentadas a seguir, geralmente modificam, extendem ou eliminam alguns destes pressupostos.

\section{Extens\~{o}es \`{a} defini\c{c}\~{a}o tradicional}

Com base na defini\c{c}\~{a}o tradicional dos problemas de escalonamento job-shop, outras defini\c{c}\~{o}es s\~{a}o feitas extendendo ou
reinterpretando as restri\c{c}\~{o}es e defini\c{c}\~{o}es aplicadas ao problema. Algumas dessas extens\~{o}es s\~{a}o o problema de
escalonamento job-shop flex\'{i}vel (FJSSP) e o problema de escalonamento job-shop multi-objetivo (MOJSSP).

\begin{description}
\item[Problema de escalonamento job-shop flex\'{i}vel (FJSSP)] Essa extens\~{a}o faz uma reinterpreta\c{c}\~{a}o do que representa cada
m\'{a}quina e, consequentemente, aumenta o espa\c{c}o de busca do problema. Al\'{e}m dos crit\'{e}rios de otimiza\c{c}\~{a}o analisados, um
outro objetivo \'{e} acrescentado: al\'{e}m de determinar a ordem de execu\c{c}\~{a}o das tarefas, o escalonador deve determinar em qual
das m\'{a}quinas determinada tarefa ser\'{a} executada em dado momento. Assim, n\~{a}o existe restri\c{c}\~{o}es tecnol\'{o}gicas quanto \`{a}
ordem que as tarefas devem ser processadas pelas diversas m\'{a}quinas. Por consequ\^{e}ncia, o espa\c{c}o de busca \'{e} aumentado.
\cite{Liouane2007} \cite{Saidi2007} \cite{Xing2010} \cite{Zhang2009} \cite{Ma2010} \cite{Zhang2010} \cite{Zhang2010b}

\item[Problema de escalonamento job-shop multi-objetivo (MOJSSP)] Essa extens\~{a}o na verdade \'{e} uma categoriza\c{c}\~{a}o de problemas
de escalonamento job-shop que estabelecem mais de um crit\'{e}rio de otimiza\c{c}\~{a}o no modelo do problema a ser solucionado. Portanto,
existem algumas abordagens que podem ser estabelecidas para guiar a solu\c{c}\~{a}o, como por exemplo estabelecer um sistema de pesos dentre
os diversos crit\'{e}rios a fim de obter uma \'{u}nica fun\c{c}\~{a}o objetiva; ou a abordagem da efici\^{e}ncia \`{a} Pareto na
resolu\c{c}\~{a}o de problemas multi-objetivos. \cite{Adibi2010} \cite{Manikas2009} \cite{Zhang2009} \cite{Zhang2010} \cite{Wang2010}
\end{description}

Estas extens\~{o}es apresentadas n\~{a}o t\^{e}m como prop\'{o}sito serem definitivas, j\'{a} que outras defini\c{c}\~{o}es podem ser obtidas
a partir de outras considera\c{c}\~{o}es sobre os pressupostos e restri\c{c}\~{o}es apresentadas.

\section{Classifica\c{c}\~{a}o das agendas}

Como foi supracitado, o espa\c{c}o de busca do problema de escalonamento job-shop \'{e} consideravelmente grande at\'{e} mesmo para pequenas
inst\^{a}ncias do problema ($m \geq 2$). Assim, \'{e} encontrado na literatura uma categoriza\c{c}\~{a}o das solu\c{c}\~{o}es com base em
alguns dos crit\'{e}rios de otimiza\c{c}\~{a}o analisados. A seguir s\~{a}o apresentadas as categoriza\c{c}\~{o}es mais usuais:

\begin{description}
\item[agendas ativas] s\~{a}o agendas nas quais n\~{a}o \'{e} poss\'{i}vel escalonar tarefas (ou opera\c{c}\~{o}es) para serem processadas
antecipadamente sem violar algumas das restri\c{c}\~{o}es tecnol\'{o}gicas \cite{Brucker2007}. Outra defini\c{c}\~{a}o apresenta agendas ativas como
agendas nas quais n\~{a}o \'{e} poss\'{i}vel ter ao menos uma tarefa terminando antecipadamente e nenhuma tarefa atrasada \cite{Pinedo2008};
\item[agendas semiativas] s\~{a}o agendas nas quais existe tempo ocioso e que n\~{a}o pode ser eliminado, i.e. algumas m\'{a}quinas estar\~{a}o
ociosas \cite{Rondon2009}. Outra defini\c{c}\~{a}o afirma que agendas semiativas s\~{a}o aquelas nas quais tarefas (ou opera\c{c}\~{o}es) n\~{a}o
podem ser processadas antecipadamente sem modificar a ordem de processamento ou violar as restri\c{c}\~{o}es tecnol\'{o}gicas \cite{Brucker2007}
\cite{Pinedo2008};
\item[agendas sem atraso] s\~{a}o agendas nas quais n\~{a}o existe tempo ocioso, i.e. m\'{a}quinas nunca est\~{a}o ociosas \cite{Rondon2009};
\item[agendas \'{o}timas] as agendas \'{o}timas s\~{a}o parte de um subconjunto das agendas sem atraso e que minimizam o tempo de processamento
de todas as tarefas, i.e. minimizam o tempo da tarefa com maior tempo de processamento ($C_{max}$) \cite{Rondon2009}.
\end{description}

Essa categoriza\c{c}\~{a}o tem como objetivo clarificar quais s\~{a}o os objetivos a serem almejados em uma solu\c{c}\~{a}o proposta, al\'{e}m
de delimitar o espa\c{c}o de busca que est\'{a} sendo usado como ambiente de busca e quais as agendas que podem ser efetivamente constru\'{i}das
em determinada abordagem de solu\c{c}\~{a}o.

\chapter{Solu\c{c}\~{o}es heur\'{i}sticas e metaheur\'{i}sticas}
\label{solucoes}

O termo heur\'{i}stica origina-se do termo grego $\epsilon\upsilon\rho\acute{\iota}\sigma\kappa\omega$ (\textit{heurisk\={o}}), que significa
``a arte de descobrir novas estrat\'{e}gias (...) para solucionar problemas'' (tradu\c{c}\~{a}o livre) \cite{Talbi2009}. Dada a complexidade do
problema de escalonamento job-shop, por exemplo, os m\'{e}todos de solu\c{c}\~{a}o trivialmente utilizados para outros problemas similares, ou o
conhecimento utilizado para resolver outros problemas semelhantemente complexos (como a classe de problemas combinatoriais), n\~{a}o s\~{a}o
facilmente empregados neste tipo de problema, dada a sua complexidade e caracter\'{i}sticas pr\'{o}prias e o fato de n\~{a}o ser solucion\'{a}vel
em tempo polinomial (dado que o problema de escalonamento job-shop \'{e} NP-dif\'{i}cil). Portanto, heur\'{i}sticas s\~{a}o desenvolvidas --- a
partir da an\'{a}lise das caracter\'{i}sticas do problema e de como contornar a necessidade de enumerar e verificar as solu\c{c}\~{o}es
poss\'{i}veis no espa\c{c}o de busca --- com o fim de apresentar solu\c{c}\~{o}es aceit\'{a}veis em tempo tamb\'{e}m aceit\'{a}vel. Geralmente
s\~{a}o feitas an\'{a}lises das int\^{a}ncias do problema, e os resultados obtidos atrav\'{e}s destas servem como est\'{i}mulo para melhorar a
solu\c{c}\~{a}o heur\'{i}stica proposta. Solu\c{c}\~{o}es heur\'{i}sticas s\~{a}o geralmente vistas como prop\'{i}cias a serem enganadas por
\'{o}timos locais \cite{Glover1986}.

Michel Gendreau, no livro \textit{Handbook of Metaheuristics}, apresenta metaheur\'{i}sticas como ``m\'{e}todos de solu\c{c}\~{a}o que orquestram
uma intera\c{c}\~{a}o entre procedimentos de melhora local e estrat\'{e}gias de alto n\'{i}vel com o fim de criar um processo capaz de escapar
de \'{o}timos locais e realizar uma busca robusta em um espa\c{c}o de busca'' (tradu\c{c}\~{a}o livre) \cite{Gendreau2010}. Metaheur\'{i}sticas
``prov\^{e}m resultados aceit\'{a}veis, em um tempo razo\'{a}vel, como solu\c{c}\~{a}o de problemas dif\'{i}ceis e complexos, nas Ci\^{e}ncias
e na Engenharia'' (tradu\c{c}\~{a}o livre) \cite{Talbi2009}. O prefixo \textit{meta} aplicado ao termo heur\'{i}stica denota que estas
solu\c{c}\~{o}es s\~{a}o como modelos para que solu\c{c}\~{o}es heur\'{i}sticas sejam facilmente constru\'{i}das a partir de modifica\c{c}\~{o}es
m\'{i}nimas nas metaheur\'{i}sticas. O termo metaheur\'{i}stica foi inicialmente apresentado por Glover \cite{Glover1986} \cite{Talbi2009}.

Metaheur\'{i}sticas auxiliam na constru\c{c}\~{a}o de solu\c{c}\~{o}es para diversos problemas complexos por apresentarem um arcabou\c{c}o geral
e flex\'{i}vel (a partir de estrat\'{e}gias de alto n\'{i}vel), que tornam a customiza\c{c}\~{a}o da solu\c{c}\~{a}o para necessidades
espec\'{i}ficas do problema relativamente simples (procedimentos de melhora local, como supracitado). Estas caracter\'{i}sticas, analisadas a
seguir, tornaram as metaheur\'{i}sticas ferramentas importantes para solucionar problemas complexos e que apresentam inst\^{a}ncias de tamanho
consider\'{a}vel.

\section{M\'{e}todos e abordagens de otimiza\c{c}\~{a}o aplicados ao problema de escalonamento job-shop}

V\'{a}rias abordagens t\^{e}m sido utilizadas para apresentar solu\c{c}\~{o}es ao problema de escalonamento job-shop \cite{Chen2011}. Esses
m\'{e}todos de resolu\c{c}\~{a}o s\~{a}o usualmente caracterizados como m\'{e}todos exatos, m\'{e}todos com regras de prioridade, m\'{e}todos
heur\'{i}sticos e metaheur\'{i}sticos e m\'{e}todos da intelig\^{e}ncia artificial \cite{Chen2011}. Al\'{e}m do uso isolado desses m\'{e}todos,
\'{e} comum o uso h\'{i}brido destes com o prop\'{o}sito de melhorar uma abordagem espec\'{i}fica \cite{Gendreau2010}. As formas h\'{i}bridas
geralmente t\^{e}m como forma a combina\c{c}\~{a}o de um m\'{e}todo de busca global acompanhado por um m\'{e}todo de busca local \cite{Chen2011}.

Realizar uma revis\~{a}o sistem\'{a}tica e extensiva dos m\'{e}todos de resolu\c{c}\~{a}o atualmente empregados n\~{a}o \'{e} o fim deste
trabalho; contudo, a partir da an\'{a}lise das heur\'{i}sticas e metaheur\'{i}sticas empregadas e citadas na literatura, os m\'{e}todos de
resolu\c{c}\~{a}o analisados a seguir s\~{a}o aqueles citados como m\'{e}todos heur\'{i}sticos e metaheur\'{i}sticos, foco primordial do presente
trabalho.

\subsection{M\'{e}todos heur\'{i}sticos e metaheur\'{i}sticos}

Existem atualmente v\'{a}rias abordagens heur\'{i}sticas e metaheur\'{i}sticas com o intuito de apresentar solu\c{c}\~{o}es para o problema de
escalonamento job-shop \cite{Chen2011}. Algumas heur\'{i}sticas e metaheur\'{i}sticas not\'{a}veis s\~{a}o: algoritmos gen\'{e}ticos
\cite{Goncalves2002} \cite{Ferrolho2007} \cite{Huang2010} \cite{Liu2011} \cite{Ma2010} \cite{Manikas2009} \cite{Jinghua2010} \cite{Rondon2009}
\cite{Xiaomei2010} \cite{Yin2007} \cite{Zhang2010b} \cite{Zhang2010}, algoritmos mem\'{e}ticos \cite{Gao2011}, \textit{particle swarm
optimization} \cite{Zhang2009} \cite{Lin2010}, \textit{simulated annealing} \cite{Lin2010}, col\^{o}nia de formiga (\textit{ant colony optmization})
\cite{Liouane2007} \cite{Rondon2009} \cite{Xing2010}, \textit{variable neighborhood search} \cite{Adibi2010} \cite{Roshanaei2009} e busca tabu
(\textit{tabu search}) \cite{Saidi2007} \cite{Zhang2010}. Atualmente, \'{e} not\'{a}vel a tend\^{e}ncia de uso h\'{i}brido dessas heur\'{i}sticas
e metaheur\'{i}sticas \cite{Gendreau2010} com o intuito de explorar as vantagens e minimizar as desvantagens de cada m\'{e}todo \cite{Chen2011}.

O uso de m\'{e}todos heur\'{i}sticos e metaheur\'{i}sticos, tamb\'{e}m denominados m\'{e}todos aproximados \cite{Rondon2009}, em contrapartida aos
m\'{e}todos exatos (ou m\'{e}todos de otimiza\c{c}\~{a}o cl\'{a}ssicos, como \textit{branch-and-bound} \cite{French1982} \cite{Artigues2009},
\textit{cutting-plane} e programa\c{c}\~{a}o inteira \cite{French1982}) tem como motivador a qualidade das solu\c{c}\~{o}es obtidas, embora n\~{a}o
garantidamente \'{o}timas. Portanto, s\~{a}o algoritmos relativamente efetivos, al\'{e}m de serem solu\c{c}\~{o}es eficientes, quando comparado
aos m\'{e}todos exatos \cite{Chen2011}. Os m\'{e}todos exatos s\~{a}o ineficientes, embora efetivos, na resolu\c{c}\~{a}o de inst\^{a}ncias de
problemas com dimens\~{a}o consider\'{a}vel (por exemplo, $15 \times 15$ ou 15 tarefas e 15 m\'{a}quinas) \cite{Chen2011}.

Os m\'{e}todos aproximados s\~{a}o geralmente classificados como t\'{e}cnicas construtivas ou de busca local \cite{Rondon2009}. Aqueles t\^{e}m o
prop\'{o}sito de obter uma solu\c{c}\~{a}o final a partir de um espa\c{c}o de solu\c{c}\~{a}o vazio e, iterativamente, melhorar essa
solu\c{c}\~{a}o inicial (ou solu\c{c}\~{o}es iniciais, no contexto das abordagens populacionais). A busca local tem como prop\'{o}sito melhorar
uma solu\c{c}\~{a}o completa inicial na tentativa de obter uma solu\c{c}\~{a}o ainda melhor. A performance e qualidade dos m\'{e}todos
aproximados \'{e} geralmente analisada a partir dos resultados obtidos experimentalmente \cite{DeJong2006} \cite{Rondon2009}.

\subsection{Padr\~{o}es identificados em solu\c{c}\~{o}es}

Alguns padr\~{o}es na constru\c{c}\~{a}o de solu\c{c}\~{o}es heur\'{i}sticas e metaheur\'{i}sticas foram identificados a partir das an\'{a}lises
feitas com os artigos selecionados:

\begin{enumerate}

\item Ainda que existam tentativas de propor novas formas de representa\c{c}\~{a}o das solu\c{c}\~{o}es que os m\'{e}todos metaheur\'{i}sticos
trabalhar\~{a}o, como uma codifica\c{c}\~{a}o cromoss\^{o}mica tridimensional ou matricial para algoritmos gen\'{e}ticos \cite{Jinghua2010}
\cite{Yin2007} \cite{Zhang2010b}, obt\'{e}m-se resultados de qualidade evidente, eficiente e relativamente efetiva com uma representa\c{c}\~{a}o
cromoss\^{o}mica a partir de chaves aleat\'{o}rias, ou ostensivamente conhecidas como \textit{random-keys} \cite{Adibi2010} \cite{Goncalves2002}
\cite{Roshanaei2009} \cite{Lin2010}. A principal vantagem da representa\c{c}\~{a}o por chaves aleat\'{o}rias \'{e} que todas as
solu\c{c}\~{o}es--indiv\'{i}duos, do m\'{e}todo metaheur\'{i}stico populacional (algoritmo gen\'{e}tico, por exemplo), s\~{a}o indiv\'{i}duos
v\'{a}lidos em todas as aplica\c{c}\~{o}es dos operadores de recombina\c{c}\~{a}o e muta\c{c}\~{a}o; nenhum esfor\c{c}o adicional \'{e}
necess\'{a}rio para tornar uma solu\c{c}\~{a}o-indiv\'{i}duo v\'{a}lida \cite{Goncalves2002}. Conquanto exista o esfor\c{c}o computacional extra
para decodificar a representa\c{c}\~{a}o da solu\c{c}\~{a}o para uma agenda (no caso do problema de escalonamento job-shop), com determinada ordem
de processamento das opera\c{c}\~{o}es das tarefas. Essa representa\c{c}\~{a}o, ou codifica\c{c}\~{a}o da solu\c{c}\~{a}o, tem sido utilizada sob
o t\'{i}tulo de ``representa\c{c}\~{a}o baseada em opera\c{c}\~{a}o'' (\textit{operation-based representation; a set of ordered operations;
operation order representation}) justamente por centrar-se na ordem de processamento das opera\c{c}\~{o}es para cada tarefa da oficina
\cite{Roshanaei2009} \cite{Xiaomei2010} \cite{Zhang2009} \cite{Saidi2007} \cite{Rondon2009}.

\item O uso de opera\c{c}\~{o}es de permuta\c{c}\~{a}o como operador de busca local ou como operador de muta\c{c}\~{a}o em algoritmos
gen\'{e}ticos, ou demais solu\c{c}\~{o}es metaheur\'{i}sticas, \'{e} relativamente comum nas solu\c{c}\~{o}es analisadas \cite{Lin2010}
\cite{Zhang2010} \cite{Xiaomei2010} \cite{Wang2010} \cite{Manikas2009} \cite{Zhang2009} \cite{Adibi2010} \cite{Roshanaei2009}. Os operadores de
permuta\c{c}\~{a}o s\~{a}o utilizados por tratar-se de um problema combinatorial e seu uso \'{e} suscitado pela representa\c{c}\~{a}o usualmente
escolhida. Contudo, \'{e} relativamente comum que algumas solu\c{c}\~{o}es optem por algoritmos de constru\c{c}\~{a}o de agendas
(\textit{constructive algoritms}) como operador de busca local \cite{Goncalves2002}; ou mesmo como um m\'{e}todo isolado de resolu\c{c}\~{a}o, embora
relativamente ineficaz \cite{French1982}. Os operadores de permuta\c{c}\~{a}o encontrados s\~{a}o:
\begin{inparaenum}[\itshape a\upshape)]
\item inser\c{c}\~{a}o;
\item invers\~{a}o (geralmente de elementos adjacentes);
\item permuta\c{c}\~{a}o ou troca de elementos em posi\c{c}\~{o}es distintas; e
\item movimento de longa dist\^{a}ncia (mover uma sequ\^{e}ncia de elementos a uma determinada posi\c{c}\~{a}o ou dire\c{c}\~{a}o).
\end{inparaenum}
Esses operadores s\~{a}o encontrados com nomes distintos em alguns artigos, mas pela defini\c{c}\~{a}o de suas opera\c{c}\~{o}es \'{e}
poss\'{i}vel determinar qual operador est\'{a} sendo utilizado.

\item M\'{e}todos metaheur\'{i}sticos em suas vers\~{o}es can\^{o}nicas s\~{a}o raramente utilizados como solu\c{c}\~{a}o de qualidade para problemas
de escalonamento job-shop --- com base nas solu\c{c}\~{o}es analisadas atrav\'{e}s dos artigos selecionados. As vers\~{o}es can\^{o}nicas de
m\'{e}todos metaheur\'{i}sticos s\~{a}o geralmente utilizadas como m\'{e}todo comparativo para os m\'{e}todos h\'{i}bridos, estabelecendo-se uma
base de compara\c{c}\~{a}o para as solu\c{c}\~{o}es que s\~{a}o propostas como extens\~{o}es, ou h\'{i}bridas.

\item Conquanto algumas solu\c{c}\~{o}es sejam avaliadas a partir de inst\^{a}ncias do problema de escalonamento job-shop constru\'{i}das
especificamente para a solu\c{c}\~{a}o proposta, ou obtidas de oficinas reais \cite{Xing2010} \cite{Liouane2007} \cite{Zhang2009} \cite{Saidi2007}
\cite{Roshanaei2009} \cite{Zhang2010b} \cite{Yin2007} \cite{Manikas2009} \cite{Ma2010}, as 82 inst\^{a}ncias dispon\'{i}veis na OR-Library
\cite{OrLibrary} e as 80 inst\^{a}ncias de Taillard \cite{Taillard1993} s\~{a}o frequentemente utilizadas como meios para estabelecer-se a qualidade
da solu\c{c}\~{a}o proposta frente aos resultados obtidos por outras solu\c{c}\~{o}es \cite{Rego2009} \cite{Pardalos2006} \cite{Kammer2011}
\cite{Ferrolho2007} \cite{Huang2010} \cite{Xiaomei2010} \cite{Zhang2010} \cite{Lin2010} \cite{Goncalves2002}. O uso de inst\^{a}ncias reconhecidas
pela literatura garante que novas solu\c{c}\~{o}es propostas sejam comparadas com outras solu\c{c}\~{o}es j\'{a} estabelecidas.

\end{enumerate}

\chapter{Princ\'{i}pios de solu\c{c}\~{a}o metaheur\'{i}stica evolucion\'{a}ria}
\label{meta}

Problemas em otimiza\c{c}\~{a}o s\~{a}o uma classe de problemas computacionais n\~{a}o facilmente trat\'{a}veis, de forma que
v\'{a}rias solu\c{c}\~{o}es baseadas em metaheur\'{i}sticas foram criadas com o fim de melhor solucion\'{a}-las \cite{Gendreau2010}, quando
aplic\'{a}vel. Uma das metaheur\'{i}sticas mais utilizadas como meio para solucionar problemas em otimiza\c{c}\~{a}o s\~{a}o os algoritmos
gen\'{e}ticos; seja em sua forma can\^{o}nica, desenvolvida por Holland \cite{Holland1992} a partir de trabalhos iniciais de outros autores
\cite{Engelbrecht2007}, ou em solu\c{c}\~{o}es h\'{i}bridas e conceitualmente modificadas \cite{Gendreau2010}. A solu\c{c}\~{a}o algor\'{i}tmica
apresentada neste trabalho tem por base os algoritmos gen\'{e}ticos, al\'{e}m de conceitos encontrados em outros algoritmos evolucion\'{a}rios.
Al\'{e}m disso, a solu\c{c}\~{a}o proposta \'{e} conflitada com a vers\~{a}o can\^{o}nica dos algoritmos gen\'{e}ticos encontrada na literatura,
com alguns conceitos modificados \cite{DeJong2006} \cite{Engelbrecht2007}, como a representa\c{c}\~{a}o e aplicabilidade dos operadores de
variabilidade.

Ao lado da programa\c{c}\~{a}o evolucion\'{a}ria e das estrat\'{e}gias evolutivas, os algoritmos gen\'{e}ticos fazem parte um conceito ainda
maior: algoritmos evolucion\'{a}rios \cite{DeJong2006}. Embora tenham uma origem temporal em comum, inicialmente os algoritmos evolucion\'{a}rios
tentavam solucionar classes de problemas espec\'{i}ficos; portanto, algumas diverg\^{e}ncias conceituais, mesmo que m\'{i}nimas, eram comuns
quando os algoritmos evolucion\'{a}rios eram aplicados \cite{DeJong2006}. Com o advento do termo algoritmos evolucion\'{a}rios e do arcabou\c{c}o
propiciado pela intelig\^{e}ncia computacional, as formas h\'{i}bridas dos algoritmos evolucion\'{a}rios passaram a coexistir \`{a}s vers\~{o}es
can\^{o}nicas daqueles algoritmos \cite{DeJong2006}. Portanto, embora a solu\c{c}\~{a}o proposta neste trabalho seja baseada em algoritmos
gen\'{e}ticos, conceitos dos demais algoritmos evolucion\'{a}rios, em especial das estrat\'{e}gias evolutivas, s\~{a}o levados em
considera\c{c}\~{a}o ao propor uma nova solu\c{c}\~{a}o algor\'{i}tmica.

As solu\c{c}\~{o}es baseadas em algoritmos evolucion\'{a}rios se distinguem das demais solu\c{c}\~{o}es metaheur\'{i}sticas por serem algoritmos
baseados na representa\c{c}\~{a}o da solu\c{c}\~{a}o do problema como um indiv\'{i}duo (solu\c{c}\~{a}o--indiv\'{i}duo) e por influenciar-se de
conceitos biol\'{o}gicos como evolu\c{c}\~{a}o, aptid\~{a}o, recombina\c{c}\~{a}o e muta\c{c}\~{a}o gen\'{e}tica e desenvolvimento geracional de
uma popula\c{c}\~{a}o.

\section{Algoritmos gen\'{e}ticos}

A partir de uma popula\c{c}\~{a}o de indiv\'{i}duos--solu\c{c}\~{o}es aleatoriamente criada, na qual cada indiv\'{i}duo \'{e} representado por
uma cadeia de \textit{bits}, os algoritmos gen\'{e}ticos em sua vers\~{a}o can\^{o}nica desenvolvem a seguinte traget\'{o}ria evolutiva sobre uma
popula\c{c}\~{a}o:
\begin{enumerate}
\item s\~{a}o selecionados indiv\'{i}duos--solu\c{c}\~{o}es da popula\c{c}\~{a}o corrente --- atrav\'{e}s de um mecanismo de sele\c{c}\~{a}o de
indiv\'{i}duos que leve em considera\c{c}\~{a}o a aptid\~{a}o (\textit{fitness}) dos indiv\'{i}duos daquela popula\c{c}\~{a}o --- para gerar novos
descendentes;
\item novos indiv\'{i}duos s\~{a}o gerados a partir de um mecanismo de recombina\c{c}\~{a}o gen\'{e}tica (\textit{crossover});
\item o novo indiv\'{i}duo gerado \'{e} mutado a partir de um mecanismo de muta\c{c}\~{a}o;
\item a popula\c{c}\~{a}o corrente \'{e} substitu\'{i}da pelos descendentes gerados, dando origem a uma nova gera\c{c}\~{a}o.
\end{enumerate}

Essa trajet\'{o}ria evolutiva geralmente \'{e} mantida at\'{e} que um n\'{u}mero limite de gera\c{c}\~{o}es seja alcan\c{c}adas, embora outras
abordagens podem ser determinadas, como um limite temporal, ou mesmo como aquela popula\c{c}\~{a}o evolui temporalmente --- se h\'{a} ganho
ou n\~{a}o a cada ciclo geracional. Um meio de mensura\c{c}\~{a}o da aptid\~{a}o dos indiv\'{i}duos deve ser provido ao algoritmo, de forma que
o algoritmo consiga estabelecer alguns par\^{a}metros de an\'{a}lise e compara\c{c}\~{a}o dos indiv\'{i}duos a cada gera\c{c}\~{a}o. Esse meio
de mensura\c{c}\~{a}o da aptid\~{a}o \'{e} geralmente obtido a partir da defini\c{c}\~{a}o do problema a ser tratado.

Assim, os algoritmos gen\'{e}ticos, como uma solu\c{c}\~{a}o metaheur\'{i}stica, tem como finalidade ser uma solu\c{c}\~{a}o para problemas de
otimiza\c{c}\~{a}o de f\'{a}cil implementa\c{c}\~{a}o e utiliza\c{c}\~{a}o: a partir de uma representa\c{c}\~{a}o bin\'{a}ria das solu\c{c}\~{o}es
e de um meio de mensura\c{c}\~{a}o da aptid\~{a}o dos indiv\'{i}duos --- uma forma de se estabelecer a qualidade daquela
solu\c{c}\~{a}o--indiv\'{i}duo em quest\~{a}o ---, o algoritmo gen\'{e}tico constr\'{o}i um conjunto de solu\c{c}\~{o}es a fim de obter novos
indiv\'{i}duos (solu\c{c}\~{o}es), percorrendo um determinado espa\c{c}o de busca, a partir da representa\c{c}\~{a}o dos indiv\'{i}duos daquela
popula\c{c}\~{a}o.

\section{Estrat\'{e}gias evolutivas}

Diferentemente dos algoritmos gen\'{e}ticos que surgiram como uma solu\c{c}\~{a}o metaheur\'{i}stica para diversos problemas de otimiza\c{c}\~{a}o
--- bastava prover ao algoritmo uma forma de mapear a representa\c{c}\~{a}o do indiv\'{i}duo de uma solu\c{c}\~{a}o para um problema espec\'{i}fico
para uma representa\c{c}\~{a}o em cadeia bin\'{a}ria; e uma fun\c{c}\~{a}o de mensura\c{c}\~{a}o da qualidade/aptid\~{a}o daquele indiv\'{i}duo
(\textit{fitness}) ---, as estrat\'{e}gias evolutivas surgiram como uma metaheur\'{i}stica espec\'{i}fica para solucionar problemas de
otimiza\c{c}\~{a}o de vari\'{a}veis cont\'{i}nuas \cite{DeJong2006}. Indiv\'{i}duos s\~{a}o representados atrav\'{e}s de vetores de valores reais,
nos quais cada dimens\~{a}o armazena o valor de uma das vari\'{a}veis a serem otimizadas. Algoritmos gen\'{e}ticos, em seus conceitos e
considera\c{c}\~{o}es can\^{o}nicos, t\^{e}m suas defici\^{e}ncias ao tratar de otimiza\c{c}\~{a}o de n\'{u}meros reais, j\'{a} que, na
representa\c{c}\~{a}o dos indiv\'{i}duos, sempre limita-se a capacidade representativa do espa\c{c}o de busca.

Outras diferen\c{c}as conceituais s\~{a}o not\'{a}veis \cite{DeJong2006}:
\begin{inparaenum}[\itshape 1\upshape)]
\item o mecanismo ou operador de variabilidade das estrat\'{e}gias evolutivas baseava-se na reprodu\c{c}\~{a}o assexuada; a cada gera\c{c}\~{a}o
novos indiv\'{i}duos eram gerados a partir de muta\c{c}\~{o}es, obrigatoriamente. Diferentemente dos algoritmos gen\'{e}ticos, que tinham um
mecanismo de reprodu\c{c}a\~{o} sexuado (recombina\c{c}\~{a}o) e um mecanismo secund\'{a}rio de muta\c{c}\~{a}o (conceitualmente assexuado).
Ambos os mec\^{a}nismos de variabilidade dos algoritmos gen\'{e}ticos s\~{a}o tratados como aplic\'{a}veis ou n\~{a}o na gera\c{c}\~{a}o de um
novo indiv\'{i}duo;
\item h\'{a} um foco importante no controle do \textit{step-size}, ou \`{a} maneira que o algoritmo percorre o espa\c{c}o de busca, se ser\'{a}
dado passos largos ou curtos nesse espa\c{c}o de busca --- controle feito a partir do operador de variabilidade (muta\c{c}\~{a}o);
\item de in\'{i}cio as estrat\'{e}gias evolutivas n\~{a}o eram uma metaheur\'{i}stica populacional, mas centrada num \'{u}nico indiv\'{i}duo,
que dava origem a v\'{a}rios indiv\'{i}duos, mas apenas um, o melhor dentre os indiv\'{i}duos gerados e o pai, era selecionado para continuar
existindo na pr\'{o}xima gera\c{c}\~{a}o. Posteriormente, verificou-se a import\^{a}ncia do conceito populacional (``busca paralela adaptativa''
\cite{DeJong2006}).
\end{inparaenum}

A principal influ\^{e}ncia das estrat\'{e}gias evolutivas na concep\c{c}\~{a}o da solu\c{c}\~{a}o algor\'{i}tmica apresentada neste trabalho
trata-se na import\^{a}ncia do mecanismo de muta\c{c}\~{a}o dos indiv\'{i}duos gerados; na import\^{a}ncia que existe em sempre pertubar aquelas
solu\c{c}\~{o}es de forma a gerar novas possibilidades de an\'{a}lise no espa\c{c}o de busca; e na import\^{a}ncia que \'{e} dado ao melhor
indiv\'{i}duo que j\'{a} existiu dentre as popula\c{c}\~{o}es, conceito empregado nos algoritmos gen\'{e}ticos sob o t\'{i}tulo de ``elitismo''.
Considera\c{c}\~{o}es a respeito destes conceitos s\~{a}o dados com maiores detalhes quando a solu\c{c}\~{a}o algor\'{i}tmica for apresentada,
no Cap\'{i}tulo \ref{solucao}.

\section{Hibridiza\c{c}\~{a}o em algoritmos evolucion\'{a}rios}

A origem dos algoritmos evolucion\'{a}rios remonta \`{a} d\'{e}cada de 1960, quando conceitos evolutivos foram implementados em algoritmos como
forma de solucionar problemas de otimiza\c{c}\~{a}o \cite{DeJong2006}. Cada algoritmo evolucion\'{a}rio surge em um contexto espec\'{i}fico de
utiliza\c{c}\~{a}o e com caracter\'{i}sticas pr\'{o}prias que os tornaram boas \textit{metasolu\c{c}\~{o}es} para determinadas classes de problemas
em otimiza\c{c}\~{a}o \cite{DeJong2006} \cite{Gendreau2010}. Assim, nas d\'{e}cadas de 1980 e 1990 surge uma expans\~{a}o do uso destes algoritmos
para outras classes de problemas, o que nem sempre resultava em solu\c{c}\~{o}es de qualidade. Outrossim, novas abordagens metaheur\'{i}sticas
tamb\'{e}m surgiram, o que contribuiu para a origem das formas h\'{i}bridas de solu\c{c}\~{a}o metaheur\'{i}sticas \cite{DeJong2006}
\cite{Gendreau2010}.

As solu\c{c}\~{o}es h\'{i}bridas surgem como uma forma de melhor aproveitar a efici\^{e}ncia e efetividade das metaheur\'{i}sticas, em foco os
algoritmos evolucion\'{a}rios, como uma forma de melhor balancear as caracter\'{i}sticas de explora\c{c}\~{a}o do espa\c{c}o de busca
(\textit{exploration}) e realizar melhoras discretas nos indiv\'{i}duos--solu\c{c}\~{o}es desenvolvidos nas solu\c{c}\~{o}es evolucion\'{a}rias
(\textit{exploitation}) \cite{Camilo2010} \cite{Gendreau2010}. Assim, operadores ou mecanismos de variabilidade s\~{a}o criados e conceitos
adaptados para melhor aproveitar as caracter\'{i}sticas dos algoritmos evolucion\'{a}rios como solu\c{c}\~{a}o. Exemplos de modifica\c{c}\~{o}es
s\~{a}o o IGA (\textit{Immune Genetic Algorithm}) \cite{Ma2010}, t\'{e}cnicas de nicho \cite{Camilo2010}, auto-adapta\c{c}\~{a}o \cite{DeJong2006},
sistemas coevolutivos \cite{DeJong2006}, dentre outros. Como apresentado, grande parte das solu\c{c}\~{o}es metaheur\'{i}sticas analisadas que
s\~{a}o aplicadas em problemas de escalonamento job-shop s\~{a}o abordagens h\'{i}bridas.

Uma abordagem h\'{i}brida nova \'{e} o algoritmo auxiliar paralelo (AAP) baseado na fertiliza\c{c}\~{a}o in vitro, inicialmente aplicado em
algoritmos gen\'{e}ticos, a qual ser\'{a} apresentada a seguir. O AAP tem por principal objetivo aproveitar melhor as informa\c{c}\~{o}es
armazenadas na representa\c{c}\~{a}o gen\'{e}tica dos indiv\'{i}duos--solu\c{c}\~{o}es dos algoritmos evolucion\'{a}rios, notalmente em algoritmos
gen\'{e}ticos. Esse melhor aproveitamento tem ainda por finalidade realizar um melhor controle de converg\^{e}ncia do algoritmo sem denegrir a
qualidade final das solu\c{c}\~{o}es obtidas.

\subsection{Algoritmo auxiliar paralelo baseado na fertiliza\c{c}\~{a}o in vitro}

O algoritmo auxiliar paralelo (AAP) baseado na fertiliza\c{c}\~{a}o in vitro tem por base uma considera\c{c}\~{a}o simples: os melhores
indiv\'{i}duos--solu\c{c}\~{o}es de uma popula\c{c}\~{a}o (melhor em termos de sua aptid\~{a}o frente aos demais indiv\'{i}duos) tem em sua
estrutura gen\'{e}tica/cromoss\^{o}mica caracter\'{i}sticas que os tornam indiv\'{i}duos de not\'{a}vel qualidade. Assim, com o fim de melhor
aproveitar estas caracter\'{i}sticas, \'{e} feita uma sele\c{c}\~{a}o desses melhores indiv\'{i}duos e uma recombina\c{c}\~{a}o dentre eles a
fim de gerar um melhor indiv\'{i}duo. Com forte influ\^{e}ncia dos conceitos biol\'{o}gicos da fertiliza\c{c}\~{a}o in vitro, o AAP seleciona
os genes dos pais a fim de construir um novo indiv\'{i}duo de qualidade compar\'{a}vel a de seus pais \cite{Camilo2010}.

O AAP tem um fluxograma conceitualmente simples, subdivide-se nas fases de coleta, manipula\c{c}\~{a}o gen\'{e}tica e transfer\^{e}ncia. A
\textbf{coleta} tem por finalidade selecionar os indiv\'{i}duos de maior qualidade da popula\c{c}\~{a}o corrente de forma a prepar\'{a}-los para
a fase seguinte, de \textbf{manipula\c{c}\~{a}o gen\'{e}tica}. Nessa fase os melhores indiv\'{i}duos selecionados t\^{e}m suas caracter\'{i}sticas
gen\'{e}ticas manipuladas de forma a compor um novo indiv\'{i}duo. O novo indiv\'{i}duo gerado \'{e} avaliado frente a todos os demais
indiv\'{i}duos da popula\c{c}\~{a}o, incluindo seus pais; caso seja um indiv\'{i}duo de maior qualidade, este passa a fazer parte da
popula\c{c}\~{a}o atrav\'{e}s de uma estrat\'{e}gia de \textbf{transfer\^{e}ncia} \cite{Camilo2010}.

Maiores considera\c{c}\~{o}es a respeito do AAP ser\~{a}o dadas na apresenta\c{c}\~{a}o da solu\c{c}\~{a}o evolucion\'{a}ria desenvolvida neste
trabalho, Cap\'{i}tulo \ref{solucao}.

\chapter{Algoritmo evolucion\'{a}rio h\'{i}brido baseado na fertiliza\c{c}\~{a}o in vitro}
\label{solucao}

A partir das caracter\'{i}sticas identificadas da classe de problemas de escalonamento job-shop e com o intuito de obter uma compara\c{c}\~{a}o
de uma abordagem h\'{i}brida frente a uma abordagem can\^{o}nica, foram desenvolvidas duas propostas de solu\c{c}\~{a}o baseadas em algoritmos
gen\'{e}ticos:
\begin{inparaenum}[\itshape a\upshape)]
\item uma solu\c{c}\~{a}o centrada nos princ\'{i}pios e pressupostos desenvolvidos em torno dos algoritmos gen\'{e}ticos, que t\^{e}m se mostrado
aceitos pela comunidade especializada, notavelmente aqueles relacionados \`{a} computa\c{c}\~{a}o evolucion\'{a}ria \cite{DeJong2006}; e
\item uma solu\c{c}\~{a}o h\'{i}brida, atrav\'{e}s de um algoritmo auxiliar paralelo inspirado na fertiliza\c{c}\~{a}o in vitro \cite{Camilo2011},
denominado algoritmo evolucion\'{a}rio h\'{i}brido baseado na fertiliza\c{c}\~{a}o in vitro como solu\c{c}\~{a}o de problemas de escalonamento
job-shop (IVF/EV-JB).
\end{inparaenum}
Ambas as solu\c{c}\~{o}es foram comparadas com configura\c{c}\~{o}es distintas em termos de operadores de sele\c{c}\~{a}o, operadores de
recombina\c{c}\~{a}o e a aplica\c{c}\~{a}o arbitr\'{a}ria de um m\'{e}todo de busca local, obtendo-se assim 27 casos de experimentos.

A solu\c{c}\~{a}o can\^{o}nica foi constru\'{i}da conforme as descri\c{c}\~{o}es feita por De Jong em \textit{Evolutionary computation: a unified
approach} \cite{DeJong2006}, contudo sem perder as refer\^{e}ncias iniciais de Holland em \textit{Adaptation in Natural and Artificial Systems}
\cite{Holland1992}. De Jong defende uma abordagem mais flex\'{i}vel dos algoritmos evolucion\'{a}rios, ainda que seja levado em considera\c{c}\~{a}o
os paradigmas can\^{o}nicos destes algoritmos e a import\^{a}ncia de cada aspecto defendido pelos algoritmos evolucion\'{a}rios. A solu\c{c}\~{a}o
proposta por neste trabalho, portanto, a partir dos algoritmos gen\'{e}ticos, est\'{a} em conson\^{a}ncia com o que \'{e} esperado desta classe
de algoritmos: a partir de um conjunto de indiv\'{i}duos--solu\c{c}\~{o}es aleatoriamente constru\'{i}dos, \'{e} feita uma recombina\c{c}\~{a}o
destes indiv\'{i}duos, como um importante mecanismo de variabilidade, e muta\c{c}\~{o}es s\~{a}o realizadas com o prop\'{o}sito de obter-se uma
explora\c{c}\~{a}o agressiva do espa\c{c}o de busca, conforme delineado no Algoritmo \ref{algoritmo1}.

\medskip
\begin{center}
\begin{minipage}{0.92\textwidth}
\begin{algorithm2e}[H]
  \dontprintsemicolon
  \linesnumbered
  \SetLine
  \BlankLine
  \Entrada{Popula\c{c}\~{a}o de tamanho $\mu$ aleatoriamente constru\'{i}da}
  \BlankLine
  $contador\_gerac\tilde{o}es \leftarrow 0$\;
  \Repeat{$contador\_gerac\tilde{o}es < m\acute{a}ximo\_gerac\tilde{o}es$}{
    Seleciona $\rho = \mu / 2$ pais para reprodu\c{c}\~{a}o\;
    \For{$i \rightarrow 0$ \KwTo $\rho$}{
      $descendente \leftarrow$ Recombina\c{c}\~{a}o do pai $2 * i$ com o pai $2 * i + 1$\;
      Muta\c{c}\~{a}o realizada sobre $descendente$ (busca local)\;
      Adi\c{c}\~{a}o de $descendente$ a um conjunto $\lambda$ de descendentes\;
    }
    O conjunto $\lambda$ de descendentes gerados substitui a gera\c{c}\~{a}o parental e passa a ser a popula\c{c}\~{a}o corrente\;
    Elitismo: o pior indiv\'{i}duo dentre os descendentes gerados \'{e} substitu\'{i}do pelo melhor indiv\'{i}duo da gera\c{c}\~{a}o parental\;
    $contador\_gerac\tilde{o}es \leftarrow contador\_gerac\tilde{o}es + 1$\;
  }
\caption{Solu\c{c}\~{a}o can\^{o}nica (\textit{GA/EV-JB})}
\label{algoritmo1}
\end{algorithm2e}
\end{minipage}
\end{center}

Os algoritmos gen\'{e}ticos, conforme apresentado anteriormente, assumem um mecanismo de controle dos operadores de recombina\c{c}\~{a}o e de
muta\c{c}\~{a}o atrav\'{e}s da probabilidade de acontecer ou n\~{a}o estes mecanismos; ou seja, h\'{a} um controle na aplica\c{c}\~{a}o ou n\~{a}o
destes operadores sobre os indiv\'{i}duos--solu\c{c}\~{o}es da popula\c{c}\~{a}o. A solu\c{c}\~{a}o inicialmente proposta foi constru\'{i}da com
estes princ\'{i}pios; contudo, durante a fase de afina\c{c}\~{a}o da solu\c{c}\~{a}o, verificou-se a perda da qualidade das solu\c{c}\~{o}es
obtidas quando os operadores de variabilidade eram estabelecidos sobre a popula\c{c}\~{a}o a partir de probabilidades. Assim, para efeito do
presente trabalho, a probabilidade de ocorrer a recombina\c{c}\~{a}o de dois ind\'{i}viduos e a muta\c{c}\~{a}o de um novo indiv\'{i}duo gerado
\'{e} de 100\%. Al\'{e}m destes mecanismos de variabilidade, o conceito de elitismo est\'{a} presente nas duas solu\c{c}\~{o}es consideradas:
o pior indiv\'{i}duo dentre os descendentes gerados \'{e} substitu\'{i}do pelo melhor indiv\'{i}duo da gera\c{c}\~{a}o parental. Desta forma 
garante-se que n\~{a}o haver\'{a} a perda de indiv\'{i}duos--solu\c{c}\~{o}es de qualidade a cada gera\c{c}\~{a}o.

\medskip
\begin{center}
\begin{minipage}{0.92\textwidth}
\begin{algorithm2e}[H]
  \dontprintsemicolon
  \linesnumbered
  \SetLine
  \BlankLine
  \Entrada{Popula\c{c}\~{a}o de tamanho $\mu$ aleatoriamente constru\'{i}da}
  \BlankLine
  $contador\_gerac\tilde{o}es \leftarrow 0$\;
  \Repeat{$contador\_gerac\tilde{o}es < m\acute{a}ximo\_gerac\tilde{o}es$}{
    $superpais \leftarrow$ Os melhores $\rho' = \mu * 25\%$ indiv\'{i}duos da gera\c{c}\~{a}o corrente\;
    $embri\tilde{a}o \leftarrow$ Manipula\c{c}\~{a}o gen\'{e}tica dos $superpais$\;
    Seleciona $\rho = \mu / 2$ pais para reprodu\c{c}\~{a}o\;
    \For{$i \rightarrow 0$ \KwTo $\rho$}{
      $descendente \leftarrow$ Recombina\c{c}\~{a}o do pai $2 * i$ com o pai $2 * i + 1$\;
      Muta\c{c}\~{a}o realizada sobre $descendente$ (busca local)\;
      Adi\c{c}\~{a}o de $descendente$ a um conjunto $\lambda$ de descendentes\;
    }
    O conjunto $\lambda$ de descendentes gerados substitui a gera\c{c}\~{a}o parental e passa a ser a popula\c{c}\~{a}o corrente\;
    Elitismo: o pior indiv\'{i}duo dentre os descendentes gerados \'{e} substitu\'{i}do pelo melhor indiv\'{i}duo da gera\c{c}\~{a}o parental\;
    Transfer\^{e}ncia do $embri\tilde{a}o$ para a popula\c{c}\~{a}o de novos indiv\'{i}duos\;
    $contador\_gerac\tilde{o}es \leftarrow contador\_gerac\tilde{o}es + 1$\;
  }
\caption{Solu\c{c}\~{a}o h\'{i}brida (\textit{IVF/EV-JB})}
\label{algoritmo2}
\end{algorithm2e}
\end{minipage}
\end{center}

A abordagem h\'{i}brida, conforme delineado no Algoritmo \ref{algoritmo2}, obtida a partir de um algoritmo auxiliar paralelo (AAP) baseado na
fertiliza\c{c}\~{a}o in vitro \cite{Camilo2011}, tem por base uma extens\~{a}o que tem por prop\'{o}sito propiciar \`{a} solu\c{c}\~{a}o
evolucion\'{a}ria mecanismos que garantam a qualidade gen\'{e}tica dos indiv\'{i}duos, atrav\'{e}s da promo\c{c}\~{a}o destas caracter\'{i}sticas,
e ter um maior controle da converg\^{e}ncia do algoritmo, tornando-a mais r\'{a}pida e de qualidade, sem denegrir os resultados que possam ser
obtidos por essa solu\c{c}\~{a}o \cite{Camilo2011} \cite{Camilo2010}.

O AAP ocorre, conceitualmente, de forma paralela ao fluxo principal dos algoritmos gen\'{e}ticos, na fase de estabelecimento de uma nova
gera\c{c}\~{a}o populacional. O AAP dividi-se em tr\^{e}s fases distintas:
\begin{inparaenum}[\itshape 1\upshape)]
\item \textbf{coleta} --- esta fase tem por prop\'{o}sito selecionar os melhores indiv\'{i}duos da popula\c{c}\~{a}o, j\'{a} que esta abordagem
algor\'{i}tmica assume que os melhores indiv\'{i}duos da popula\c{c}\~{a}o apresentam, geneticamente, as melhores caracter\'{i}sticas daquele
conjunto de indiv\'{i}duos--solu\c{c}\~{o}es. Nos experimentos desenvolvidos, os 25\% melhores indiv\'{i}duos da popula\c{c}\~{a}o s\~{a}o
coletados nessa fase;
\item \textbf{manipula\c{c}\~{a}o gen\'{e}tica} --- a partir dos indiv\'{i}duos previamente coletados \'{e} feita uma manipula\c{c}\~{a}o
gen\'{e}tica sobre estes de forma a gerar novos indiv\'{i}duos atrav\'{e}s da avalia\c{c}\~{a}o, altera\c{c}\~{a}o e recombina\c{c}\~{a}o
daqueles. Os novos indiv\'{i}duos s\~{a}o gerados a partir de um superpai e os demais indiv\'{i}duos da coleta;
\item \textbf{transfer\^{e}ncia} --- caso o processo de manipula\c{c}\~{a}o gen\'{e}tica gere melhores indiv\'{i}duos do que o melhor indiv\'{i}duo
corrente, esses s\~{a}o transferidos para a nova gera\c{c}\~{a}o populacional \cite{Camilo2011}.
\end{inparaenum}
A estrat\'{e}gia de transfer\^{e}ncia utilizada \'{e} a substitui\c{c}\~{a}o do pior indiv\'{i}duo da gera\c{c}\~{a}o parental, conforme delineado
no Algoritmo \ref{algoritmo2}.

Os operadores de sele\c{c}\~{a}o utilizados s\~{a}o o ranqueamento linear \cite{DeJong2006}, o torneio \cite{Engelbrecht2007} \cite{Miller1995}
e a sele\c{c}\~{a}o proporcional \`{a} aptid\~{a}o dos indiv\'{i}duos da popula\c{c}\~{a}o (\textit{fitness proportional selection})
\cite{DeJong2006}. Os operadores de recombina\c{c}\~{a}o individualmente aplicados e analisados foram a recombina\c{c}\~{a}o de $1$-- e
$n$--pontos \cite{DeJong2006} e a recombina\c{c}\~{a}o uniforme (\textit{parameterized uniform crossover}) \cite{DeJong2006} \cite{Goncalves2002}.

Os operadores de muta\c{c}\~{a}o utilizados s\~{a}o dois operadores que s\~{a}o usualmente utilizados em problemas combinatorias e que j\'{a}
foram testados por outros textos relacionados a problemas de escalonamento job-shop. Os operadores de muta\c{c}\~{a}o s\~{a}o inicialmente
testados separadamente em 18 casos de experimentos (1--18) e posteriormente usados em conjunto nos 9 casos de experimentos finais (19--27).
O operador de permuta\c{c}\~{a}o \cite{DeJong2006} \cite{Lin2010} realiza modifica\c{c}\~{o}es na representa\c{c}\~{a}o dos indiv\'{i}duos,
de forma a pertubar as representa\c{c}\~{o}es destes indiv\'{i}duos obtendo-se varia\c{c}\~{o}es que contribuam para uma melhor prospec\c{c}\~{a}o do
espa\c{c}o de busca, em conformidade com o que \'{e} reconhecido pela literatura como um operador de muta\c{c}\~{a}o \cite{DeJong2006}.

\medskip
\begin{center}
\begin{minipage}{0.92\textwidth}
\begin{algorithm2e}[H]
  \dontprintsemicolon
  \linesnumbered
  \SetLine
  \BlankLine
  \Entrada{indiv\'{i}duo a ser mutado}
  \BlankLine
  $n\acute{u}mero\_aleat\acute{o}rio \leftarrow distribuic\tilde{a}o\_normal(0, 100)$\;
  \eIf{$n\acute{u}mero\_aleat\acute{o}rio \leq Prob_{Exchange}$}{
    Operador de troca ($indiv\acute{i}duo$)\;
  }{
    \eIf{$Prob_{Exchange} < n\acute{u}mero\_aleat\acute{o}rio \leq Prob_{Exchange} + Prob_{Insertion}$}{
      Operador de inser\c{c}\~{a}o ($indiv\acute{i}duo$)\;
    }{
      \eIf{$Prob_{Exchange} + Prob_{Insertion} < n\acute{u}mero\_aleat\acute{o}rio \leq Prob_{Exchange} + Prob_{Insertion} + Prob_{Invertion}$}{
        Operador de invers\~{a}o ($indiv\acute{i}duo$)\;
      }{
        Operador de deslocamento ($indiv\acute{i}duo$)\;
      }
    }
  }
\caption{M\'{e}todo de busca local ou muta\c{c}\~{a}o atrav\'{e}s de permuta\c{c}\~{o}es}
\label{algoritmo3}
\end{algorithm2e}
\end{minipage}
\end{center}

O operador de permuta\c{c}\~{a}o, conforme descrito no Algoritmo \ref{algoritmo3}, realiza no m\'{a}ximo um tipo de modifica\c{c}\~{a}o no
indiv\'{i}duo:
\begin{inparaenum}[\itshape a\upshape)]
\item a \textbf{opera\c{c}\~{a}o de troca} (\textit{exchange}) --- a partir de dois valores, $i$ e $j$, em um vetor com $n \times m$ posi\c{c}\~{o}es
($0 < i < j \leq n \times m$), os valores nas posi\c{c}\~{o}es $i$ e $j$ s\~{a}o trocados entre si;
\item a \textbf{opera\c{c}\~{a}o de inser\c{c}\~{a}o} (\textit{insertion}) --- a partir de dois valores, $i$ e $j$, em um vetor com $n \times m$
posi\c{c}\~{o}es ($0 < i < j \leq n \times m$), o valor na posi\c{c}\~{a}o $i$ \'{e} inserido na posi\c{c}\~{a}o $j$, deslocando os valores nas
posi\c{c}\~{o}es menores ou iguais a $j$ para a esquerda, de forma a possibilitar a inser\c{c}\~{a}o do valor em $i$ sem perder nenhum valor do vetor;
\item a \textbf{opera\c{c}\~{a}o de invers\~{a}o} (\textit{invertion}) --- a partir de dois valores, $i$ e $j$, em um vetor $S$ com $n \times m$
posi\c{c}\~{o}es ($0 < i < j \leq n \times m$), o segmento de valores entre as posi\c{c}\~{o}es $i$ e $j$ s\~{a}o invertidos de forma que o vetor
inicial $S = \langle..., s_{i}, s_{i+1}, ..., s_{j-1}, s_{j}, ...\rangle$ \'{e} transformado no vetor
$S' = \langle..., s_{j}, s_{j-1}, ..., s_{i+1}, s_{i}, ...\rangle$;
\item a \textbf{opera\c{c}\~{a}o de deslocamento} (\textit{shifting}) --- a partir de tr\^{e}s valores, $i$, $j$ e $k$, em um vetor
com $n \times m$ posi\c{c}\~{o}es ($0 < i < j \leq n \times m$; $k < i$ ou $k > j$ e $0 < k < n \times m$), o segmento de valores entre as
posi\c{c}\~{o}es $i$ e $j$ s\~{a}o deslocados \`{a} posi\c{c}\~{a}o $k$ do vetor.
\end{inparaenum}
Esse operador de permuta\c{c}\~{a}o foi constru\'{i}do com base em artigos encontrados na literatura e com a experi\^{e}ncia ben\'{e}fica obtida
por solu\c{c}\~{a}o espec\'{i}fica baseada em \textit{particle swarm optimization} \cite{Lin2010}. Os valores das probabilidades determinados
para an\'{a}lise da efetividade da solu\c{c}\~{a}o proposta s\~{a}o: $Prob_{Exchange} = 35\%$, $Prob_{Insertion} = 35\%$, $Prob_{Invertion} = 10\%$ e
$Prob_{Shifting} = 20\%$. Estes valores foram determinados experimentalmente, com base nas escolhas feitas em artigo espec\'{i}fico \cite{Lin2010}.

O segundo operador de muta\c{c}\~{a}o utilizado \'{e} o operador de gera\c{c}\~{a}o aleat\'{o}ria \cite{Goncalves2002}. O prop\'{o}sito desse operador
\'{e} substituir 20\% dos piores indiv\'{i}duos da popula\c{c}\~{a}o por novos indiv\'{i}duos constru\'{i}dos aleatoriamente a partir de uma
distribui\c{c}\~{a}o normal. Assim, obtemos novos valores aos alelos dos indiv\'{i}duos, j\'{a} que, conforme descrito a seguir, os indiv\'{i}duos
s\~{a}o representados como cadeias de n\'{u}meros reais.

\section{Representa\c{c}\~{a}o dos indiv\'{i}duos--solu\c{c}\~{o}es}

Os indiv\'{i}duos--solu\c{c}\~{o}es s\~{a}o representados utilizando a representa\c{c}\~{a}o \textit{random keys} \cite{Bean1993} \cite{Bean1994}
\cite{Snyder2006}, geralmente utilizada para esse tipo de problema por ser uma representa\c{c}\~{a}o simples, quando comparada a outras
representa\c{c}\~{o}es, como a matricial/tridimensional \cite{Zhang2010b} \cite{Yin2007}, e efetiva, por deixar a solu\c{c}\~{a}o algor\'{i}tmica
livre de algumas preocupa\c{c}\~{o}es \`{a} mais, como a constru\c{c}\~{a}o de operadores de recombina\c{c}\~{a}o e muta\c{c}\~{a}o que sempre
gerem indiv\'{i}duos v\'{a}lidos.

A representa\c{c}\~{a}o do indiv\'{i}duo atrav\'{e}s das \textit{random keys} \'{e} feita a partir de um vetor de $n \times m$ posi\c{c}\~{o}es
--- $n$ denota o n\'{u}mero de tarefas e $m$ o n\'{u}mero de m\'{a}quinas. Cada dimens\~{a}o do vetor \'{e} iniciada com valores reais a partir
de uma distribui\c{c}\~{a}o normal $N(0, n \times m)$. Essa representa\c{c}\~{a}o \'{e} centrada nas opera\c{c}\~{o}es: o vetor como solu\c{c}\~{a}o
descreve a ordem a qual as opera\c{c}\~{o}es na oficina ser\~{a}o processadas. A fim de descobrir cada uma das opera\c{c}\~{o}es a partir da
representa\c{c}\~{a}o inicial, em valores reais, s\~{a}o feitas as seguintes opera\c{c}\~{o}es --- a t\'{i}tulo de exemplifica\c{c}\~{a}o, tomemos
como exemplo o vetor de 8 posi\c{c}\~{o}es, com $n = 2$ e $m = 4$, $\langle0.2, 0.5, 1.8, 6.7, 3.3, 2.4, 3.5, 2.4\rangle$:
\begin{inparaenum}[\itshape 1\upshape)]
\item a partir do vetor inicialmente constru\'{i}do atrav\'{e}s da distribui\c{c}\~{a}o normal, \'{e} feito um ranqueamento dos valores de cada
dimens\~{a}o, de forma a ordenar as dimens\~{o}es do vetor em rela\c{c}\~{a}o \`{a}s demais dimens\~{o}es. Ent\~{a}o, no vetor de 8 posi\c{c}\~{o}es
tomado como exemplo, o ranqueamento de cada dimens\~{a}o resultaria em um vetor com a seguinte configura\c{c}\~{a}o
$\langle1, 2, 3, 8, 6, 4, 7, 5\rangle$;
\item ap\'{o}s o ranqueamento das dimens\~{o}es do vetor \'{e} necess\'{a}rio identificar cada uma das tarefas que o vetor--solu\c{c}\~{a}o est\'{a}
referenciando e estabelecer a ordem de processamento de cada uma das opera\c{c}\~{o}es. Para a dimens\~{a}o $i$ do vetor $S$ ranqueado, temos
$S'_{i} = (S_{i}$ \textit{mod} $n) + 1$, $S'$ o vetor final com a sequ\^{e}ncia de processamento das opera\c{c}\~{o}es. No vetor tomado como exemplo
ter\'{i}amos, para $n = 2$, a seguinte configura\c{c}\~{a}o final: $\langle2, 1, 2, 1, 1, 1, 2, 2\rangle$;
\item por fim, a partir da sequ\^{e}ncia de processamento das opera\c{c}\~{o}es, obtemos o vetor 
$\langle o_{2,1}, o_{1,1}, o_{2,2}, o_{1,2}, o_{1,3}, o_{1,4}, o_{2,3}, o_{2,4}\rangle$, com $o_{i,k}$ a opera\c{c}\~{a}o da tarefa $i$ na
$k$-\'{e}sima m\'{a}quina de sua configura\c{c}\~{a}o tecnol\'{o}gica --- se a tarefa 2 fosse inicialmente processada na m\'{a}quina 4, $o_{2,1}$
refere-se \`{a} opera\c{c}\~{a}o da tarefa 2 na m\'{a}quina 4.
\end{inparaenum}

\subsection{Popula\c{c}\~{a}o inicial}
\label{inicial}

Como \'{e} de costume em solu\c{c}\~{o}es baseadas em algoritmos evolucion\'{a}rios \cite{DeJong2006} \cite{Goncalves2002}, os
indiv\'{i}duos--solu\c{c}\~{o}es da popula\c{c}\~{a}o inicial s\~{a}o criados a partir de distribui\c{c}\~{o}es normais. Cada dimens\~{a}o do
vetor--indiv\'{i}duo \'{e} iniciada com o valor da distribui\c{c}\~{a}o $N(0, n \times m)$, sendo $n$ o n\'{u}mero de tarefas e $m$ o n\'{u}mero
de m\'{a}quinas da oficina. O tamanho da popula\c{c}\~{a}o ($\mu$) tamb\'{e}m \'{e} influenciado pelas configura\c{c}\~{o}es da oficina (problema)
em quest\~{a}o: $\mu = 2 \times n \times m$.

\section{Operadores de variabilidade e seus efeitos sobre a solu\c{c}\~{a}o evolucion\'{a}ria}

Considera\c{c}\~{o}es a respeito da influ\^{e}ncia dos operadores de variabilidade em algoritmos evolucion\'{a}rios s\~{a}o reconhecidos e estudados
na literatura \cite{DeJong2006} \cite{Miller1995}. No presente trabalho, al\'{e}m da an\'{a}lise do comportamento do algoritmo h\'{i}brido frente
a abordagem can\^{o}nica, foi feita uma an\'{a}lise dos efeitos que os operadores de variabilidade utilizados causaram sobre a efetividade da
solu\c{c}\~{a}o proposta.

Os operadores de variabilidade dividem-se em operadores de recombina\c{c}\~{a}o, os quais, a partir de um ou mais indiv\'{i}duos, geram um novo
indiv\'{i}duo; e o operadores de muta\c{c}\~{a}o, que a partir de um indiv\'{i}duo gerado por recombina\c{c}\~{a}o obt\'{e}m-se um indiv\'{i}duo
com caracter\'{i}sticas novas naquela popula\c{c}\~{a}o, como ser\'{a} explicado a seguir.

\subsection{Operadores de recombina\c{c}\~{a}o}
Os operadores de recombina\c{c}\~{a}o utilizados s\~{a}o aqueles usualmente encontrados em solu\c{c}\~{o}es baseadas nos algoritmos gen\'{e}ticos.
Conquanto a representa\c{c}\~{a}o dos indiv\'{i}duos--solu\c{c}\~{o}es seja feita a partir de um vetor de n\'{u}meros reais (\textit{random-keys}),
conceitualmente a recombina\c{c}\~{a}o permanece id\^{e}ntica a uma solu\c{c}\~{a}o baseada em indiv\'{i}duos representados atrav\'{e}s de cadeias
bin\'{a}rias.

\begin{description}
\item[recombina\c{c}\~{a}o de $1$--ponto] a partir de dois indiv\'{i}duos \'{e} gerado um novo indiv\'{i}duo. Com esse fim, \'{e} feita a
recombina\c{c}\~{a}o dos genes dos dois indiv\'{i}duos, doravante denominados pais. Inicialmente \'{e} feita a escolha aleat\'{o}ria de um
ponto de corte atrav\'{e}s de uma distribui\c{c}\~{a}o discreta a fim de escolher qual ponto de corte determinar\'{a} a composi\c{c}\~{a}o
do novo indiv\'{i}duo. Seja $p$ e $m$ os vetores--pais e $f$ o vetor--filho, o novo indiv\'{i}duo a ser gerado, e $i$ o ponto de corte
($|p| = |m| = |f| = n, 0 \leq i < n$), o vetor--filho $f$ ser\'{a} composto da seguinte forma:
$f = \langle p_{0}, ..., p_{i-1}, m_{i}, m_{i+1}, ..., m_{n} \rangle$;
\item[recombina\c{c}\~{a}o de $n$--pontos] de forma semelhante \`{a} recombina\c{c}\~{a}o de $1$--ponto, a recombina\c{c}\~{a}o de $n$--pontos
determina $n$ pontos distintos como pontos de corte para compor o novo indiv\'{i}duo. Ap\'{o}s an\'{a}lises dos efeitos dos valores de $n$,
a solu\c{c}\~{a}o final tem um valor de $n = 4$, portanto fixo. O termo correto, assim, seria recombina\c{c}\~{a}o de $4$-pontos;
\item[recombina\c{c}\~{a}o uniforme] a recombina\c{c}\~{a}o uniforme funciona de forma distinta; n\~{a}o s\~{a}o estabelecidos pontos de
corte. Ainda assim,  a partir de dois indiv\'{i}duos \'{e} gerado um novo indiv\'{i}duo. Para compor esse novo indiv\'{i}duo, cada gene do
vetor--indiv\'{i}duo deve ser escolhido arbitrariamente entre os vetores--pais; essa escolha \'{e} feita a partir do lan\c{c}ado de uma moeda
``viciada'': \'{e} feita uma distrui\c{c}\~{a}o normal, $m \leftarrow N(0, 100)$. Caso $m < 70$, \'{e} cara, o contr\'{a}rio \'{e} coroa.
Assim, para cada gene, determina-se cara-ou-coroa, gene no pai ou gene da m\~{a}e.
\end{description}

\subsection{Operadores de muta\c{c}\~{a}o}
Como a representa\c{c}\~{a}o dos indiv\'{i}duos--solu\c{c}\~{o}es \'{e} feita a partir de vetores de n\'{u}meros reais, os operadores de
muta\c{c}\~{a}o usualmente utilizados em algoritmos gen\'{e}ticos n\~{a}o s\~{a}o utilizados na solu\c{c}\~{a}o algor\'{i}tmica constru\'{i}da.
Os operadores de muta\c{c}\~{a}o utilizados s\~{a}o baseados em experi\^{e}ncias de qualidade encontradas na literatura \cite{Goncalves2002}
\cite{Lin2010}.

\begin{description}
\item[permuta\c{c}\~{a}o] conforme delineado anteriormente, esse operador de muta\c{c}\~{a}o tem por finalidade reorganizar as dimens\~{o}es do
vetor de forma aleat\'{o}ria, pertubando os novos indiv\'{i}duos--solu\c{c}\~{o}es gerados. Essa pertuba\c{c}\~{a}o em n\'{i}vel genot\'{i}pico
\'{e} revertido em n\'{i}vel fenot\'{i}pico para uma pertuba\c{c}\~{a}o na ordem de processamento das opera\c{c}\~{o}es das tarefas;
\item[gera\c{c}\~{a}o aleat\'{o}ria] esse operador tem por influ\^{e}ncia as estrat\'{e}gias evolutivas \cite{Goncalves2002} e tem por finalidade
prover \`{a} popula\c{c}\~{a}o de indiv\'{i}duos--solu\c{c}\~{o}es novos valores genot\'{i}picos, o que acarretar\'{a} em pertuba\c{c}\~{o}es em
n\'{i}vel fenot\'{i}pico. Os 20\% piores indiv\'{i}duos da popula\c{c}\~{a}o s\~{a}o substitu\'{i}dos por novos indiv\'{i}duos aleatoriamente
criados, de forma an\'{a}loga \`{a} popula\c{c}\~{a}o inicial (Se\c{c}\~{a}o \ref{inicial}).
\end{description}

\section{Operadores de sele\c{c}\~{a}o}

Os operadores de variabilidade t\^{e}m sua influ\^{e}ncia na forma como novos indiv\'{i}duos ser\~{a}o gerados e como o espa\c{c}o de busca do
problema de otimiza\c{c}\~{a}o ser\'{a} percorrido, a partir das escolhas feitas pela representa\c{c}\~{a}o dos indiv\'{i}duos--solu\c{c}\~{o}es
e como os operadores de variabilidade s\~{a}o capazes de evoluir/transformar essas representa\c{c}\~{o}es. Por outro lado, os operadores de
sele\c{c}\~{a}o determinam as escolhas a longo prazo que o algoritmo far\'{a} de quais caracter\'{i}sticas--indiv\'{i}duos permanecer\~{a}o
naquela popula\c{c}\~{a}o a cada gera\c{c}\~{a}o. Portanto, a intera\c{c}\~{a}o entre os operadores de variabilidade e os operadores de
sele\c{c}\~{a}o s\~{a}o fatores importantes na an\'{a}lise da efetividade de uma solu\c{c}\~{a}o baseada em algoritmos evolucion\'{a}rios.

Para efeito de an\'{a}lises foram utilizados tr\^{e}s operadores de sele\c{c}\~{a}o distintos \cite{DeJong2006} \cite{Engelbrecht2007}:
\begin{description}
\item[sele\c{c}\~{a}o por ranqueamento linear] operador de sele\c{c}\~{a}o mais simples e com maior press\~{a}o seletiva, a sele\c{c}\~{a}o
por ranqueamento linear organiza os indiv\'{i}duos--solu\c{c}\~{o}es dos melhores aos piores; os melhores indiv\'{i}duos s\~{a}o sempre
escolhidos em detrimento dos piores indiv\'{i}duos da popula\c{c}\~{a}o;
\item[sele\c{c}\~{a}o proporcional \`{a} aptid\~{a}o] a sele\c{c}\~{a}o proporcional \`{a} aptid\~{a}o (\textit{fitness}) tem press\~{a}o
seletiva m\'{e}dia e garante a todos os indiv\'{i}duos a oportunidade de serem selecionados. Inicialmente \'{e} feito o somat\'{o}rio da
aptid\~{a}o dos indiv\'{i}duos da popula\c{c}\~{a}o: seja $n$ o tamanho da popula\c{c}\~{a}o, $f_{i}$ a aptid\~{a}o do indiv\'{i}duo
$i$, $1 \leq i \leq n$; o somat\'{o}rio da aptid\~{a}o dos indiv\'{i}udos da popula\c{c}\~{a}o $S_{f} = \sum^{n}_{k=1}f_{k}$. Em seguida
\'{e} feita uma distribui\c{c}\~{a}o normal $d \leftarrow N(0, S_{f})$ para determinar qual indiv\'{i}duo ser\'{a} selecionado. Por fim,
\'{e} feita uma roleta para determinar qual indiv\'{i}duo ser\'{a} selecionado, conforme algoritmo \ref{roleta}:


\medskip
\begin{center}
\begin{minipage}{0.92\textwidth}
\begin{algorithm2e}[H]
  \dontprintsemicolon
  \linesnumbered
  \SetLine
  \BlankLine
  \Entrada{Popula\c{c}\~{a}o de $\mu$ indiv\'{i}duos; e valor $d$ de escolha da roleta}
  \Saida{$I_{i}$, o indiv\'{i}duo $i$ da popula\c{c}\~{a}o}
  \BlankLine
  $i \leftarrow 1$, denota o indiv\'{i}duo $i$ da popula\c{c}\~{a}o, $1 \leq i \leq \mu$\;
  $soma \leftarrow f_{i}$\;
  \While{$soma < d$}{
    $i \leftarrow i + 1$, passa para o pr\'{o}ximo indiv\'{i}duo\;
    $soma \leftarrow soma + f_{i}$\;
  }
  Retorna $I_{i}$, o indiv\'{i}duo $i$ da popula\c{c}\~{a}o\;
\caption{Roleta}
\label{roleta}
\end{algorithm2e}
\end{minipage}
\end{center}

\item[sele\c{c}\~{a}o por torneio] a sele\c{c}\~{a}o por torneio classifica $n$ indiv\'{i}duos da popula\c{c}\~{a}o,
aleatoriamente, para participar do torneio e escolhe um dentre os escolhidos. Para efeito de an\'{a}lises e ap\'{o}s alguns testes para
afina\c{c}\~{a}o da solu\c{c}\~{a}o, o valor de $n$ foi fixado em $5$.
\end{description}

\chapter{Resultados obtidos}
\label{resultados}

Com o fim de analisar a solu\c{c}\~{a}o algor\'{i}tmica proposta, foram criados 27 casos de experimentos realizados sobre 18 inst\^{a}ncias
da OR-Library \cite{OrLibrary}. Cada inst\^{a}ncia foi exercitada 20 vezes em cada uma das solu\c{c}\~{o}es propostas, em cada um dos 27
casos de experimentos. A execu\c{c}\~{a}o dos casos de experimentos foi feita sobre um Intel Core 2 Duo de 2GHz, 3GB de RAM e sistema
operacional GNU/Linux Ubuntu 11.10, Kernel Linux 3.0.0-14-generic. A implementa\c{c}\~{a}o das solu\c{c}\~{o}es foi feita em linguagem
\textit{C} --- um framework para execu\c{c}\~{a}o autom\'{a}tica dos casos de experimentos foi escrito em \textit{Python}. Conquanto a
OR-Library forne\c{c}a 82 inst\^{a}ncias de problemas de escalonamento job-shop, n\~{a}o necessariamente a literatura realiza o exerc\'{i}cio
de todas estas inst\^{a}ncias \cite{Goncalves2002} \cite{Lin2010} \cite{Gao2011} \cite{Rego2009}.

Os 27 casos de experimentos s\~{a}o organizados conforme delineado na tabela \ref{experimentos}. Nessa, \textit{1PXO} e \textit{NPXO}
referem-se, respectivamente, a recombina\c{c}\~{a}o de $1$ e $n$--pontos; \textit{UXO} refere-se \`{a} recombina\c{c}\~{a}o uniforme;
\textit{MP} refere-se a muta\c{c}\~{a}o por permuta\c{c}\~{o}es; \textit{GA} refere-se \`{a} gera\c{c}\~{a}o aleat\'{o}ria de descendentes;
\textit{RANK}, \textit{FP} e \textit{T} referem-se, respectivamente, \`{a} sele\c{c}\~{a}o por ranqueamento linear, proporcional \`{a}
aptid\~{a}o e por torneio; e \textit{EXP} refere-se ao n\'{u}mero do experimento. Em seguida s\~{a}o apresentadas as 27 tabelas com os
resultados obtidos para cada experimento.

Cada tabela com os resultados (da tabela \ref{experimento1} \`{a} tabela \ref{experimento27}) apresenta, para cada algoritmo desenvolvido,
o melhor resultado obtido nas 20 execu\c{c}\~{o}es, a m\'{e}dia dos melhores valores obtidos nas 20 execu\c{c}\~{o}es, o pior resultado
obtido nas 20 execu\c{c}\~{o}es e a m\'{e}dia populacional de todos os indiv\'{i}duos que foram criados nas 20 execu\c{c}\~{o}es.
\textit{GA/EV-JB} refere-se \`{a} solu\c{c}\~{a}o algor\'{i}tmica baseada na vers\~{a}o can\^{o}nica dos algoritmos gen\'{e}ticos e
\textit{IVF/EV-JB} refere-se \`{a} solu\c{c}\~{a}o h\'{i}brida, objeto de desenvolvimento e an\'{a}lise no presente trabalho --- conforme
descrito no Cap\'{i}tulo \ref{solucao}. \textit{BKS} (\textit{best known solution}) refere-se \`{a} melhor solu\c{c}\~{a}o reconhecida pela
literatura para aquela inst\^{a}ncia \cite{Gao2011} \cite{Lin2010} \cite{Goncalves2002}. O tamanho ou dimens\~{a}o da inst\^{a}ncia \'{e}
determinado pelo n\'{u}mero $n$ de tarefas e o n\'{u}mero $m$ de m\'{a}quinas.

\newcommand{\x}{$\bullet$}
\begin{table*}[hp]
\caption{Configura\c{c}\~{a}o dos casos de experimentos}
\centering
\label{experimentos}
\begin{tabular}{ccccccccc}
\toprule
\multicolumn{8}{c}{Configura\c{c}\~{o}es} \\
\cmidrule(c){1-8}
 1PXO & NPXO & UXO &  MP  &  GA  & RANK &  FP  &  T  & EXP \\
\midrule
 \x   &      &     &  \x  &      &  \x  &      &     &  1  \\
      &  \x  &     &  \x  &      &  \x  &      &     &  2  \\
      &      & \x  &  \x  &      &  \x  &      &     &  3  \\
 \x   &      &     &  \x  &      &      &  \x  &     &  4  \\
      &  \x  &     &  \x  &      &      &  \x  &     &  5  \\
      &      & \x  &  \x  &      &      &  \x  &     &  6  \\
 \x   &      &     &  \x  &      &      &      & \x  &  7  \\
      &  \x  &     &  \x  &      &      &      & \x  &  8  \\
      &      & \x  &  \x  &      &      &      & \x  &  9  \\
 \x   &      &     &      &  \x  &  \x  &      &     &  10 \\
      &  \x  &     &      &  \x  &  \x  &      &     &  11 \\
      &      & \x  &      &  \x  &  \x  &      &     &  12 \\
 \x   &      &     &      &  \x  &      &  \x  &     &  13 \\
      &  \x  &     &      &  \x  &      &  \x  &     &  14 \\
      &      & \x  &      &  \x  &      &  \x  &     &  15 \\
 \x   &      &     &      &  \x  &      &      & \x  &  16 \\
      &  \x  &     &      &  \x  &      &      & \x  &  17 \\
      &      & \x  &      &  \x  &      &      & \x  &  18 \\
 \x   &      &     &  \x  &  \x  &  \x  &      &     &  19 \\
      &  \x  &     &  \x  &  \x  &  \x  &      &     &  20 \\
      &      & \x  &  \x  &  \x  &  \x  &      &     &  21 \\
 \x   &      &     &  \x  &  \x  &      &  \x  &     &  22 \\
      &  \x  &     &  \x  &  \x  &      &  \x  &     &  23 \\
      &      & \x  &  \x  &  \x  &      &  \x  &     &  24 \\
 \x   &      &     &  \x  &  \x  &      &      & \x  &  25 \\
      &  \x  &     &  \x  &  \x  &      &      & \x  &  26 \\
      &      & \x  &  \x  &  \x  &      &      & \x  &  27 \\
\bottomrule
\end{tabular}
\end{table*}

\begin{sidewaystable}
\caption{Resultados do caso de experimento 1}
\centering
\label{experimento1}
\begin{tabular}{cccccccccccc}
\toprule
& & & \multicolumn{4}{c}{GA/EV-JB} & & \multicolumn{4}{c}{IVF/EV-JB} \\
\cmidrule(c){4-7}
\cmidrule(c){9-12}
Inst\^{a}ncia & Tamanho ($n \times m$) & BKS & Melhor & M\'{e}dia (melhores) & Pior & M\'{e}dia (pop.) & & Melhor & M\'{e}dia (melhores) & Pior & M\'{e}dia (pop.) \\
\midrule
FT06 & $6 \times 6$ & 55 & 55 & 55 & 147 & 69 & & 55 & 55 & 146 & 69 \\
FT10 & $10 \times 10$ & 930 & 961 & 1018 & 2498 & 1243 & & 969 & 1001 & 2445 & 1374 \\
FT20 & $20 \times 5$ & 1165 & 1180 & 1254 & 2982 & 1462 & & 1185 & 1253 & 2776 & 1473 \\
LA01 & $10 \times 5$ & 666 & 666 & 671 & 1737 & 823 & & 666 & 666 & 1699 & 799 \\
LA02 & $10 \times 5$ & 655 & 655 & 677 & 1651 & 878 & & 655 & 672 & 1561 & 849 \\
LA03 & $10 \times 5$ & 597 & 603 & 618 & 1596 & 802 & & 605 & 621 & 1450 & 765 \\
LA04 & $10 \times 5$ & 590 & 590 & 609 & 1607 & 841 & & 590 & 606 & 1512 & 738 \\
LA05 & $10 \times 5$ & 593 & 593 & 593 & 1417 & 698 & & 593 & 593 & 1479 & 698 \\
LA06 & $15 \times 5$ & 926 & 926 & 926 & 2078 & 1118 & & 926 & 926 & 2080 & 1110 \\
LA07 & $15 \times 5$ & 890 & 890 & 890 & 2124 & 1110 & & 890 & 890 & 1986 & 1111 \\
LA08 & $15 \times 5$ & 863 & 863 & 863 & 2083 & 1138 & & 863 & 863 & 2123 & 1066 \\
LA09 & $15 \times 5$ & 951 & 951 & 951 & 2342 & 1114 & & 951 & 951 & 2120 & 1216 \\
LA10 & $15 \times 5$ & 958 & 958 & 958 & 2065 & 1194 & & 958 & 958 & 2029 & 1186 \\
LA11 & $20 \times 5$ & 1222 & 1222 & 1222 & 2605 & 1441 & & 1222 & 1222 & 2621 & 1515 \\
LA12 & $20 \times 5$ & 1039 & 1039 & 1039 & 2308 & 1216 & & 1039 & 1039 & 2189 & 1182 \\
LA13 & $20 \times 5$ & 1150 & 1150 & 1150 & 2497 & 1317 & & 1150 & 1150 & 2374 & 1313 \\
LA14 & $20 \times 5$ & 1292 & 1292 & 1292 & 2584 & 1428 & & 1292 & 1292 & 2553 & 1450 \\
LA15 & $20 \times 5$ & 1207 & 1207 & 1207 & 2810 & 1464 & & 1207 & 1209 & 2655 & 1449 \\
\bottomrule
\end{tabular}
\end{sidewaystable}

\cleardoublepage
\begin{sidewaystable}
\caption{Resultados do caso de experimento 2}
\centering
\label{experimento2}
\begin{tabular}{cccccccccccc}
\toprule
& & & \multicolumn{4}{c}{GA/EV-JB} & & \multicolumn{4}{c}{IVF/EV-JB} \\
\cmidrule(c){4-7}
\cmidrule(c){9-12}
Inst\^{a}ncia & Tamanho ($n \times m$) & BKS & Melhor & M\'{e}dia (melhores) & Pior & M\'{e}dia (pop.) & & Melhor & M\'{e}dia (melhores) & Pior & M\'{e}dia (pop.) \\
\midrule
FT06 & $6 \times 6$ & 55 & 55 & 55 & 155 & 77 & & 55 & 55 & 139 & 77 \\
FT10 & $10 \times 10$ & 930 & 986 & 1033 & 2548 & 1408 & & 977 & 1050 & 2461 & 1470 \\
FT20 & $20 \times 5$ & 1165 & 1202 & 1256 & 3060 & 1724 & & 1196 & 1267 & 2817 & 1608 \\
LA01 & $10 \times 5$ & 666 & 666 & 671 & 1824 & 861 & & 666 & 668 & 1675 & 906 \\
LA02 & $10 \times 5$ & 655 & 655 & 676 & 1626 & 901 & & 663 & 667 & 1640 & 911 \\
LA03 & $10 \times 5$ & 597 & 611 & 622 & 1550 & 855 & & 606 & 616 & 1508 & 796 \\
LA04 & $10 \times 5$ & 590 & 595 & 607 & 1616 & 881 & & 593 & 601 & 1574 & 843 \\
LA05 & $10 \times 5$ & 593 & 593 & 593 & 1445 & 792 & & 593 & 593 & 1420 & 798 \\
LA06 & $15 \times 5$ & 926 & 926 & 926 & 2253 & 1162 & & 926 & 926 & 1972 & 1219 \\
LA07 & $15 \times 5$ & 890 & 890 & 892 & 2089 & 1171 & & 890 & 893 & 1955 & 1192 \\
LA08 & $15 \times 5$ & 863 & 863 & 863 & 2085 & 1180 & & 863 & 863 & 2087 & 1178 \\
LA09 & $15 \times 5$ & 951 & 951 & 951 & 2248 & 1305 & & 951 & 951 & 2135 & 1249 \\
LA10 & $15 \times 5$ & 958 & 958 & 958 & 2281 & 1209 & & 958 & 958 & 2065 & 1222 \\
LA11 & $20 \times 5$ & 1222 & 1222 & 1222 & 2599 & 1531 & & 1222 & 1222 & 2487 & 1452 \\
LA12 & $20 \times 5$ & 1039 & 1039 & 1039 & 2341 & 1342 & & 1039 & 1039 & 2351 & 1417 \\
LA13 & $20 \times 5$ & 1150 & 1150 & 1150 & 2556 & 1487 & & 1150 & 1150 & 2457 & 1390 \\
LA14 & $20 \times 5$ & 1292 & 1292 & 1292 & 2691 & 1642 & & 1292 & 1292 & 2565 & 1561 \\
LA15 & $20 \times 5$ & 1207 & 1207 & 1223 & 2806 & 1689 & & 1207 & 1207 & 2769 & 1482 \\
\bottomrule
\end{tabular}
\end{sidewaystable}

\cleardoublepage
\begin{sidewaystable}
\caption{Resultados do caso de experimento 3}
\centering
\label{experimento3}
\begin{tabular}{cccccccccccc}
\toprule
& & & \multicolumn{4}{c}{GA/EV-JB} & & \multicolumn{4}{c}{IVF/EV-JB} \\
\cmidrule(c){4-7}
\cmidrule(c){9-12}
Inst\^{a}ncia & Tamanho ($n \times m$) & BKS & Melhor & M\'{e}dia (melhores) & Pior & M\'{e}dia (pop.) & & Melhor & M\'{e}dia (melhores) & Pior & M\'{e}dia (pop.) \\
\midrule
FT06 & $6 \times 6$ & 55 & 55 & 55 & 152 & 73 & & 55 & 55 & 139 & 78 \\
FT10 & $10 \times 10$ & 930 & 996 & 1047 & 2554 & 1603 & & 972 & 1065 & 2454 & 1546 \\
FT20 & $20 \times 5$ & 1165 & 1246 & 1315 & 2996 & 1990 & & 1220 & 1351 & 2766 & 1735 \\
LA01 & $10 \times 5$ & 666 & 666 & 667 & 1713 & 959 & & 666 & 669 & 1663 & 903 \\
LA02 & $10 \times 5$ & 655 & 655 & 684 & 1670 & 974 & & 663 & 687 & 1752 & 924 \\
LA03 & $10 \times 5$ & 597 & 609 & 645 & 1590 & 904 & & 617 & 626 & 1468 & 850 \\
LA04 & $10 \times 5$ & 590 & 590 & 597 & 1661 & 930 & & 590 & 604 & 1518 & 806 \\
LA05 & $10 \times 5$ & 593 & 593 & 593 & 1548 & 725 & & 593 & 593 & 1391 & 803 \\
LA06 & $15 \times 5$ & 926 & 926 & 926 & 2180 & 1259 & & 926 & 926 & 2027 & 1195 \\
LA07 & $15 \times 5$ & 890 & 890 & 891 & 2109 & 1330 & & 890 & 890 & 1957 & 1234 \\
LA08 & $15 \times 5$ & 863 & 863 & 863 & 2209 & 1268 & & 863 & 863 & 2044 & 1250 \\
LA09 & $15 \times 5$ & 951 & 951 & 951 & 2304 & 1320 & & 951 & 951 & 2145 & 1211 \\
LA10 & $15 \times 5$ & 958 & 958 & 958 & 2290 & 1287 & & 958 & 958 & 2051 & 1248 \\
LA11 & $20 \times 5$ & 1222 & 1222 & 1222 & 2733 & 1614 & & 1222 & 1222 & 2458 & 1557 \\
LA12 & $20 \times 5$ & 1039 & 1039 & 1039 & 2328 & 1483 & & 1039 & 1039 & 2260 & 1447 \\
LA13 & $20 \times 5$ & 1150 & 1150 & 1150 & 2611 & 1571 & & 1150 & 1150 & 2450 & 1612 \\
LA14 & $20 \times 5$ & 1292 & 1292 & 1292 & 2663 & 1631 & & 1292 & 1292 & 2437 & 1582 \\
LA15 & $20 \times 5$ & 1207 & 1207 & 1246 & 2906 & 1774 & & 1207 & 1207 & 2691 & 1629 \\
\bottomrule
\end{tabular}
\end{sidewaystable}

\cleardoublepage
\begin{sidewaystable}
\caption{Resultados do caso de experimento 4}
\centering
\label{experimento4}
\begin{tabular}{cccccccccccc}
\toprule
& & & \multicolumn{4}{c}{GA/EV-JB} & & \multicolumn{4}{c}{IVF/EV-JB} \\
\cmidrule(c){4-7}
\cmidrule(c){9-12}
Inst\^{a}ncia & Tamanho ($n \times m$) & BKS & Melhor & M\'{e}dia (melhores) & Pior & M\'{e}dia (pop.) & & Melhor & M\'{e}dia (melhores) & Pior & M\'{e}dia (pop.) \\
\midrule
FT06 & $6 \times 6$ & 55 & 55 & 57 & 156 & 87 & & 55 & 56 & 140 & 87 \\
FT10 & $10 \times 10$ & 930 & 1156 & 1185 & 2630 & 1716 & & 1143 & 1171 & 2511 & 1722 \\
FT20 & $20 \times 5$ & 1165 & 1497 & 1525 & 3090 & 2088 & & 1435 & 1476 & 2848 & 2078 \\
LA01 & $10 \times 5$ & 666 & 691 & 705 & 1850 & 1055 & & 669 & 713 & 1669 & 1046 \\
LA02 & $10 \times 5$ & 655 & 709 & 744 & 1814 & 1089 & & 710 & 746 & 1621 & 1066 \\
LA03 & $10 \times 5$ & 597 & 655 & 682 & 1656 & 996 & & 643 & 684 & 1454 & 984 \\
LA04 & $10 \times 5$ & 590 & 640 & 658 & 1669 & 1007 & & 610 & 675 & 1565 & 1001 \\
LA05 & $10 \times 5$ & 593 & 593 & 593 & 1519 & 891 & & 593 & 593 & 1380 & 872 \\
LA06 & $15 \times 5$ & 926 & 926 & 944 & 2439 & 1369 & & 926 & 953 & 2019 & 1366 \\
LA07 & $15 \times 5$ & 890 & 933 & 971 & 2222 & 1412 & & 951 & 982 & 1992 & 1404 \\
LA08 & $15 \times 5$ & 863 & 923 & 943 & 2361 & 1381 & & 885 & 942 & 2035 & 1366 \\
LA09 & $15 \times 5$ & 951 & 955 & 994 & 2391 & 1429 & & 951 & 979 & 2160 & 1426 \\
LA10 & $15 \times 5$ & 958 & 958 & 959 & 2233 & 1357 & & 958 & 958 & 2023 & 1337 \\
LA11 & $20 \times 5$ & 1222 & 1259 & 1303 & 2764 & 1753 & & 1233 & 1286 & 2471 & 1750 \\
LA12 & $20 \times 5$ & 1039 & 1075 & 1087 & 2515 & 1550 & & 1065 & 1072 & 2248 & 1547 \\
LA13 & $20 \times 5$ & 1150 & 1187 & 1214 & 2771 & 1700 & & 1151 & 1217 & 2437 & 1696 \\
LA14 & $20 \times 5$ & 1292 & 1292 & 1292 & 2802 & 1739 & & 1292 & 1292 & 2576 & 1737 \\
LA15 & $20 \times 5$ & 1207 & 1347 & 1388 & 2928 & 1890 & & 1347 & 1404 & 2653 & 1891 \\
\bottomrule
\end{tabular}
\end{sidewaystable}

\cleardoublepage
\begin{sidewaystable}
\caption{Resultados do caso de experimento 5}
\centering
\label{experimento5}
\begin{tabular}{cccccccccccc}
\toprule
& & & \multicolumn{4}{c}{GA/EV-JB} & & \multicolumn{4}{c}{IVF/EV-JB} \\
\cmidrule(c){4-7}
\cmidrule(c){9-12}
Inst\^{a}ncia & Tamanho ($n \times m$) & BKS & Melhor & M\'{e}dia (melhores) & Pior & M\'{e}dia (pop.) & & Melhor & M\'{e}dia (melhores) & Pior & M\'{e}dia (pop.) \\
\midrule
FT06 & $6 \times 6$ & 55 & 55 & 57 & 160 & 88 & & 55 & 55 & 139 & 87 \\
FT10 & $10 \times 10$ & 930 & 1156 & 1227 & 2681 & 1721 & & 1156 & 1225 & 2405 & 1714 \\
FT20 & $20 \times 5$ & 1165 & 1447 & 1489 & 3176 & 2088 & & 1467 & 1501 & 2881 & 2070 \\
LA01 & $10 \times 5$ & 666 & 678 & 711 & 1856 & 1053 & & 681 & 701 & 1781 & 1048 \\
LA02 & $10 \times 5$ & 655 & 727 & 745 & 1802 & 1082 & & 718 & 737 & 1587 & 1071 \\
LA03 & $10 \times 5$ & 597 & 662 & 680 & 1640 & 996 & & 648 & 678 & 1475 & 987 \\
LA04 & $10 \times 5$ & 590 & 632 & 665 & 1707 & 1009 & & 634 & 658 & 1524 & 992 \\
LA05 & $10 \times 5$ & 593 & 593 & 593 & 1585 & 887 & & 593 & 593 & 1365 & 865 \\
LA06 & $15 \times 5$ & 926 & 933 & 965 & 2318 & 1370 & & 927 & 946 & 2037 & 1359 \\
LA07 & $15 \times 5$ & 890 & 980 & 1006 & 2181 & 1409 & & 973 & 1000 & 2056 & 1405 \\
LA08 & $15 \times 5$ & 863 & 920 & 967 & 2250 & 1375 & & 917 & 947 & 2087 & 1364 \\
LA09 & $15 \times 5$ & 951 & 972 & 982 & 2345 & 1432 & & 958 & 973 & 2196 & 1422 \\
LA10 & $15 \times 5$ & 958 & 958 & 958 & 2263 & 1349 & & 958 & 958 & 2039 & 1338 \\
LA11 & $20 \times 5$ & 1222 & 1267 & 1282 & 2750 & 1753 & & 1255 & 1286 & 2484 & 1746 \\
LA12 & $20 \times 5$ & 1039 & 1072 & 1093 & 2508 & 1551 & & 1059 & 1086 & 2370 & 1547 \\
LA13 & $20 \times 5$ & 1150 & 1164 & 1205 & 2723 & 1704 & & 1182 & 1203 & 2449 & 1690 \\
LA14 & $20 \times 5$ & 1292 & 1292 & 1292 & 2745 & 1737 & & 1292 & 1292 & 2529 & 1739 \\
LA15 & $20 \times 5$ & 1207 & 1353 & 1416 & 2836 & 1893 & & 1368 & 1396 & 2730 & 1885 \\
\bottomrule
\end{tabular}
\end{sidewaystable}

\cleardoublepage
\begin{sidewaystable}
\caption{Resultados do caso de experimento 6}
\centering
\label{experimento6}
\begin{tabular}{cccccccccccc}
\toprule
& & & \multicolumn{4}{c}{GA/EV-JB} & & \multicolumn{4}{c}{IVF/EV-JB} \\
\cmidrule(c){4-7}
\cmidrule(c){9-12}
Inst\^{a}ncia & Tamanho ($n \times m$) & BKS & Melhor & M\'{e}dia (melhores) & Pior & M\'{e}dia (pop.) & & Melhor & M\'{e}dia (melhores) & Pior & M\'{e}dia (pop.) \\
\midrule
FT06 & $6 \times 6$ & 55 & 55 & 56 & 160 & 90 & & 55 & 56 & 136 & 85 \\
FT10 & $10 \times 10$ & 930 & 1162 & 1229 & 2713 & 1719 & & 1148 & 1210 & 2456 & 1710 \\
FT20 & $20 \times 5$ & 1165 & 1517 & 1555 & 3066 & 2081 & & 1507 & 1535 & 2935 & 2073 \\
LA01 & $10 \times 5$ & 666 & 687 & 718 & 1821 & 1058 & & 681 & 692 & 1642 & 1050 \\
LA02 & $10 \times 5$ & 655 & 724 & 754 & 1755 & 1084 & & 714 & 747 & 1596 & 1067 \\
LA03 & $10 \times 5$ & 597 & 668 & 688 & 1703 & 995 & & 654 & 678 & 1483 & 988 \\
LA04 & $10 \times 5$ & 590 & 651 & 674 & 1721 & 1005 & & 627 & 660 & 1551 & 992 \\
LA05 & $10 \times 5$ & 593 & 593 & 593 & 1560 & 888 & & 593 & 593 & 1376 & 876 \\
LA06 & $15 \times 5$ & 926 & 927 & 949 & 2244 & 1365 & & 928 & 943 & 2178 & 1359 \\
LA07 & $15 \times 5$ & 890 & 957 & 981 & 2311 & 1420 & & 973 & 997 & 2200 & 1395 \\
LA08 & $15 \times 5$ & 863 & 931 & 946 & 2255 & 1369 & & 913 & 960 & 2017 & 1366 \\
LA09 & $15 \times 5$ & 951 & 951 & 984 & 2418 & 1439 & & 951 & 983 & 2164 & 1415 \\
LA10 & $15 \times 5$ & 958 & 958 & 960 & 2205 & 1346 & & 958 & 958 & 1988 & 1340 \\
LA11 & $20 \times 5$ & 1222 & 1269 & 1304 & 2790 & 1762 & & 1231 & 1283 & 2505 & 1747 \\
LA12 & $20 \times 5$ & 1039 & 1081 & 1094 & 2658 & 1548 & & 1047 & 1074 & 2264 & 1544 \\
LA13 & $20 \times 5$ & 1150 & 1188 & 1218 & 2792 & 1701 & & 1189 & 1207 & 2448 & 1696 \\
LA14 & $20 \times 5$ & 1292 & 1292 & 1292 & 2772 & 1735 & & 1292 & 1292 & 2559 & 1731 \\
LA15 & $20 \times 5$ & 1207 & 1379 & 1420 & 2816 & 1887 & & 1349 & 1383 & 2793 & 1887 \\
\bottomrule
\end{tabular}
\end{sidewaystable}

\cleardoublepage
\begin{sidewaystable}
\caption{Resultados do caso de experimento 7}
\centering
\label{experimento7}
\begin{tabular}{cccccccccccc}
\toprule
& & & \multicolumn{4}{c}{GA/EV-JB} & & \multicolumn{4}{c}{IVF/EV-JB} \\
\cmidrule(c){4-7}
\cmidrule(c){9-12}
Inst\^{a}ncia & Tamanho ($n \times m$) & BKS & Melhor & M\'{e}dia (melhores) & Pior & M\'{e}dia (pop.) & & Melhor & M\'{e}dia (melhores) & Pior & M\'{e}dia (pop.) \\
\midrule
FT06 & $6 \times 6$ & 55 & 55 & 55 & 154 & 79 & & 55 & 55 & 133 & 79 \\
FT10 & $10 \times 10$ & 930 & 1005 & 1053 & 2638 & 1642 & & 980 & 1020 & 2407 & 1580 \\
FT20 & $20 \times 5$ & 1165 & 1303 & 1347 & 3053 & 1948 & & 1252 & 1267 & 2765 & 1817 \\
LA01 & $10 \times 5$ & 666 & 666 & 666 & 1728 & 967 & & 666 & 671 & 1585 & 927 \\
LA02 & $10 \times 5$ & 655 & 657 & 693 & 1703 & 1006 & & 658 & 671 & 1590 & 920 \\
LA03 & $10 \times 5$ & 597 & 619 & 627 & 1632 & 900 & & 604 & 628 & 1416 & 873 \\
LA04 & $10 \times 5$ & 590 & 598 & 611 & 1787 & 914 & & 595 & 610 & 1491 & 880 \\
LA05 & $10 \times 5$ & 593 & 593 & 593 & 1540 & 779 & & 593 & 593 & 1347 & 774 \\
LA06 & $15 \times 5$ & 926 & 926 & 926 & 2169 & 1275 & & 926 & 926 & 2033 & 1270 \\
LA07 & $15 \times 5$ & 890 & 890 & 912 & 2161 & 1271 & & 890 & 890 & 1899 & 1250 \\
LA08 & $15 \times 5$ & 863 & 863 & 868 & 2342 & 1266 & & 863 & 863 & 2008 & 1235 \\
LA09 & $15 \times 5$ & 951 & 951 & 951 & 2314 & 1358 & & 951 & 951 & 2100 & 1283 \\
LA10 & $15 \times 5$ & 958 & 958 & 958 & 2238 & 1258 & & 958 & 958 & 1943 & 1235 \\
LA11 & $20 \times 5$ & 1222 & 1222 & 1222 & 2796 & 1674 & & 1222 & 1222 & 2571 & 1644 \\
LA12 & $20 \times 5$ & 1039 & 1039 & 1039 & 2374 & 1469 & & 1039 & 1039 & 2152 & 1413 \\
LA13 & $20 \times 5$ & 1150 & 1150 & 1150 & 2574 & 1607 & & 1150 & 1150 & 2384 & 1586 \\
LA14 & $20 \times 5$ & 1292 & 1292 & 1292 & 2791 & 1673 & & 1292 & 1292 & 2433 & 1635 \\
LA15 & $20 \times 5$ & 1207 & 1231 & 1297 & 2855 & 1813 & & 1209 & 1239 & 2703 & 1784 \\
\bottomrule
\end{tabular}
\end{sidewaystable}

\cleardoublepage
\begin{sidewaystable}
\caption{Resultados do caso de experimento 8}
\centering
\label{experimento8}
\begin{tabular}{cccccccccccc}
\toprule
& & & \multicolumn{4}{c}{GA/EV-JB} & & \multicolumn{4}{c}{IVF/EV-JB} \\
\cmidrule(c){4-7}
\cmidrule(c){9-12}
Inst\^{a}ncia & Tamanho ($n \times m$) & BKS & Melhor & M\'{e}dia (melhores) & Pior & M\'{e}dia (pop.) & & Melhor & M\'{e}dia (melhores) & Pior & M\'{e}dia (pop.) \\
\midrule
FT06 & $6 \times 6$ & 55 & 55 & 55 & 166 & 85 & & 55 & 55 & 138 & 83 \\
FT10 & $10 \times 10$ & 930 & 1131 & 1167 & 2706 & 1690 & & 1059 & 1115 & 2422 & 1693 \\
FT20 & $20 \times 5$ & 1165 & 1356 & 1459 & 3106 & 2004 & & 1299 & 1413 & 2975 & 2026 \\
LA01 & $10 \times 5$ & 666 & 666 & 685 & 1839 & 1017 & & 666 & 675 & 1569 & 988 \\
LA02 & $10 \times 5$ & 655 & 691 & 727 & 1764 & 1061 & & 684 & 688 & 1540 & 1034 \\
LA03 & $10 \times 5$ & 597 & 626 & 654 & 1716 & 965 & & 632 & 649 & 1498 & 957 \\
LA04 & $10 \times 5$ & 590 & 613 & 624 & 1710 & 932 & & 607 & 624 & 1525 & 957 \\
LA05 & $10 \times 5$ & 593 & 593 & 593 & 1518 & 724 & & 593 & 593 & 1380 & 831 \\
LA06 & $15 \times 5$ & 926 & 926 & 933 & 2203 & 1336 & & 926 & 926 & 1985 & 1315 \\
LA07 & $15 \times 5$ & 890 & 919 & 944 & 2166 & 1360 & & 892 & 913 & 2004 & 1347 \\
LA08 & $15 \times 5$ & 863 & 890 & 927 & 2230 & 1343 & & 863 & 893 & 1992 & 1334 \\
LA09 & $15 \times 5$ & 951 & 951 & 962 & 2360 & 1391 & & 951 & 951 & 2109 & 1372 \\
LA10 & $15 \times 5$ & 958 & 958 & 958 & 2199 & 1305 & & 958 & 958 & 2060 & 1315 \\
LA11 & $20 \times 5$ & 1222 & 1238 & 1280 & 2770 & 1736 & & 1222 & 1242 & 2517 & 1729 \\
LA12 & $20 \times 5$ & 1039 & 1041 & 1088 & 2407 & 1518 & & 1039 & 1058 & 2312 & 1503 \\
LA13 & $20 \times 5$ & 1150 & 1150 & 1190 & 2740 & 1677 & & 1150 & 1182 & 2444 & 1664 \\
LA14 & $20 \times 5$ & 1292 & 1292 & 1292 & 2842 & 1689 & & 1292 & 1292 & 2534 & 1712 \\
LA15 & $20 \times 5$ & 1207 & 1302 & 1357 & 2783 & 1840 & & 1263 & 1322 & 2667 & 1858 \\
\bottomrule
\end{tabular}
\end{sidewaystable}

\cleardoublepage
\begin{sidewaystable}
\caption{Resultados do caso de experimento 9}
\centering
\label{experimento9}
\begin{tabular}{cccccccccccc}
\toprule
& & & \multicolumn{4}{c}{GA/EV-JB} & & \multicolumn{4}{c}{IVF/EV-JB} \\
\cmidrule(c){4-7}
\cmidrule(c){9-12}
Inst\^{a}ncia & Tamanho ($n \times m$) & BKS & Melhor & M\'{e}dia (melhores) & Pior & M\'{e}dia (pop.) & & Melhor & M\'{e}dia (melhores) & Pior & M\'{e}dia (pop.) \\
\midrule
FT06 & $6 \times 6$ & 55 & 55 & 55 & 155 & 86 & & 55 & 56 & 135 & 83 \\
FT10 & $10 \times 10$ & 930 & 1102 & 1188 & 2658 & 1701 & & 1063 & 1151 & 2399 & 1698 \\
FT20 & $20 \times 5$ & 1165 & 1413 & 1480 & 3129 & 2033 & & 1342 & 1437 & 2784 & 2022 \\
LA01 & $10 \times 5$ & 666 & 666 & 667 & 1823 & 989 & & 666 & 666 & 1622 & 1007 \\
LA02 & $10 \times 5$ & 655 & 704 & 734 & 1740 & 1061 & & 672 & 704 & 1566 & 1035 \\
LA03 & $10 \times 5$ & 597 & 638 & 676 & 1633 & 974 & & 617 & 644 & 1612 & 963 \\
LA04 & $10 \times 5$ & 590 & 624 & 635 & 1703 & 952 & & 607 & 621 & 1565 & 930 \\
LA05 & $10 \times 5$ & 593 & 593 & 593 & 1549 & 802 & & 593 & 593 & 1416 & 762 \\
LA06 & $15 \times 5$ & 926 & 926 & 936 & 2279 & 1325 & & 926 & 934 & 1985 & 1333 \\
LA07 & $15 \times 5$ & 890 & 919 & 946 & 2201 & 1368 & & 894 & 927 & 2046 & 1356 \\
LA08 & $15 \times 5$ & 863 & 875 & 905 & 2271 & 1347 & & 863 & 880 & 2013 & 1334 \\
LA09 & $15 \times 5$ & 951 & 951 & 968 & 2346 & 1387 & & 951 & 956 & 2076 & 1373 \\
LA10 & $15 \times 5$ & 958 & 958 & 958 & 2246 & 1225 & & 958 & 958 & 2026 & 1282 \\
LA11 & $20 \times 5$ & 1222 & 1231 & 1271 & 2930 & 1740 & & 1222 & 1235 & 2479 & 1720 \\
LA12 & $20 \times 5$ & 1039 & 1039 & 1061 & 2443 & 1526 & & 1039 & 1041 & 2245 & 1514 \\
LA13 & $20 \times 5$ & 1150 & 1150 & 1163 & 2705 & 1668 & & 1150 & 1157 & 2405 & 1666 \\
LA14 & $20 \times 5$ & 1292 & 1292 & 1292 & 2920 & 1725 & & 1292 & 1292 & 2449 & 1713 \\
LA15 & $20 \times 5$ & 1207 & 1311 & 1386 & 2987 & 1867 & & 1308 & 1352 & 2672 & 1849 \\
\bottomrule
\end{tabular}
\end{sidewaystable}

\cleardoublepage
\begin{sidewaystable}
\caption{Resultados do caso de experimento 10}
\centering
\label{experimento10}
\begin{tabular}{cccccccccccc}
\toprule
& & & \multicolumn{4}{c}{GA/EV-JB} & & \multicolumn{4}{c}{IVF/EV-JB} \\
\cmidrule(c){4-7}
\cmidrule(c){9-12}
Inst\^{a}ncia & Tamanho ($n \times m$) & BKS & Melhor & M\'{e}dia (melhores) & Pior & M\'{e}dia (pop.) & & Melhor & M\'{e}dia (melhores) & Pior & M\'{e}dia (pop.) \\
\midrule
FT06 & $6 \times 6$ & 55 & 57 & 58 & 149 & 63 & & 57 & 57 & 136 & 62 \\
FT10 & $10 \times 10$ & 930 & 1204 & 1258 & 2633 & 1351 & & 1179 & 1250 & 2473 & 1337 \\
FT20 & $20 \times 5$ & 1165 & 1547 & 1566 & 3343 & 1670 & & 1482 & 1554 & 2869 & 1653 \\
LA01 & $10 \times 5$ & 666 & 715 & 728 & 1753 & 795 & & 706 & 745 & 1543 & 805 \\
LA02 & $10 \times 5$ & 655 & 747 & 773 & 1714 & 835 & & 757 & 764 & 1567 & 821 \\
LA03 & $10 \times 5$ & 597 & 680 & 685 & 1612 & 747 & & 675 & 708 & 1440 & 762 \\
LA04 & $10 \times 5$ & 590 & 632 & 697 & 1717 & 760 & & 661 & 689 & 1457 & 747 \\
LA05 & $10 \times 5$ & 593 & 593 & 593 & 1484 & 648 & & 593 & 593 & 1365 & 641 \\
LA06 & $15 \times 5$ & 926 & 937 & 968 & 2296 & 1046 & & 940 & 975 & 2086 & 1046 \\
LA07 & $15 \times 5$ & 890 & 999 & 1024 & 2257 & 1100 & & 993 & 1024 & 1984 & 1092 \\
LA08 & $15 \times 5$ & 863 & 947 & 972 & 2207 & 1052 & & 938 & 969 & 2000 & 1043 \\
LA09 & $15 \times 5$ & 951 & 960 & 1008 & 2331 & 1090 & & 991 & 1005 & 2116 & 1084 \\
LA10 & $15 \times 5$ & 958 & 958 & 964 & 2360 & 1044 & & 958 & 978 & 1956 & 1049 \\
LA11 & $20 \times 5$ & 1222 & 1276 & 1322 & 2730 & 1410 & & 1257 & 1320 & 2435 & 1399 \\
LA12 & $20 \times 5$ & 1039 & 1098 & 1127 & 2497 & 1212 & & 1063 & 1110 & 2225 & 1191 \\
LA13 & $20 \times 5$ & 1150 & 1219 & 1247 & 2641 & 1338 & & 1210 & 1256 & 2455 & 1340 \\
LA14 & $20 \times 5$ & 1292 & 1292 & 1306 & 2698 & 1394 & & 1292 & 1292 & 2488 & 1377 \\
LA15 & $20 \times 5$ & 1207 & 1340 & 1419 & 2758 & 1512 & & 1416 & 1450 & 2735 & 1537 \\
\bottomrule
\end{tabular}
\end{sidewaystable}

\cleardoublepage
\begin{sidewaystable}
\caption{Resultados do caso de experimento 11}
\centering
\label{experimento11}
\begin{tabular}{cccccccccccc}
\toprule
& & & \multicolumn{4}{c}{GA/EV-JB} & & \multicolumn{4}{c}{IVF/EV-JB} \\
\cmidrule(c){4-7}
\cmidrule(c){9-12}
Inst\^{a}ncia & Tamanho ($n \times m$) & BKS & Melhor & M\'{e}dia (melhores) & Pior & M\'{e}dia (pop.) & & Melhor & M\'{e}dia (melhores) & Pior & M\'{e}dia (pop.) \\
\midrule
FT06 & $6 \times 6$ & 55 & 58 & 58 & 153 & 63 & & 56 & 58 & 133 & 63 \\
FT10 & $10 \times 10$ & 930 & 1162 & 1247 & 2632 & 1344 & & 1204 & 1249 & 2353 & 1338 \\
FT20 & $20 \times 5$ & 1165 & 1538 & 1558 & 3000 & 1668 & & 1511 & 1563 & 2783 & 1660 \\
LA01 & $10 \times 5$ & 666 & 690 & 736 & 1744 & 801 & & 702 & 738 & 1557 & 798 \\
LA02 & $10 \times 5$ & 655 & 743 & 781 & 1698 & 842 & & 739 & 766 & 1563 & 822 \\
LA03 & $10 \times 5$ & 597 & 672 & 688 & 1602 & 749 & & 679 & 700 & 1459 & 753 \\
LA04 & $10 \times 5$ & 590 & 640 & 688 & 1616 & 754 & & 663 & 692 & 1490 & 752 \\
LA05 & $10 \times 5$ & 593 & 593 & 593 & 1479 & 648 & & 593 & 594 & 1312 & 644 \\
LA06 & $15 \times 5$ & 926 & 935 & 962 & 2127 & 1037 & & 926 & 977 & 1936 & 1046 \\
LA07 & $15 \times 5$ & 890 & 998 & 1027 & 2145 & 1103 & & 962 & 1027 & 1998 & 1098 \\
LA08 & $15 \times 5$ & 863 & 909 & 958 & 2269 & 1041 & & 915 & 968 & 2005 & 1047 \\
LA09 & $15 \times 5$ & 951 & 972 & 997 & 2273 & 1085 & & 965 & 996 & 2065 & 1076 \\
LA10 & $15 \times 5$ & 958 & 958 & 985 & 2212 & 1060 & & 958 & 972 & 1924 & 1044 \\
LA11 & $20 \times 5$ & 1222 & 1275 & 1299 & 2727 & 1387 & & 1258 & 1289 & 2413 & 1376 \\
LA12 & $20 \times 5$ & 1039 & 1061 & 1079 & 2436 & 1172 & & 1085 & 1121 & 2172 & 1204 \\
LA13 & $20 \times 5$ & 1150 & 1203 & 1257 & 2660 & 1344 & & 1202 & 1249 & 2412 & 1333 \\
LA14 & $20 \times 5$ & 1292 & 1292 & 1303 & 2722 & 1398 & & 1292 & 1292 & 2458 & 1378 \\
LA15 & $20 \times 5$ & 1207 & 1397 & 1425 & 2853 & 1519 & & 1385 & 1409 & 2704 & 1502 \\
\bottomrule
\end{tabular}
\end{sidewaystable}

\cleardoublepage
\begin{sidewaystable}
\caption{Resultados do caso de experimento 12}
\centering
\label{experimento12}
\begin{tabular}{cccccccccccc}
\toprule
& & & \multicolumn{4}{c}{GA/EV-JB} & & \multicolumn{4}{c}{IVF/EV-JB} \\
\cmidrule(c){4-7}
\cmidrule(c){9-12}
Inst\^{a}ncia & Tamanho ($n \times m$) & BKS & Melhor & M\'{e}dia (melhores) & Pior & M\'{e}dia (pop.) & & Melhor & M\'{e}dia (melhores) & Pior & M\'{e}dia (pop.) \\
\midrule
FT06 & $6 \times 6$ & 55 & 57 & 57 & 148 & 63 & & 55 & 59 & 145 & 64 \\
FT10 & $10 \times 10$ & 930 & 1190 & 1265 & 2561 & 1357 & & 1182 & 1210 & 2467 & 1307 \\
FT20 & $20 \times 5$ & 1165 & 1514 & 1566 & 2978 & 1668 & & 1504 & 1560 & 2859 & 1660 \\
LA01 & $10 \times 5$ & 666 & 675 & 730 & 1719 & 797 & & 675 & 704 & 1606 & 771 \\
LA02 & $10 \times 5$ & 655 & 751 & 765 & 1708 & 826 & & 720 & 780 & 1571 & 835 \\
LA03 & $10 \times 5$ & 597 & 674 & 683 & 1608 & 743 & & 683 & 698 & 1434 & 749 \\
LA04 & $10 \times 5$ & 590 & 653 & 695 & 1662 & 759 & & 653 & 689 & 1459 & 748 \\
LA05 & $10 \times 5$ & 593 & 593 & 593 & 1516 & 649 & & 593 & 593 & 1346 & 644 \\
LA06 & $15 \times 5$ & 926 & 940 & 973 & 2222 & 1053 & & 934 & 973 & 1916 & 1045 \\
LA07 & $15 \times 5$ & 890 & 978 & 1014 & 2183 & 1090 & & 986 & 1023 & 1938 & 1096 \\
LA08 & $15 \times 5$ & 863 & 950 & 975 & 2129 & 1057 & & 906 & 961 & 1937 & 1037 \\
LA09 & $15 \times 5$ & 951 & 957 & 999 & 2250 & 1083 & & 959 & 1023 & 2091 & 1100 \\
LA10 & $15 \times 5$ & 958 & 958 & 968 & 2208 & 1049 & & 958 & 977 & 1935 & 1047 \\
LA11 & $20 \times 5$ & 1222 & 1283 & 1309 & 2807 & 1395 & & 1275 & 1296 & 2379 & 1381 \\
LA12 & $20 \times 5$ & 1039 & 1067 & 1130 & 2446 & 1213 & & 1083 & 1117 & 2216 & 1200 \\
LA13 & $20 \times 5$ & 1150 & 1189 & 1238 & 2618 & 1328 & & 1213 & 1249 & 2336 & 1334 \\
LA14 & $20 \times 5$ & 1292 & 1292 & 1295 & 2765 & 1384 & & 1292 & 1294 & 2420 & 1378 \\
LA15 & $20 \times 5$ & 1207 & 1386 & 1428 & 2794 & 1519 & & 1399 & 1442 & 2597 & 1525 \\
\bottomrule
\end{tabular}
\end{sidewaystable}

\cleardoublepage
\begin{sidewaystable}
\caption{Resultados do caso de experimento 13}
\centering
\label{experimento13}
\begin{tabular}{cccccccccccc}
\toprule
& & & \multicolumn{4}{c}{GA/EV-JB} & & \multicolumn{4}{c}{IVF/EV-JB} \\
\cmidrule(c){4-7}
\cmidrule(c){9-12}
Inst\^{a}ncia & Tamanho ($n \times m$) & BKS & Melhor & M\'{e}dia (melhores) & Pior & M\'{e}dia (pop.) & & Melhor & M\'{e}dia (melhores) & Pior & M\'{e}dia (pop.) \\
\midrule
FT06 & $6 \times 6$ & 55 & 55 & 57 & 145 & 84 & & 55 & 56 & 145 & 82 \\
FT10 & $10 \times 10$ & 930 & 1172 & 1216 & 2640 & 1672 & & 1164 & 1200 & 2331 & 1670 \\
FT20 & $20 \times 5$ & 1165 & 1481 & 1543 & 3037 & 2026 & & 1470 & 1518 & 2841 & 2025 \\
LA01 & $10 \times 5$ & 666 & 668 & 709 & 1811 & 1021 & & 687 & 707 & 1660 & 1023 \\
LA02 & $10 \times 5$ & 655 & 727 & 763 & 1746 & 1035 & & 728 & 764 & 1740 & 1040 \\
LA03 & $10 \times 5$ & 597 & 668 & 690 & 1555 & 964 & & 655 & 678 & 1482 & 954 \\
LA04 & $10 \times 5$ & 590 & 648 & 670 & 1638 & 983 & & 632 & 666 & 1493 & 970 \\
LA05 & $10 \times 5$ & 593 & 593 & 593 & 1448 & 831 & & 593 & 593 & 1318 & 831 \\
LA06 & $15 \times 5$ & 926 & 936 & 954 & 2106 & 1320 & & 928 & 956 & 2054 & 1305 \\
LA07 & $15 \times 5$ & 890 & 972 & 995 & 2183 & 1363 & & 951 & 997 & 2019 & 1355 \\
LA08 & $15 \times 5$ & 863 & 928 & 947 & 2165 & 1329 & & 906 & 944 & 2018 & 1322 \\
LA09 & $15 \times 5$ & 951 & 951 & 989 & 2347 & 1384 & & 957 & 972 & 2111 & 1377 \\
LA10 & $15 \times 5$ & 958 & 958 & 961 & 2197 & 1321 & & 958 & 959 & 1954 & 1308 \\
LA11 & $20 \times 5$ & 1222 & 1269 & 1290 & 2637 & 1700 & & 1252 & 1270 & 2442 & 1697 \\
LA12 & $20 \times 5$ & 1039 & 1077 & 1101 & 2359 & 1504 & & 1052 & 1072 & 2185 & 1493 \\
LA13 & $20 \times 5$ & 1150 & 1197 & 1215 & 2643 & 1644 & & 1179 & 1192 & 2353 & 1648 \\
LA14 & $20 \times 5$ & 1292 & 1292 & 1292 & 2744 & 1692 & & 1292 & 1292 & 2419 & 1684 \\
LA15 & $20 \times 5$ & 1207 & 1371 & 1390 & 2805 & 1847 & & 1359 & 1407 & 2759 & 1839 \\
\bottomrule
\end{tabular}
\end{sidewaystable}

\cleardoublepage
\begin{sidewaystable}
\caption{Resultados do caso de experimento 14}
\centering
\label{experimento14}
\begin{tabular}{cccccccccccc}
\toprule
& & & \multicolumn{4}{c}{GA/EV-JB} & & \multicolumn{4}{c}{IVF/EV-JB} \\
\cmidrule(c){4-7}
\cmidrule(c){9-12}
Inst\^{a}ncia & Tamanho ($n \times m$) & BKS & Melhor & M\'{e}dia (melhores) & Pior & M\'{e}dia (pop.) & & Melhor & M\'{e}dia (melhores) & Pior & M\'{e}dia (pop.) \\
\midrule
FT06 & $6 \times 6$ & 55 & 55 & 57 & 150 & 84 & & 55 & 55 & 135 & 83 \\
FT10 & $10 \times 10$ & 930 & 1130 & 1213 & 2572 & 1678 & & 1173 & 1217 & 2520 & 1669 \\
FT20 & $20 \times 5$ & 1165 & 1493 & 1544 & 3063 & 2030 & & 1495 & 1518 & 2829 & 2034 \\
LA01 & $10 \times 5$ & 666 & 689 & 710 & 1748 & 1017 & & 680 & 693 & 1573 & 1016 \\
LA02 & $10 \times 5$ & 655 & 725 & 748 & 1697 & 1039 & & 732 & 752 & 1646 & 1039 \\
LA03 & $10 \times 5$ & 597 & 670 & 683 & 1597 & 956 & & 662 & 689 & 1395 & 953 \\
LA04 & $10 \times 5$ & 590 & 639 & 675 & 1618 & 982 & & 637 & 674 & 1455 & 972 \\
LA05 & $10 \times 5$ & 593 & 593 & 593 & 1559 & 840 & & 593 & 593 & 1300 & 828 \\
LA06 & $15 \times 5$ & 926 & 939 & 961 & 2220 & 1315 & & 926 & 948 & 2010 & 1303 \\
LA07 & $15 \times 5$ & 890 & 956 & 978 & 2165 & 1361 & & 955 & 974 & 2108 & 1353 \\
LA08 & $15 \times 5$ & 863 & 908 & 931 & 2184 & 1330 & & 927 & 935 & 1981 & 1331 \\
LA09 & $15 \times 5$ & 951 & 974 & 982 & 2347 & 1375 & & 954 & 980 & 2175 & 1375 \\
LA10 & $15 \times 5$ & 958 & 958 & 958 & 2172 & 1312 & & 958 & 960 & 1960 & 1311 \\
LA11 & $20 \times 5$ & 1222 & 1266 & 1285 & 2622 & 1707 & & 1247 & 1281 & 2536 & 1695 \\
LA12 & $20 \times 5$ & 1039 & 1055 & 1094 & 2426 & 1500 & & 1049 & 1100 & 2203 & 1497 \\
LA13 & $20 \times 5$ & 1150 & 1196 & 1232 & 2602 & 1639 & & 1159 & 1205 & 2566 & 1649 \\
LA14 & $20 \times 5$ & 1292 & 1292 & 1292 & 2565 & 1697 & & 1292 & 1292 & 2442 & 1688 \\
LA15 & $20 \times 5$ & 1207 & 1342 & 1411 & 3031 & 1841 & & 1369 & 1421 & 2531 & 1843 \\
\bottomrule
\end{tabular}
\end{sidewaystable}

\cleardoublepage
\begin{sidewaystable}
\caption{Resultados do caso de experimento 15}
\centering
\label{experimento15}
\begin{tabular}{cccccccccccc}
\toprule
& & & \multicolumn{4}{c}{GA/EV-JB} & & \multicolumn{4}{c}{IVF/EV-JB} \\
\cmidrule(c){4-7}
\cmidrule(c){9-12}
Inst\^{a}ncia & Tamanho ($n \times m$) & BKS & Melhor & M\'{e}dia (melhores) & Pior & M\'{e}dia (pop.) & & Melhor & M\'{e}dia (melhores) & Pior & M\'{e}dia (pop.) \\
\midrule
FT06 & $6 \times 6$ & 55 & 56 & 57 & 148 & 85 & & 55 & 56 & 134 & 83 \\
FT10 & $10 \times 10$ & 930 & 1174 & 1211 & 2582 & 1676 & & 1145 & 1202 & 2399 & 1664 \\
FT20 & $20 \times 5$ & 1165 & 1478 & 1554 & 2985 & 2032 & & 1494 & 1524 & 2807 & 2025 \\
LA01 & $10 \times 5$ & 666 & 678 & 709 & 1771 & 1024 & & 677 & 709 & 1600 & 1020 \\
LA02 & $10 \times 5$ & 655 & 735 & 752 & 1754 & 1035 & & 718 & 760 & 1539 & 1026 \\
LA03 & $10 \times 5$ & 597 & 664 & 691 & 1564 & 961 & & 666 & 679 & 1419 & 959 \\
LA04 & $10 \times 5$ & 590 & 649 & 664 & 1613 & 979 & & 619 & 663 & 1566 & 973 \\
LA05 & $10 \times 5$ & 593 & 593 & 593 & 1476 & 830 & & 593 & 593 & 1359 & 828 \\
LA06 & $15 \times 5$ & 926 & 926 & 946 & 2290 & 1311 & & 929 & 945 & 1940 & 1305 \\
LA07 & $15 \times 5$ & 890 & 975 & 983 & 2088 & 1364 & & 968 & 999 & 1972 & 1350 \\
LA08 & $15 \times 5$ & 863 & 919 & 938 & 2336 & 1329 & & 891 & 940 & 1971 & 1319 \\
LA09 & $15 \times 5$ & 951 & 957 & 983 & 2197 & 1387 & & 967 & 987 & 2012 & 1368 \\
LA10 & $15 \times 5$ & 958 & 958 & 959 & 2272 & 1317 & & 958 & 958 & 1974 & 1312 \\
LA11 & $20 \times 5$ & 1222 & 1259 & 1286 & 2760 & 1699 & & 1258 & 1279 & 2394 & 1702 \\
LA12 & $20 \times 5$ & 1039 & 1072 & 1091 & 2387 & 1498 & & 1041 & 1103 & 2154 & 1499 \\
LA13 & $20 \times 5$ & 1150 & 1193 & 1202 & 2657 & 1651 & & 1187 & 1210 & 2382 & 1646 \\
LA14 & $20 \times 5$ & 1292 & 1292 & 1292 & 2613 & 1699 & & 1292 & 1292 & 2452 & 1690 \\
LA15 & $20 \times 5$ & 1207 & 1391 & 1406 & 2806 & 1842 & & 1391 & 1430 & 2796 & 1841 \\
\bottomrule
\end{tabular}
\end{sidewaystable}

\cleardoublepage
\begin{sidewaystable}
\caption{Resultados do caso de experimento 16}
\centering
\label{experimento16}
\begin{tabular}{cccccccccccc}
\toprule
& & & \multicolumn{4}{c}{GA/EV-JB} & & \multicolumn{4}{c}{IVF/EV-JB} \\
\cmidrule(c){4-7}
\cmidrule(c){9-12}
Inst\^{a}ncia & Tamanho ($n \times m$) & BKS & Melhor & M\'{e}dia (melhores) & Pior & M\'{e}dia (pop.) & & Melhor & M\'{e}dia (melhores) & Pior & M\'{e}dia (pop.) \\
\midrule
FT06 & $6 \times 6$ & 55 & 55 & 58 & 151 & 63 & & 55 & 58 & 138 & 63 \\
FT10 & $10 \times 10$ & 930 & 1186 & 1225 & 2609 & 1342 & & 1187 & 1241 & 2340 & 1332 \\
FT20 & $20 \times 5$ & 1165 & 1517 & 1545 & 2990 & 1674 & & 1495 & 1564 & 2845 & 1662 \\
LA01 & $10 \times 5$ & 666 & 692 & 715 & 1739 & 785 & & 696 & 729 & 1510 & 788 \\
LA02 & $10 \times 5$ & 655 & 752 & 766 & 1742 & 829 & & 713 & 769 & 1529 & 826 \\
LA03 & $10 \times 5$ & 597 & 654 & 695 & 1565 & 755 & & 677 & 704 & 1466 & 754 \\
LA04 & $10 \times 5$ & 590 & 661 & 693 & 1705 & 758 & & 647 & 670 & 1577 & 735 \\
LA05 & $10 \times 5$ & 593 & 593 & 594 & 1508 & 648 & & 593 & 593 & 1346 & 644 \\
LA06 & $15 \times 5$ & 926 & 926 & 976 & 2167 & 1059 & & 951 & 982 & 1902 & 1057 \\
LA07 & $15 \times 5$ & 890 & 935 & 1023 & 2104 & 1100 & & 974 & 1019 & 1983 & 1092 \\
LA08 & $15 \times 5$ & 863 & 920 & 964 & 2194 & 1051 & & 938 & 957 & 1931 & 1035 \\
LA09 & $15 \times 5$ & 951 & 986 & 1014 & 2296 & 1097 & & 971 & 1014 & 2110 & 1094 \\
LA10 & $15 \times 5$ & 958 & 958 & 977 & 2159 & 1056 & & 958 & 965 & 2066 & 1039 \\
LA11 & $20 \times 5$ & 1222 & 1260 & 1312 & 2717 & 1411 & & 1269 & 1290 & 2420 & 1378 \\
LA12 & $20 \times 5$ & 1039 & 1070 & 1121 & 2450 & 1211 & & 1076 & 1125 & 2205 & 1205 \\
LA13 & $20 \times 5$ & 1150 & 1212 & 1244 & 2651 & 1334 & & 1210 & 1250 & 2340 & 1336 \\
LA14 & $20 \times 5$ & 1292 & 1292 & 1292 & 2672 & 1382 & & 1292 & 1292 & 2437 & 1377 \\
LA15 & $20 \times 5$ & 1207 & 1391 & 1461 & 2768 & 1546 & & 1378 & 1459 & 2570 & 1544 \\
\bottomrule
\end{tabular}
\end{sidewaystable}

\cleardoublepage
\begin{sidewaystable}
\caption{Resultados do caso de experimento 17}
\centering
\label{experimento17}
\begin{tabular}{cccccccccccc}
\toprule
& & & \multicolumn{4}{c}{GA/EV-JB} & & \multicolumn{4}{c}{IVF/EV-JB} \\
\cmidrule(c){4-7}
\cmidrule(c){9-12}
Inst\^{a}ncia & Tamanho ($n \times m$) & BKS & Melhor & M\'{e}dia (melhores) & Pior & M\'{e}dia (pop.) & & Melhor & M\'{e}dia (melhores) & Pior & M\'{e}dia (pop.) \\
\midrule
FT06 & $6 \times 6$ & 55 & 55 & 58 & 159 & 64 & & 55 & 58 & 136 & 64 \\
FT10 & $10 \times 10$ & 930 & 1183 & 1236 & 2559 & 1424 & & 1168 & 1250 & 2431 & 1342 \\
FT20 & $20 \times 5$ & 1165 & 1472 & 1513 & 3038 & 1623 & & 1515 & 1560 & 2806 & 1658 \\
LA01 & $10 \times 5$ & 666 & 701 & 722 & 1786 & 798 & & 699 & 732 & 1571 & 793 \\
LA02 & $10 \times 5$ & 655 & 737 & 772 & 1766 & 829 & & 722 & 773 & 1703 & 830 \\
LA03 & $10 \times 5$ & 597 & 672 & 704 & 1564 & 762 & & 663 & 700 & 1442 & 755 \\
LA04 & $10 \times 5$ & 590 & 637 & 683 & 1639 & 750 & & 667 & 691 & 1485 & 750 \\
LA05 & $10 \times 5$ & 593 & 593 & 594 & 1614 & 647 & & 593 & 594 & 1355 & 643 \\
LA06 & $15 \times 5$ & 926 & 931 & 974 & 2188 & 1057 & & 941 & 955 & 1921 & 1031 \\
LA07 & $15 \times 5$ & 890 & 978 & 1007 & 2166 & 1126 & & 980 & 1015 & 1957 & 1090 \\
LA08 & $15 \times 5$ & 863 & 917 & 979 & 2332 & 1064 & & 921 & 954 & 2005 & 1031 \\
LA09 & $15 \times 5$ & 951 & 975 & 1016 & 2437 & 1097 & & 961 & 1014 & 2049 & 1094 \\
LA10 & $15 \times 5$ & 958 & 958 & 965 & 2162 & 1046 & & 958 & 982 & 1986 & 1051 \\
LA11 & $20 \times 5$ & 1222 & 1255 & 1295 & 2653 & 1391 & & 1247 & 1292 & 2469 & 1379 \\
LA12 & $20 \times 5$ & 1039 & 1074 & 1104 & 2478 & 1207 & & 1081 & 1088 & 2165 & 1175 \\
LA13 & $20 \times 5$ & 1150 & 1200 & 1234 & 2711 & 1328 & & 1196 & 1240 & 2479 & 1327 \\
LA14 & $20 \times 5$ & 1292 & 1292 & 1293 & 2686 & 1384 & & 1292 & 1292 & 2419 & 1377 \\
LA15 & $20 \times 5$ & 1207 & 1407 & 1433 & 2951 & 1528 & & 1375 & 1427 & 2602 & 1516 \\
\bottomrule
\end{tabular}
\end{sidewaystable}

\cleardoublepage
\begin{sidewaystable}
\caption{Resultados do caso de experimento 18}
\centering
\label{experimento18}
\begin{tabular}{cccccccccccc}
\toprule
& & & \multicolumn{4}{c}{GA/EV-JB} & & \multicolumn{4}{c}{IVF/EV-JB} \\
\cmidrule(c){4-7}
\cmidrule(c){9-12}
Inst\^{a}ncia & Tamanho ($n \times m$) & BKS & Melhor & M\'{e}dia (melhores) & Pior & M\'{e}dia (pop.) & & Melhor & M\'{e}dia (melhores) & Pior & M\'{e}dia (pop.) \\
\midrule
FT06 & $6 \times 6$ & 55 & 55 & 57 & 158 & 63 & & 57 & 58 & 137 & 62 \\
FT10 & $10 \times 10$ & 930 & 1169 & 1236 & 2655 & 1350 & & 1185 & 1244 & 2414 & 1334 \\
FT20 & $20 \times 5$ & 1165 & 1457 & 1514 & 3017 & 1975 & & 1469 & 1567 & 2930 & 1665 \\
LA01 & $10 \times 5$ & 666 & 686 & 717 & 1807 & 787 & & 694 & 710 & 1601 & 776 \\
LA02 & $10 \times 5$ & 655 & 736 & 783 & 1814 & 840 & & 750 & 777 & 1552 & 834 \\
LA03 & $10 \times 5$ & 597 & 682 & 706 & 1588 & 762 & & 679 & 700 & 1402 & 753 \\
LA04 & $10 \times 5$ & 590 & 660 & 702 & 1660 & 765 & & 648 & 688 & 1486 & 751 \\
LA05 & $10 \times 5$ & 593 & 593 & 594 & 1499 & 650 & & 593 & 593 & 1282 & 643 \\
LA06 & $15 \times 5$ & 926 & 926 & 956 & 2320 & 1035 & & 931 & 967 & 2014 & 1041 \\
LA07 & $15 \times 5$ & 890 & 976 & 1002 & 2216 & 1084 & & 982 & 998 & 1974 & 1074 \\
LA08 & $15 \times 5$ & 863 & 936 & 955 & 2250 & 1039 & & 936 & 965 & 2000 & 1042 \\
LA09 & $15 \times 5$ & 951 & 966 & 987 & 2269 & 1073 & & 951 & 1002 & 2060 & 1081 \\
LA10 & $15 \times 5$ & 958 & 958 & 979 & 2194 & 1061 & & 958 & 964 & 1926 & 1039 \\
LA11 & $20 \times 5$ & 1222 & 1266 & 1301 & 2781 & 1621 & & 1263 & 1298 & 2437 & 1383 \\
LA12 & $20 \times 5$ & 1039 & 1085 & 1102 & 2420 & 1207 & & 1085 & 1113 & 2164 & 1196 \\
LA13 & $20 \times 5$ & 1150 & 1188 & 1222 & 2604 & 1448 & & 1197 & 1224 & 2323 & 1313 \\
LA14 & $20 \times 5$ & 1292 & 1292 & 1292 & 2723 & 1600 & & 1292 & 1292 & 2486 & 1378 \\
LA15 & $20 \times 5$ & 1207 & 1383 & 1419 & 2855 & 1645 & & 1388 & 1433 & 2617 & 1516 \\
\bottomrule
\end{tabular}
\end{sidewaystable}

\cleardoublepage
\begin{sidewaystable}
\caption{Resultados do caso de experimento 19}
\centering
\label{experimento19}
\begin{tabular}{cccccccccccc}
\toprule
& & & \multicolumn{4}{c}{GA/EV-JB} & & \multicolumn{4}{c}{IVF/EV-JB} \\
\cmidrule(c){4-7}
\cmidrule(c){9-12}
Inst\^{a}ncia & Tamanho ($n \times m$) & BKS & Melhor & M\'{e}dia (melhores) & Pior & M\'{e}dia (pop.) & & Melhor & M\'{e}dia (melhores) & Pior & M\'{e}dia (pop.) \\
\midrule
FT06 & $6 \times 6$ & 55 & 55 & 55 & 148 & 70 & & 55 & 56 & 134 & 69 \\
FT10 & $10 \times 10$ & 930 & 963 & 1047 & 2684 & 1435 & & 967 & 992 & 2356 & 1262 \\
FT20 & $20 \times 5$ & 1165 & 1180 & 1255 & 3417 & 1603 & & 1218 & 1251 & 2839 & 1516 \\
LA01 & $10 \times 5$ & 666 & 666 & 666 & 1705 & 814 & & 666 & 679 & 1636 & 841 \\
LA02 & $10 \times 5$ & 655 & 655 & 666 & 1805 & 865 & & 655 & 686 & 1554 & 872 \\
LA03 & $10 \times 5$ & 597 & 603 & 617 & 1655 & 782 & & 597 & 623 & 1455 & 783 \\
LA04 & $10 \times 5$ & 590 & 590 & 600 & 1650 & 767 & & 590 & 606 & 1484 & 776 \\
LA05 & $10 \times 5$ & 593 & 593 & 593 & 1447 & 719 & & 593 & 593 & 1410 & 719 \\
LA06 & $15 \times 5$ & 926 & 926 & 926 & 2120 & 1118 & & 926 & 926 & 1984 & 1085 \\
LA07 & $15 \times 5$ & 890 & 890 & 890 & 2110 & 1073 & & 890 & 890 & 1930 & 1094 \\
LA08 & $15 \times 5$ & 863 & 863 & 863 & 2138 & 1124 & & 863 & 863 & 2011 & 1047 \\
LA09 & $15 \times 5$ & 951 & 951 & 951 & 2299 & 1128 & & 951 & 951 & 2089 & 1133 \\
LA10 & $15 \times 5$ & 958 & 958 & 958 & 2261 & 1116 & & 958 & 958 & 2085 & 1142 \\
LA11 & $20 \times 5$ & 1222 & 1222 & 1222 & 2695 & 1402 & & 1222 & 1222 & 2386 & 1453 \\
LA12 & $20 \times 5$ & 1039 & 1039 & 1039 & 2464 & 1237 & & 1039 & 1039 & 2226 & 1263 \\
LA13 & $20 \times 5$ & 1150 & 1150 & 1150 & 2576 & 1376 & & 1150 & 1150 & 2455 & 1385 \\
LA14 & $20 \times 5$ & 1292 & 1292 & 1292 & 2697 & 1489 & & 1292 & 1292 & 2517 & 1474 \\
LA15 & $20 \times 5$ & 1207 & 1207 & 1207 & 2842 & 1503 & & 1207 & 1211 & 2529 & 1467 \\
\bottomrule
\end{tabular}
\end{sidewaystable}

\cleardoublepage
\begin{sidewaystable}
\caption{Resultados do caso de experimento 20}
\centering
\label{experimento20}
\begin{tabular}{cccccccccccc}
\toprule
& & & \multicolumn{4}{c}{GA/EV-JB} & & \multicolumn{4}{c}{IVF/EV-JB} \\
\cmidrule(c){4-7}
\cmidrule(c){9-12}
Inst\^{a}ncia & Tamanho ($n \times m$) & BKS & Melhor & M\'{e}dia (melhores) & Pior & M\'{e}dia (pop.) & & Melhor & M\'{e}dia (melhores) & Pior & M\'{e}dia (pop.) \\
\midrule
FT06 & $6 \times 6$ & 55 & 55 & 55 & 152 & 71 & & 55 & 55 & 142 & 82 \\
FT10 & $10 \times 10$ & 930 & 983 & 1037 & 2610 & 1459 & & 982 & 1058 & 2359 & 1566 \\
FT20 & $20 \times 5$ & 1165 & 1185 & 1301 & 3028 & 1765 & & 1240 & 1338 & 2903 & 1881 \\
LA01 & $10 \times 5$ & 666 & 666 & 677 & 1731 & 944 & & 666 & 671 & 1563 & 965 \\
LA02 & $10 \times 5$ & 655 & 655 & 668 & 1696 & 868 & & 676 & 722 & 1578 & 987 \\
LA03 & $10 \times 5$ & 597 & 597 & 634 & 1581 & 854 & & 613 & 627 & 1473 & 907 \\
LA04 & $10 \times 5$ & 590 & 590 & 606 & 1695 & 808 & & 598 & 614 & 1428 & 939 \\
LA05 & $10 \times 5$ & 593 & 593 & 593 & 1554 & 747 & & 593 & 593 & 1353 & 802 \\
LA06 & $15 \times 5$ & 926 & 926 & 926 & 2151 & 1193 & & 926 & 926 & 1942 & 1261 \\
LA07 & $15 \times 5$ & 890 & 890 & 897 & 2245 & 1173 & & 890 & 924 & 2046 & 1290 \\
LA08 & $15 \times 5$ & 863 & 863 & 863 & 2259 & 1203 & & 863 & 875 & 1925 & 1269 \\
LA09 & $15 \times 5$ & 951 & 951 & 951 & 2266 & 1225 & & 951 & 951 & 2058 & 1338 \\
LA10 & $15 \times 5$ & 958 & 958 & 958 & 2210 & 1134 & & 958 & 958 & 2003 & 1268 \\
LA11 & $20 \times 5$ & 1222 & 1222 & 1222 & 2755 & 1592 & & 1222 & 1222 & 2449 & 1590 \\
LA12 & $20 \times 5$ & 1039 & 1039 & 1039 & 2468 & 1320 & & 1039 & 1039 & 2171 & 1458 \\
LA13 & $20 \times 5$ & 1150 & 1150 & 1150 & 2639 & 1490 & & 1150 & 1150 & 2367 & 1564 \\
LA14 & $20 \times 5$ & 1292 & 1292 & 1292 & 2668 & 1543 & & 1292 & 1292 & 2421 & 1644 \\
LA15 & $20 \times 5$ & 1207 & 1207 & 1207 & 2870 & 1584 & & 1207 & 1244 & 2597 & 1742 \\
\bottomrule
\end{tabular}
\end{sidewaystable}

\cleardoublepage
\begin{sidewaystable}
\caption{Resultados do caso de experimento 21}
\centering
\label{experimento21}
\begin{tabular}{cccccccccccc}
\toprule
& & & \multicolumn{4}{c}{GA/EV-JB} & & \multicolumn{4}{c}{IVF/EV-JB} \\
\cmidrule(c){4-7}
\cmidrule(c){9-12}
Inst\^{a}ncia & Tamanho ($n \times m$) & BKS & Melhor & M\'{e}dia (melhores) & Pior & M\'{e}dia (pop.) & & Melhor & M\'{e}dia (melhores) & Pior & M\'{e}dia (pop.) \\
\midrule
FT06 & $6 \times 6$ & 55 & 55 & 55 & 149 & 77 & & 55 & 55 & 132 & 82 \\
FT10 & $10 \times 10$ & 930 & 1002 & 1046 & 2586 & 1623 & & 1023 & 1189 & 2405 & 1654 \\
FT20 & $20 \times 5$ & 1165 & 1208 & 1282 & 3073 & 1880 & & 1305 & 1370 & 2807 & 1966 \\
LA01 & $10 \times 5$ & 666 & 666 & 680 & 1804 & 958 & & 666 & 670 & 1582 & 979 \\
LA02 & $10 \times 5$ & 655 & 665 & 674 & 1684 & 999 & & 665 & 702 & 1512 & 1006 \\
LA03 & $10 \times 5$ & 597 & 603 & 627 & 1785 & 907 & & 617 & 635 & 1434 & 939 \\
LA04 & $10 \times 5$ & 590 & 590 & 619 & 1682 & 919 & & 607 & 632 & 1612 & 937 \\
LA05 & $10 \times 5$ & 593 & 593 & 593 & 1512 & 774 & & 593 & 593 & 1350 & 809 \\
LA06 & $15 \times 5$ & 926 & 926 & 926 & 2240 & 1248 & & 926 & 926 & 1908 & 1284 \\
LA07 & $15 \times 5$ & 890 & 890 & 908 & 2100 & 1283 & & 890 & 925 & 1967 & 1306 \\
LA08 & $15 \times 5$ & 863 & 863 & 873 & 2251 & 1227 & & 863 & 888 & 1919 & 1298 \\
LA09 & $15 \times 5$ & 951 & 951 & 951 & 2378 & 1332 & & 951 & 954 & 2070 & 1333 \\
LA10 & $15 \times 5$ & 958 & 958 & 958 & 2153 & 1199 & & 958 & 958 & 1912 & 1293 \\
LA11 & $20 \times 5$ & 1222 & 1222 & 1222 & 2693 & 1650 & & 1222 & 1230 & 2419 & 1659 \\
LA12 & $20 \times 5$ & 1039 & 1039 & 1039 & 2479 & 1432 & & 1039 & 1053 & 2203 & 1467 \\
LA13 & $20 \times 5$ & 1150 & 1150 & 1150 & 2624 & 1603 & & 1150 & 1178 & 2410 & 1597 \\
LA14 & $20 \times 5$ & 1292 & 1292 & 1292 & 2750 & 1588 & & 1292 & 1292 & 2511 & 1658 \\
LA15 & $20 \times 5$ & 1207 & 1207 & 1241 & 2829 & 1778 & & 1244 & 1312 & 2568 & 1829 \\
\bottomrule
\end{tabular}
\end{sidewaystable}

\cleardoublepage
\begin{sidewaystable}
\caption{Resultados do caso de experimento 22}
\centering
\label{experimento22}
\begin{tabular}{cccccccccccc}
\toprule
& & & \multicolumn{4}{c}{GA/EV-JB} & & \multicolumn{4}{c}{IVF/EV-JB} \\
\cmidrule(c){4-7}
\cmidrule(c){9-12}
Inst\^{a}ncia & Tamanho ($n \times m$) & BKS & Melhor & M\'{e}dia (melhores) & Pior & M\'{e}dia (pop.) & & Melhor & M\'{e}dia (melhores) & Pior & M\'{e}dia (pop.) \\
\midrule
FT06 & $6 \times 6$ & 55 & 55 & 57 & 149 & 83 & & 55 & 56 & 139 & 83 \\
FT10 & $10 \times 10$ & 930 & 1109 & 1197 & 2707 & 1669 & & 1140 & 1203 & 2428 & 1674 \\
FT20 & $20 \times 5$ & 1165 & 1492 & 1533 & 3000 & 2031 & & 1451 & 1514 & 2728 & 2025 \\
LA01 & $10 \times 5$ & 666 & 692 & 703 & 1731 & 1034 & & 683 & 697 & 1553 & 1025 \\
LA02 & $10 \times 5$ & 655 & 724 & 753 & 1760 & 1038 & & 691 & 745 & 1520 & 1038 \\
LA03 & $10 \times 5$ & 597 & 647 & 667 & 1632 & 954 & & 641 & 663 & 1423 & 953 \\
LA04 & $10 \times 5$ & 590 & 633 & 658 & 1678 & 974 & & 627 & 677 & 1484 & 975 \\
LA05 & $10 \times 5$ & 593 & 593 & 593 & 1452 & 835 & & 593 & 593 & 1355 & 823 \\
LA06 & $15 \times 5$ & 926 & 934 & 953 & 2121 & 1314 & & 926 & 947 & 1980 & 1313 \\
LA07 & $15 \times 5$ & 890 & 969 & 1008 & 2239 & 1362 & & 960 & 981 & 1976 & 1357 \\
LA08 & $15 \times 5$ & 863 & 903 & 939 & 2149 & 1329 & & 892 & 918 & 1932 & 1320 \\
LA09 & $15 \times 5$ & 951 & 952 & 989 & 2319 & 1382 & & 951 & 971 & 2226 & 1371 \\
LA10 & $15 \times 5$ & 958 & 958 & 959 & 2204 & 1313 & & 958 & 961 & 2030 & 1302 \\
LA11 & $20 \times 5$ & 1222 & 1237 & 1276 & 2654 & 1696 & & 1250 & 1274 & 2512 & 1696 \\
LA12 & $20 \times 5$ & 1039 & 1064 & 1082 & 2356 & 1495 & & 1056 & 1064 & 2207 & 1491 \\
LA13 & $20 \times 5$ & 1150 & 1172 & 1193 & 2663 & 1647 & & 1156 & 1200 & 2335 & 1639 \\
LA14 & $20 \times 5$ & 1292 & 1292 & 1292 & 2792 & 1685 & & 1292 & 1292 & 2513 & 1684 \\
LA15 & $20 \times 5$ & 1207 & 1365 & 1410 & 2942 & 1841 & & 1338 & 1367 & 2584 & 1837 \\
\bottomrule
\end{tabular}
\end{sidewaystable}

\cleardoublepage
\begin{sidewaystable}
\caption{Resultados do caso de experimento 23}
\centering
\label{experimento23}
\begin{tabular}{cccccccccccc}
\toprule
& & & \multicolumn{4}{c}{GA/EV-JB} & & \multicolumn{4}{c}{IVF/EV-JB} \\
\cmidrule(c){4-7}
\cmidrule(c){9-12}
Inst\^{a}ncia & Tamanho ($n \times m$) & BKS & Melhor & M\'{e}dia (melhores) & Pior & M\'{e}dia (pop.) & & Melhor & M\'{e}dia (melhores) & Pior & M\'{e}dia (pop.) \\
\midrule
FT06 & $6 \times 6$ & 55 & 55 & 57 & 150 & 84 & & 55 & 56 & 140 & 82 \\
FT10 & $10 \times 10$ & 930 & 1169 & 1206 & 2631 & 1676 & & 1152 & 1176 & 2474 & 1667 \\
FT20 & $20 \times 5$ & 1165 & 1448 & 1547 & 3127 & 2036 & & 1494 & 1525 & 2792 & 2025 \\
LA01 & $10 \times 5$ & 666 & 680 & 718 & 1898 & 1019 & & 675 & 701 & 1599 & 1018 \\
LA02 & $10 \times 5$ & 655 & 718 & 741 & 1701 & 1038 & & 728 & 745 & 1590 & 1033 \\
LA03 & $10 \times 5$ & 597 & 655 & 688 & 1629 & 961 & & 652 & 685 & 1435 & 959 \\
LA04 & $10 \times 5$ & 590 & 642 & 663 & 1789 & 972 & & 635 & 660 & 1522 & 967 \\
LA05 & $10 \times 5$ & 593 & 593 & 593 & 1501 & 835 & & 593 & 593 & 1360 & 831 \\
LA06 & $15 \times 5$ & 926 & 940 & 954 & 2151 & 1317 & & 926 & 950 & 2040 & 1317 \\
LA07 & $15 \times 5$ & 890 & 956 & 988 & 2274 & 1358 & & 955 & 986 & 1958 & 1360 \\
LA08 & $15 \times 5$ & 863 & 912 & 939 & 2361 & 1327 & & 905 & 950 & 1950 & 1327 \\
LA09 & $15 \times 5$ & 951 & 957 & 993 & 2338 & 1381 & & 962 & 985 & 2042 & 1373 \\
LA10 & $15 \times 5$ & 958 & 958 & 965 & 2200 & 1311 & & 958 & 961 & 2066 & 1307 \\
LA11 & $20 \times 5$ & 1222 & 1243 & 1285 & 2653 & 1703 & & 1245 & 1286 & 2379 & 1700 \\
LA12 & $20 \times 5$ & 1039 & 1080 & 1103 & 2591 & 1499 & & 1039 & 1082 & 2258 & 1502 \\
LA13 & $20 \times 5$ & 1150 & 1190 & 1217 & 2710 & 1652 & & 1174 & 1208 & 2388 & 1639 \\
LA14 & $20 \times 5$ & 1292 & 1292 & 1293 & 2861 & 1691 & & 1292 & 1292 & 2524 & 1682 \\
LA15 & $20 \times 5$ & 1207 & 1371 & 1410 & 2884 & 1846 & & 1363 & 1405 & 2616 & 1837 \\
\bottomrule
\end{tabular}
\end{sidewaystable}

\cleardoublepage
\begin{sidewaystable}
\caption{Resultados do caso de experimento 24}
\centering
\label{experimento24}
\begin{tabular}{cccccccccccc}
\toprule
& & & \multicolumn{4}{c}{GA/EV-JB} & & \multicolumn{4}{c}{IVF/EV-JB} \\
\cmidrule(c){4-7}
\cmidrule(c){9-12}
Inst\^{a}ncia & Tamanho ($n \times m$) & BKS & Melhor & M\'{e}dia (melhores) & Pior & M\'{e}dia (pop.) & & Melhor & M\'{e}dia (melhores) & Pior & M\'{e}dia (pop.) \\
\midrule
FT06 & $6 \times 6$ & 55 & 55 & 57 & 147 & 83 & & 55 & 56 & 129 & 83 \\
FT10 & $10 \times 10$ & 930 & 1164 & 1200 & 2572 & 1681 & & 1145 & 1203 & 2365 & 1671 \\
FT20 & $20 \times 5$ & 1165 & 1509 & 1539 & 3195 & 2028 & & 1502 & 1539 & 2859 & 2023 \\
LA01 & $10 \times 5$ & 666 & 678 & 700 & 1904 & 1026 & & 679 & 710 & 1528 & 1016 \\
LA02 & $10 \times 5$ & 655 & 730 & 749 & 1709 & 1039 & & 722 & 756 & 1559 & 1035 \\
LA03 & $10 \times 5$ & 597 & 660 & 684 & 1602 & 962 & & 662 & 689 & 1436 & 955 \\
LA04 & $10 \times 5$ & 590 & 651 & 676 & 1640 & 976 & & 633 & 676 & 1461 & 977 \\
LA05 & $10 \times 5$ & 593 & 593 & 593 & 1506 & 836 & & 593 & 593 & 1347 & 831 \\
LA06 & $15 \times 5$ & 926 & 928 & 952 & 2112 & 1314 & & 926 & 939 & 2063 & 1311 \\
LA07 & $15 \times 5$ & 890 & 968 & 991 & 2115 & 1359 & & 956 & 989 & 1962 & 1357 \\
LA08 & $15 \times 5$ & 863 & 920 & 947 & 2167 & 1327 & & 923 & 949 & 2016 & 1322 \\
LA09 & $15 \times 5$ & 951 & 956 & 989 & 2234 & 1382 & & 951 & 986 & 2012 & 1373 \\
LA10 & $15 \times 5$ & 958 & 958 & 960 & 2242 & 1309 & & 958 & 958 & 2040 & 1306 \\
LA11 & $20 \times 5$ & 1222 & 1268 & 1292 & 2612 & 1702 & & 1245 & 1285 & 2468 & 1696 \\
LA12 & $20 \times 5$ & 1039 & 1078 & 1099 & 2441 & 1496 & & 1054 & 1095 & 2220 & 1494 \\
LA13 & $20 \times 5$ & 1150 & 1170 & 1202 & 2620 & 1650 & & 1159 & 1221 & 2307 & 1640 \\
LA14 & $20 \times 5$ & 1292 & 1292 & 1292 & 2593 & 1689 & & 1292 & 1292 & 2460 & 1683 \\
LA15 & $20 \times 5$ & 1207 & 1384 & 1420 & 2948 & 1848 & & 1379 & 1406 & 2587 & 1842 \\
\bottomrule
\end{tabular}
\end{sidewaystable}

\cleardoublepage
\begin{sidewaystable}
\caption{Resultados do caso de experimento 25}
\centering
\label{experimento25}
\begin{tabular}{cccccccccccc}
\toprule
& & & \multicolumn{4}{c}{GA/EV-JB} & & \multicolumn{4}{c}{IVF/EV-JB} \\
\cmidrule(c){4-7}
\cmidrule(c){9-12}
Inst\^{a}ncia & Tamanho ($n \times m$) & BKS & Melhor & M\'{e}dia (melhores) & Pior & M\'{e}dia (pop.) & & Melhor & M\'{e}dia (melhores) & Pior & M\'{e}dia (pop.) \\
\midrule
FT06 & $6 \times 6$ & 55 & 55 & 55 & 151 & 81 & & 55 & 55 & 136 & 74 \\
FT10 & $10 \times 10$ & 930 & 1036 & 1125 & 2580 & 1663 & & 988 & 1032 & 2409 & 1576 \\
FT20 & $20 \times 5$ & 1165 & 1255 & 1367 & 3201 & 1910 & & 1261 & 1285 & 2927 & 1817 \\
LA01 & $10 \times 5$ & 666 & 666 & 668 & 1823 & 971 & & 666 & 666 & 1699 & 922 \\
LA02 & $10 \times 5$ & 655 & 659 & 673 & 1789 & 982 & & 663 & 681 & 1515 & 958 \\
LA03 & $10 \times 5$ & 597 & 613 & 642 & 1565 & 890 & & 603 & 628 & 1529 & 856 \\
LA04 & $10 \times 5$ & 590 & 593 & 614 & 1630 & 878 & & 590 & 617 & 1553 & 898 \\
LA05 & $10 \times 5$ & 593 & 593 & 593 & 1476 & 799 & & 593 & 593 & 1335 & 742 \\
LA06 & $15 \times 5$ & 926 & 926 & 927 & 2238 & 1289 & & 926 & 926 & 1923 & 1229 \\
LA07 & $15 \times 5$ & 890 & 890 & 930 & 2139 & 1323 & & 890 & 891 & 1942 & 1237 \\
LA08 & $15 \times 5$ & 863 & 863 & 889 & 2208 & 1252 & & 863 & 866 & 2026 & 1218 \\
LA09 & $15 \times 5$ & 951 & 951 & 955 & 2280 & 1335 & & 951 & 951 & 2115 & 1274 \\
LA10 & $15 \times 5$ & 958 & 958 & 958 & 2377 & 1280 & & 958 & 958 & 2000 & 1235 \\
LA11 & $20 \times 5$ & 1222 & 1222 & 1222 & 2669 & 1682 & & 1222 & 1222 & 2427 & 1619 \\
LA12 & $20 \times 5$ & 1039 & 1039 & 1043 & 2432 & 1482 & & 1039 & 1039 & 2187 & 1424 \\
LA13 & $20 \times 5$ & 1150 & 1150 & 1157 & 2524 & 1619 & & 1150 & 1150 & 2293 & 1569 \\
LA14 & $20 \times 5$ & 1292 & 1292 & 1292 & 2720 & 1677 & & 1292 & 1292 & 2449 & 1637 \\
LA15 & $20 \times 5$ & 1207 & 1255 & 1295 & 2779 & 1821 & & 1207 & 1242 & 2583 & 1731 \\
\bottomrule
\end{tabular}
\end{sidewaystable}

\cleardoublepage
\begin{sidewaystable}
\caption{Resultados do caso de experimento 26}
\centering
\label{experimento26}
\begin{tabular}{cccccccccccc}
\toprule
& & & \multicolumn{4}{c}{GA/EV-JB} & & \multicolumn{4}{c}{IVF/EV-JB} \\
\cmidrule(c){4-7}
\cmidrule(c){9-12}
Inst\^{a}ncia & Tamanho ($n \times m$) & BKS & Melhor & M\'{e}dia (melhores) & Pior & M\'{e}dia (pop.) & & Melhor & M\'{e}dia (melhores) & Pior & M\'{e}dia (pop.) \\
\midrule
FT06 & $6 \times 6$ & 55 & 55 & 55 & 146 & 82 & & 55 & 55 & 146 & 80 \\
FT10 & $10 \times 10$ & 930 & 1097 & 1165 & 2621 & 1673 & & 1085 & 1133 & 2304 & 1669 \\
FT20 & $20 \times 5$ & 1165 & 1431 & 1515 & 3042 & 2018 & & 1343 & 1407 & 2939 & 2001 \\
LA01 & $10 \times 5$ & 666 & 666 & 688 & 1832 & 1020 & & 666 & 674 & 1571 & 999 \\
LA02 & $10 \times 5$ & 655 & 699 & 731 & 1671 & 1031 & & 669 & 729 & 1537 & 1022 \\
LA03 & $10 \times 5$ & 597 & 626 & 654 & 1621 & 955 & & 619 & 651 & 1426 & 940 \\
LA04 & $10 \times 5$ & 590 & 614 & 630 & 1697 & 944 & & 598 & 634 & 1523 & 938 \\
LA05 & $10 \times 5$ & 593 & 593 & 593 & 1497 & 824 & & 593 & 593 & 1456 & 820 \\
LA06 & $15 \times 5$ & 926 & 926 & 933 & 2198 & 1301 & & 926 & 926 & 2022 & 1291 \\
LA07 & $15 \times 5$ & 890 & 930 & 968 & 2218 & 1344 & & 893 & 931 & 1917 & 1335 \\
LA08 & $15 \times 5$ & 863 & 885 & 929 & 2151 & 1316 & & 863 & 907 & 2001 & 1310 \\
LA09 & $15 \times 5$ & 951 & 951 & 964 & 2474 & 1370 & & 951 & 954 & 2221 & 1359 \\
LA10 & $15 \times 5$ & 958 & 958 & 958 & 2129 & 1303 & & 958 & 958 & 1993 & 1297 \\
LA11 & $20 \times 5$ & 1222 & 1234 & 1264 & 2641 & 1694 & & 1222 & 1231 & 2501 & 1691 \\
LA12 & $20 \times 5$ & 1039 & 1039 & 1075 & 2335 & 1487 & & 1039 & 1059 & 2194 & 1487 \\
LA13 & $20 \times 5$ & 1150 & 1150 & 1185 & 2602 & 1651 & & 1150 & 1154 & 2475 & 1626 \\
LA14 & $20 \times 5$ & 1292 & 1292 & 1292 & 2884 & 1687 & & 1292 & 1292 & 2396 & 1677 \\
LA15 & $20 \times 5$ & 1207 & 1315 & 1348 & 2762 & 1838 & & 1298 & 1357 & 2652 & 1821 \\
\bottomrule
\end{tabular}
\end{sidewaystable}

\cleardoublepage
\begin{sidewaystable}
\caption{Resultados do caso de experimento 27}
\centering
\label{experimento27}
\begin{tabular}{cccccccccccc}
\toprule
& & & \multicolumn{4}{c}{GA/EV-JB} & & \multicolumn{4}{c}{IVF/EV-JB} \\
\cmidrule(c){4-7}
\cmidrule(c){9-12}
Inst\^{a}ncia & Tamanho ($n \times m$) & BKS & Melhor & M\'{e}dia (melhores) & Pior & M\'{e}dia (pop.) & & Melhor & M\'{e}dia (melhores) & Pior & M\'{e}dia (pop.) \\
\midrule
FT06 & $6 \times 6$ & 55 & 55 & 55 & 145 & 82 & & 55 & 55 & 134 & 82 \\
FT10 & $10 \times 10$ & 930 & 1111 & 1145 & 2573 & 1671 & & 1082 & 1163 & 2383 & 1665 \\
FT20 & $20 \times 5$ & 1165 & 1420 & 1493 & 3025 & 2025 & & 1355 & 1474 & 2759 & 2012 \\
LA01 & $10 \times 5$ & 666 & 666 & 693 & 1790 & 1021 & & 666 & 679 & 1584 & 1003 \\
LA02 & $10 \times 5$ & 655 & 713 & 749 & 1715 & 1029 & & 681 & 707 & 1514 & 1028 \\
LA03 & $10 \times 5$ & 597 & 649 & 673 & 1645 & 948 & & 618 & 646 & 1448 & 938 \\
LA04 & $10 \times 5$ & 590 & 619 & 651 & 1648 & 959 & & 616 & 631 & 1476 & 940 \\
LA05 & $10 \times 5$ & 593 & 593 & 593 & 1482 & 823 & & 593 & 593 & 1358 & 821 \\
LA06 & $15 \times 5$ & 926 & 926 & 938 & 2228 & 1307 & & 926 & 926 & 1929 & 1299 \\
LA07 & $15 \times 5$ & 890 & 936 & 952 & 2096 & 1349 & & 907 & 958 & 1982 & 1341 \\
LA08 & $15 \times 5$ & 863 & 886 & 915 & 2182 & 1318 & & 863 & 907 & 1976 & 1310 \\
LA09 & $15 \times 5$ & 951 & 951 & 970 & 2299 & 1373 & & 951 & 951 & 2031 & 1362 \\
LA10 & $15 \times 5$ & 958 & 958 & 958 & 2079 & 1307 & & 958 & 958 & 2050 & 1299 \\
LA11 & $20 \times 5$ & 1222 & 1230 & 1273 & 2631 & 1703 & & 1222 & 1238 & 2461 & 1690 \\
LA12 & $20 \times 5$ & 1039 & 1039 & 1067 & 2430 & 1487 & & 1039 & 1048 & 2218 & 1484 \\
LA13 & $20 \times 5$ & 1150 & 1150 & 1200 & 2556 & 1637 & & 1150 & 1170 & 2311 & 1640 \\
LA14 & $20 \times 5$ & 1292 & 1292 & 1292 & 2795 & 1691 & & 1292 & 1292 & 2425 & 1680 \\
LA15 & $20 \times 5$ & 1207 & 1342 & 1375 & 2927 & 1844 & & 1315 & 1351 & 2584 & 1831 \\
\bottomrule
\end{tabular}
\end{sidewaystable}

\cleardoublepage

\section{Efetividade da solu\c{c}\~{a}o}

Pelos 27 casos de experimentos \'{e} poss\'{i}vel avaliar que ambas as solu\c{c}\~{o}es desenvolvidas s\~{a}o solu\c{c}\~{o}es de qualidade por
apresentarem respostas pr\'{o}ximas ou semelhantes \`{a}quelas encontradas pela literatura. Embora em alguns casos de experimento as solu\c{c}\~{o}es
n\~{a}o forne\c{c}am a melhor solu\c{c}\~{a}o conhecida (\textit{BKS}), ainda assim obtemos uma resposta relevante. Nos experimentos 1 (tabela
\ref{experimento1}), 2 (tabela \ref{experimento2}), 3 (tabela \ref{experimento3}) e 19 (tabela \ref{experimento19}) obtemos respostas pr\'{o}ximas
ou semelhantes \`{a} melhor solu\c{c}\~{a}o conhecida em mais de 85\% das inst\^{a}ncias exercitadas. Nos experimentos 6 (tabela \ref{experimento6}),
7 (tabela \ref{experimento7}), 8 (tabela \ref{experimento8}), 9 (tabela \ref{experimento9}), 10 (tabela \ref{experimento10}), 13 (tabela
\ref{experimento13}), 22 (tabela \ref{experimento22}), 24 (tabela \ref{experimento24}), 25 (tabela \ref{experimento25}), 26 (tabela
\ref{experimento26}) e 27 (tabela \ref{experimento27}) o IVF/EV-JB obt\'{e}m ao menos 80\% melhores resultados frente ao GA/EV-JB. Nos experimentos
20 (tabela \ref{experimento20}) e 21 (tabela \ref{experimento21}) h\'{a} uma perda na qualidade das solu\c{c}\~{o}es encontradas pelo IVF/EV-JB
frente ao GA/EV-JB. Nos experimentos 11 (tabela \ref{experimento11}), 12 (tabela \ref{experimento12}), 14 (tabela \ref{experimento14}), 15 (tabela
\ref{experimento15}), 16 (tabela \ref{experimento16}), 17 (tabela \ref{experimento17}), 18 (tabela \ref{experimento18}) e 23 (tabela
\ref{experimento23}) o IVF/EV-JB perde grande parte da sua efetividade e obt\'{e}m ao menos 55\% melhores resultados frente ao GA/EV-JB.

Analisando os piores resultados desenvolvidos em cada solu\c{c}\~{a}o, o IVF/EV-JB se mostra uma solu\c{c}\~{a}o inicialmente gulosa, por ter os
menores valores de pior indiv\'{i}duo gerado nas execu\c{c}\~{o}es; contudo, esse comportamento n\~{a}o demonstra ser prop\'{i}cio a tornar a
solu\c{c}\~{a}o vulner\'{a}vel em \'{o}timos locais. A m\'{e}dia populacional do IVF/EV-JB juntamente os valores de pior indiv\'{i}duo corroboram
com a motiva\c{c}\~{a}o do algoritmo auxiliar paralelo (AAP) em manter qualidades genot\'{i}picas na popula\c{c}\~{a}o e prover resultados de
qualidade.

\section{Influ\^{e}ncia dos operadores de variabilidade e de sele\c{c}\~{a}o sobre a efetividade da solu\c{c}\~{a}o}

Pelos resultados obtidos n\~{a}o \'{e} poss\'{i}vel identificar quais s\~{a}o as reais influ\^{e}ncias dos operadores de variabilidade e
sele\c{c}\~{a}o sobre a efetividade da solu\c{c}\~{a}o. Contudo, \'{e} poss\'{i}vel aferir que as melhores configura\c{c}\~{o}es das solu\c{c}\~{o}es
propostas foram as configura\c{c}\~{o}es dos experimentos 1 (tabela \ref{experimento1}), 2 (tabela \ref{experimento2}), 3 (tabela \ref{experimento3}),
25 (tabela \ref{experimento25}), 26 (tabela \ref{experimento26}) e 27 (tabela \ref{experimento27}); as piores configura\c{c}\~{o}es est\~{a}o nos
experimentos 16 (tabela \ref{experimento16}), 17 (tabela \ref{experimento17}) e 18 (tabela \ref{experimento18}).

Pelos resultados n\~{a}o \'{e} poss\'{i}vel aferir qual \'{e} o melhor ou pior operador de variabilidade ou de sele\c{c}\~{a}o, mas qual a
configura\c{c}\~{a}o \'{e} mais prop\'{i}cia a gerar resultados de qualidade, em conson\^{a}ncia com os desenvolvimentos da literatura a respeito dos
algoritmos evolucion\'{a}rios e como uma solu\c{c}\~{a}o metaheur\'{i}stica nestes par\^{a}metros deve ser projetada \cite{DeJong2006}. \'{E}
poss\'{i}vel identificar que a sele\c{c}\~{a}o por ranqueamento linear (com alta press\~{a}o seletiva \cite{DeJong2006} \cite{Engelbrecht2007})
prov\^{e} resultados de qualidade quando combinada \`{a} muta\c{c}\~{a}o por permuta\c{c}\~{a}o; e tamb\'{e}m poss\'{i}vel verificar a partir dos
experimentos que a sele\c{c}\~{a}o por torneio (com m\'{e}dia press\~{a}o seletiva \cite{DeJong2006} \cite{Engelbrecht2007}) prov\^{e} resultados
de qualidade quando combinada \`{a} muta\c{c}\~{a}o por permuta\c{c}\~{a}o juntamente com a gera\c{c}\~{a}o aleat\'{o}ria de indiv\'{i}duos.

Alguns autores defendem que mecanismos de sele\c{c}\~{a}o de alta press\~{a}o seletiva devem ser combinados com mecanismos que geram grande
variabilidade a fim de obter-se resultados de qualidade \cite{DeJong2006}. Por esta considera\c{c}\~{a}o, ser\'{a} v\'{a}lido afirmar que o mecanismo
de muta\c{c}\~{a}o por permuta\c{c}\~{a}o \'{e} um operador de variabilidade que gera grande variabilidade em n\'{i}vel fenot\'{i}pico, tornando
os indiv\'{i}duos--solu\c{c}\~{o}es aptos a sofrer maior press\~{a}o seletiva. A gera\c{c}\~{a}o aleat\'{o}ria de indiv\'{i}duos, conforme
demonstrados nos experimentos 25 (tabela \ref{experimento25}), 26 (tabela \ref{experimento26}) e 27 (tabela \ref{experimento27}), aparentemente
diminui os efeitos da muta\c{c}\~{a}o por permuta\c{c}\~{a}o e tornam os indiv\'{i}duos--solu\c{c}\~{o}es aptos a sofrer uma menor press\~{a}o
seletiva.

\chapter{Conclus\~{a}o}
\label{conclusao}

A partir de uma classe de problemas de otimiza\c{c}\~{a}o --- problemas de escalonamento job-shop ---, foi realizada a an\'{a}lise do algoritmo
auxiliar paralelo (AAP) sobre uma solu\c{c}\~{a}o baseada em algoritmos gen\'{e}ticos, mas com influ\^{e}ncias de outras abordagens
evolucion\'{a}rias. Al\'{e}m desta an\'{a}lise central, foi tamb\'{e}m realizada uma an\'{a}lise dos efeitos que operadores de sele\c{c}\~{a}o
e de variabilidade (recombina\c{c}\~{a}o e muta\c{c}\~{a}o) influem sobre as solu\c{c}\~{o}es--respostas obtidas por estes algoritmos.

Foi poss\'{i}vel constastar a relativa efetividade e efici\^{e}ncia das solu\c{c}\~{o}es constru\'{i}das a partir de 27 casos de experimentos,
o que contribui para enaltecer a import\^{a}ncia das metaheur\'{i}sticas, sobretudo dos algoritmos evolucion\'{a}rios, como m\'{e}todos de
solu\c{c}\~{a}o para problemas em otimiza\c{c}\~{a}o. Os casos de experimentos constatam que a partir de configura\c{c}\~{o}es, an\'{a}lises e
ajustes obt\^{e}m-se resultados de qualidade com algoritmos conceitualmente simples e de f\'{a}cil aplicabilidade, mesmo em classes de problemas
de dif\'{i}cil modelagem matem\'{a}tica, como \'{e} o caso dos problemas de escalonamento job-shop \cite{French1982}.

O AAP se consolida como uma proposta de melhoria n\~{a}o s\'{o} em algoritmos gen\'{e}ticos, mas, tamb\'{e}m, em outras classes de algoritmos
evolucion\'{a}rios. Obteve-se uma resposta positiva do AAP frente \`{a} sua contrapartida nos experimentos. Al\'{e}m disso, constatou-se a
import\^{a}ncia da escolha dos mecanismos de variabilidade e de sela\c{c}\~{a}o em uma solu\c{c}\~{a}o baseada em algoritmos evolucion\'{a}rios.

\section{Trabalhos futuros}

O presente trabalho realizou a aplica\c{c}\~{a}o de uma extens\~{a}o dos algoritmos gen\'{e}ticos --- a saber, o algoritmo auxiliar
paralelo (AAP) baseado na fertiliza\c{c}\~{a}o in vitro --- como solu\c{c}\~{a}o para problemas de escalonamento job-shop. Em seguida,
foi feito um comparativo com uma vers\~{a}o dita can\^{o}nica com a finalidade de evidenciar os efeitos do AAP sobre a solu\c{c}\~{a}o
metaheur\'{i}stica e suas influ\^{e}ncias nas solu\c{c}\~{o}es obtidas.

Possibilidades de expans\~{a}o do presente trabalho seria aplicar o AAP em outras classes de algoritmos evolucion\'{a}rios, como as estrat\'{e}gias
evolutivas, ou em outras abordagens metaheur\'{i}sticas baseadas num conjunto de solu\c{c}\~{o}es--indiv\'{i}duos, ou popula\c{c}\~{o}es de
indiv\'{i}duos (``busca adaptativa paralela'', como \textit{particle swarm optimization}). O AAP tem por base conceitos simples, como um indiv\'{i}duo
representado genot\'{i}pica ou fenotipicamente; mecanismos de gera\c{c}\~{a}o de novas solu\c{c}\~{o}es--indiv\'{i}duos e de variabilidade destas
solu\c{c}\~{o}es. Portanto, a aplicabilidade do AAP em outras classes de algoritmos evolucion\'{a}rios garante-se pois este \'{e} de f\'{a}cil
implementa\c{c}\~{a}o, necessitando apenas de conceitos inerentes aos algoritmos evolucion\'{a}rios.

Outra expans\~{a}o poss\'{i}vel \'{e} realizar um comparativo mais ostensivo do AAP frente a outras abordagens h\'{i}bridas, como os
\textit{immune genetic algorithms}.

\cleardoublepage
\nocite{*}
\arial
\bibliography{../bibliograph}

\apendices

\end{document}

